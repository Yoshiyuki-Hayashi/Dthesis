\chapter{Method}
\label{method}
\section{Synthesis}
We prepared two samples with different quality.
The first one was revealed to be low-quality with substaintial defects after intensive measurements (Fig.\ref{lowQ}).
Thus, we started the synthesis by ourselves to clarify the physics of $\beta$-Li$_2$IrO$_3$ with high-quality crystals.

There is a suggestion that crystal growth of $\alpha, \beta$-Li$_2$IrO$_3$ is reaction between IrO$_3$ gas and LiOH gas \cite{Freund2016}.
The reaction is as follows.
First, the oxidization,
\begin{align}
\mathrm{Ir} + \mathrm{O_2} &\rightarrow \mathrm{IrO_2},\\
\mathrm{2Li} + \mathrm{\frac{1}{2}O_2} &\rightarrow \mathrm{Li_2O},
\end{align}
is necessary.
Then, IrO$_3$ gas and LiOH gas are generated in the equivalnece of
\begin{align}
\mathrm{IrO_2 (s)} + \mathrm{\frac{1}{2}O_2} &\rightleftharpoons \mathrm{IrO_3 (g)},\\
\mathrm{Li_2O (s)} + \mathrm{H_2O (g)} &\rightleftharpoons \mathrm{2LiOH (g)}.
\end{align}
Finally, the gases reaact,
\begin{align}
\mathrm{IrO_3 (g)} + \mathrm{2LiOH (g)} &\rightarrow \mathrm{Li_2IrO_3 (s)} + \mathrm{H_2O (g)} + \mathrm{\frac{1}{2}O_2}.
\end{align}

Considering it, we adopted a vapor-transport-like method using H$_2$O and O$_2$ as agent.
In the very early stage for synthesis, we used Li and Ir.
It was a simple synthesis where Li and Ir are put together without mixing and heated.
Since the initial low-quality crystal had hexagonal shape, we tried to grow the hexagonal crystal.
However, later, the hexagonal crystal was revealed to be $\alpha$-Li$_2$IrO$_3$.
(Initial hexagonal low-quality crystal was at least $\beta$-Li$_2$IrO$_3$.)
Then, we switched the Li supplyer from Li to LiOH$\cdot$H$_2$O.
We obtained rhombic crystals and they were $\beta$-Li$_2$IrO$_3$.
Since the difference between Li and LiOH$\cdot$H$_2$O is considered as the amount of water vapor, we induced the humid air flow into furnace.
The rhombic $\beta$-Li$_2$IrO$_3$ crystal become bigger, but the quality of crystal was still low.
It shows the magnetic transition point at $T_\mathrm{mag} = 38$ K, but the low-$T$ Curie tail coming from defects was still substaintially large.
Then, we optimized the time for inducing humid air flow.
With the optimization, low-$T$ Curie tail was minimized.
However, the size of crystal was still too small $\sim 0.2$ mm.
Then, we switched the Li supplyer from LiOH$\cdot$H$_2$O to Li$_2$CO$_3$.
Thus, finally, after 60 trials, we could obtain the large high-quality crystal of $\beta$-Li$_2$IrO$_3$.

The maximum dimension of $\beta$-Li$_2$IrO$_3$ single crystal was 0.69 $\times$ 0.45 $\times$ 0.56 mm (b $\times$ a$\times$ c-axes) (Fig. \ref{crystal}).

%Fig. photo of crystal
\begin{figure}
  \centering
  \includegraphics[scale=0.7]{crystal.png}
  \caption{$\beta$-Li$_2$IrO$_3$ single crystal.
  (a) Top view from c axis.
  (b) Side view.}
  \label{crystal}
\end{figure}

The detail of optimized synthesis condition is as follows.
Ir (99.9\%, 100 $\mu m$ mesh, rare metallic) of 0.5 g was placed in a alumina crucible (SSA-S) of $\phi$ 14 mm $\times$ 38 mm and
Li$_2$CO$_3$ (99.999\%, rare metallic) with 5 \% shortage of Li from the stoichiometric ratio of Li/Ir
was just put on the Ir without mixing by mortar (Fig. \ref{synthesis} (a)).
The crucible with loose lid was further placed in the larger crucible with loose lid.
The setup was heated in humid air with the sequence of
4 h up to 880$^\circ$C, 1 h up to 1000$^\circ$C, 24 h up to 1050$^\circ$C, 24 h down to 1000$^\circ$C (Fig. \ref{synthesis} (b)).
From 1000$^\circ$C, the crucible was in dry air with the sequence of 24 h down to 950$^\circ$C and 5 h down to room temperature.
Humid air was made by flowing the dry air through de-ionized water (Fig.\ref{flow}).
The air flowed from the bottom to the top in the order-made airtight vertical tube furnace.
%aim of the each sequence
The original aim of each sequence in Fig. \ref{synthesis} (b) is as follows;
The one-hour sequence from 880$^\circ$C to 1000$^\circ$C is done for the oxidization of Ir.
The 24-hour sequence from 1000$^\circ$C to 1050$^\circ$C is done for the creation of seed crystals of $\beta$-Li$_2$IrO$_3$.
The 48-hour sequence from 1050$^\circ$C to 950$^\circ$C is for the crystal growth.
However, especially, the meanings of the one-hour sequence is almost nothing but the tracing of the initial success.

%picture of setup and sequence
\begin{figure}
  \centering
  \includegraphics[scale=0.7]{synthesis.png}
  \caption{(a) Setup for the synthesis.
  The reagents are placed in alumina crucible doubly.
  The dimension is in mm.
  Air flows from the bottom to the top.
  (b) Sequence of furnace for the synthesis.
  RT means room temperature.}
  \label{synthesis}
\end{figure}

\begin{figure}
  \centering
  \includegraphics[scale=0.7]{air_condition.png}
  \caption{Flow chart for (a) humid air condition
  and (b) dry air condition.}
  \label{flow}
\end{figure}

The large crystals are mainly obtained from the top surface of the resulting compound with surrounded by hard Li-Ir oxide,
as if they were buried fossils.

Although the shape of the crystal, magnetic anisotropy and magnetic transition point are similar to $\gamma$-Li$_2$IrO$_3$ \cite{modic2014realization},
there are differences in extinction rule for (hh0) Bragg peaks and in the absolute value of magnetic susceptibility.
We confirmed that our crystal is truly $\beta$-Li$_2$IrO$_3$ through single-crystal X-ray diffraction with Bruker D8 Discover (Bruker AXS)
and magnetic susceptibility measurement with Magnetic Property Measurement System (MPMS) (Quantum Design).

The magnetic transition point of our crystal is 38 K, which is the same as reported \cite{ruiz2017correlated}.

\section{Measurement of magnetic susceptibility}
We measured magnetic susceptibility along b-axis of $\beta$-Li$_2$IrO$_3$ single crystals (0.587 mg as a whole)
by MPMS in opposed-anvil-type high-pressure cell (mCAC cell, Fig.\ref{tateiwa_cell} \cite{Tateiwa2011}) up to 1.4 GPa and down to 6 K at 1 T.
Pressure medium was Daphne 7575 (succesor of Daphne7474 \cite{Murata2008}).

The advantages of this method are mainly three points \cite{Tateiwa2011}.
(i) The cell has low background magnetization.
It enables us to measure not only high-pressure ferromagnetism but also antiferromagnetism under pressure.
(ii) The cost to manufacture the cell is lower than diamond anvil cell.
(iii) It is easy to use for the beginners of high-pressure technique.

The detail technique is as follows.
For the background measurement, first we attached the lower zilconia anvil and CuBe gasket
with small amount of GE varnish in high-pressure cell for the later easier treatment of setup.
Then we put tiny piece of Pb as manometer and Daphne 7575 as pressure medium.
The setup was cloesd with upper zilconia anvil and the background was measured (Fig.\ref{chi_meas_setup} (a)).

After the background measurement, setup was decomposed.
The three crystals were attached to be oriented to b-axis with small amount of GE varnish and the crystals are fixed at the center of lower anvil with GE varnish
(Fig. \ref{chi_setup}).
Then again we attached the lower anvil and gasket with GE varnish, put the same Pb piece and Daphne 7575 and measured the magnetization (Fig.\ref{chi_meas_setup} (b)).
The background was directly substracted in the scan data for each temperature.

With the above method, the ambiguity arises in the background substraction for the difference in small amount of GE varnish, but it can be neglected in our measurement.

The gasket was broken at 1.7 GPa.
After the measurement, the top edge of b-axis was scratched, but the orientation was maintained.

We summarized the dimension of the setup in Fig.\ref{chi_setup} (a).

\begin{figure}
  \centering
  \includegraphics[scale=0.6]{tateiwa_cell.png}
  \caption{The over all view of the high-pressure cell \cite{Tateiwa2011}.}
  \label{tateiwa_cell}
\end{figure}

\begin{figure}
  \centering
  \includegraphics[scale=0.6]{mag_suscep_meas.png}
  \caption{(a) The setup for background measurement
  (b) The setup for the magnetic susceptibility measurement}
  \label{chi_meas_setup}
\end{figure}

%photo of setup
\begin{figure}
  \centering
  \includegraphics[scale=0.6]{chi_setup.png}
  \caption{(a) The dimension of the anvil and gasket. The unit is in mm.
  (b) Photos of the setup for the magnetic susceptibility measurement under high pressure (top view).
  Three crystals are attached with GE varnish and it is glued on lower zilconia anvil.
  Field $B$ is parallel to b axis.
  The dimension is in mm.
  (c) Side view of (b).}
  \label{chi_setup}
\end{figure}

\newpage
\section{Principles of nuclear magnetic resonance (NMR)}
%explanation of NMR

\begin{figure}[h]
  \centering
  \includegraphics[scale=0.7]{nmr.png}
  \caption{Schematic image for the situation of NMR measurement.
  By measuring the Zeeman levels of nuclear spins, we can know the magnetism of electrons through hyperfine field.
  $H_0$ is applied static field for the Zeeman level formation.
  $H_1(t)$ is radio wave with the energy of $\sim\hbar\omega = \gamma\hbar H_0$ to manipulate the direction of nuclear moments.
  Here, $\gamma$ is gyromagnetic ratio of the nuclear spin.}
  \label{nmr}
\end{figure}

Nuclear spins in solid interact with electrons.
The interaction is called hyperfine interaction.
It can be described as the field applied at the nuclear site, which is called hyperfine field (Fig.\ref{nmr}).
It consists of dipolar field of electron spin, the field induced by orbital motion of electrons, and Fermi contact term.
Here, Fermi contact term comes from the finite overlap of electron wave function at the nuclear spin site.

By applying magnetic field, the degenerate levels of the nuclear spin splits to Zeeman levels.
The hyperfine field gives the deviation in Zeeman levels or induce the transition between Zeeman levels.

If we apply the radio wave (pulse) whose frequency corresponds to the Zeeman level splitting, the absorption of radio wave occurs (Fig.\ref{nmr}).
In Nuclear Magnetic Resonance (NMR) measurement, we utilized this phenomenon and manipulate the coherent motion of the total nuclear magnetization.
With the manipulation, we can obtain the information about the Zeeman level.
Since it contains the effect of the hyperfine field, we can know the magnetism of electrons via NMR.

First, we will present the motion of free nuclear spin under magnetic field in order to explain the experimental manipulation of total nuclear magnetization.
Then, we will explain what we can measure in NMR --- NMR spectrum, spin-spin relaxation time $T_2$, Knight shift $K$, and spin-lattice relaxation rate $T^{-1}_1$.

The overall discussion in this section is based on Ref.\cite{Kitaoka, Asayama, Takigawa}.

\subsection{Free nuclear spin under field}
Suppose we have a free nucleus with spin $\hbar\vec{I}$ and gyromagnetic ratio $\gamma$.
The magnetic moment $\vec{\mu}$ of nucleus is written as,
\begin{equation}
\vec{\mu} = \gamma\hbar\vec{I}.
\end{equation}
If we apply the static field along $z$ direction, $\overrightarrow{H_0} = H_0\vec{e_z}$, the nuclear moment interacts with the field.
The Hamiltonian for the Zeeman interaction is,
\begin{equation}
\mathcal{H} = -\vec{\mu}\cdot\overrightarrow{H_0}.
\end{equation}
The motion of the nuclear moment is described by equation of motion in Heisenberg picture,
\begin{equation}
\frac{\mathrm{d}\vec{\mu}}{\mathrm{d}t} = \frac{i}{\hbar}[\mathcal{H},\vec{\mu}] = \gamma\vec{\mu}\times\overrightarrow{H_0}.
\label{eom_H0}
\end{equation}
Here, commutation relations for spin components, $[I^x, I^y] = iI^z$ etc. were used.
The solution of Eq.(\ref{eom_H0}) is
\begin{align}
\mu^x (t) &= \mu^x(0)\cos\omega_0t + \mu^y(0)\sin\omega_0t,\\
\mu^y (t) &= \mu^y(0)\cos\omega_0t - \mu^x(0)\sin\omega_0t,\\
\mu^z (t) &= \mu^z(0),
\end{align}
where we set
\begin{equation}
\omega_0 = \gamma H_0
\end{equation}
This is a precession motion (Fig.\ref{precession1}(a)).
If we view it from $+z$ direction , it rotates clockwise.

\begin{figure}
  \centering
  \includegraphics[scale=0.7]{precession1.png}
  \caption{(a) Precession motion of nuclear moment under static field $H_0$ in laboratory coordinate system.
  It is clockwise with the angular velocity $\gamma H_0$
  (b) Nuclear moments doesn't move in the clockwise rotating coordinate system with angular velocity $\gamma H_0$.}
  \label{precession1}
\end{figure}

Then, let's change the coordinate system to the rotating coordinate system.
The rotating coordinate system shares the origin and $z$ axis with the original laborarory coordinate system and it rotates
around $z$ axis with angular velocity of $\vec{\omega}$.
We name the coordinate axes fixed to the rotating system as $X, Y, Z$ axes.
$Z$ axis is the same as $z$ axis.
In the rotating system, the equation of motion of $\vec{\mu}$ can be written as,
\begin{equation}
\frac{\delta\vec{\mu}}{\delta t} = \gamma\vec{\mu}\times\left(\overrightarrow{H_0} + \frac{\vec{\omega}}{\gamma}\right),
\end{equation}
where $\frac{\delta}{\delta t}$ is a derivative in rotating coordinate system.
If we choose
\begin{equation}
\vec{\omega} = -\gamma\overrightarrow{H_0}
\end{equation}
the equation of motion becomes
\begin{equation}
\frac{\delta\vec{\mu}}{\delta t} = 0
\end{equation}
It means that nuclear moment doesn't move in the rotating coordinate with $\vec{\omega} = -\gamma\overrightarrow{H_0}$ (Fig.\ref{precession1}(b)).
The rotation of cooridnate system is clockwise if we view it from $+z$ direction.

Suppose that the initial condition of spin component is
\begin{align}
\mu^x(0) = 0\\
\mu^y(0) = 0\\
\mu^z(0) = m.
\end{align}
In this case, the nuclear moment is fixed along $z$ direction and doesn't move under $\overrightarrow{H_0}$
both in the laboratory coordinate system and in the rotating coordinate system (Fig.\ref{precession2}(a)).
In this situation, let's apply the clockwise rotating magnetic field with the angular frequency $\omega_0$ perpendicular to $\overrightarrow{H_0}$,
\begin{equation}
\overrightarrow{H_1}(t) = H_1(\cos\omega_0t\vec{e_x} - \sin\omega_0t\vec{e_y}).
\end{equation}
In rotating coordinate system with $\vec{\omega} = -\gamma\overrightarrow{H_0}$ (clockwise), $\overrightarrow{H_1}(t)$ doesn't move.
Thus, we can set the direction of $\overrightarrow{H_1}(t)$ as $X$ axis.
The equation of motion becomes
\begin{equation}
\frac{\delta\vec{\mu}}{\delta t} = \gamma\vec{\mu}\times\overrightarrow{H_1}.
\label{eom_H1}
\end{equation}
Here, we omit the argument $t$ for $\overrightarrow{H_1}$ because it is time independent in the rotating coordinate system.
Eq.(\ref{eom_H1}) is equivalent to eq.(\ref{eom_H0}).
Therefore, the nuclear moment starts precession motion around $X$ direction with the angular frequency of $\omega_1 = \gamma H_1$ (Fig.\ref{precession2}(b)).
It is rotation within ZY plane and clockwise if we view it from $+X$ direction.

\begin{figure}
  \centering
  \includegraphics[scale=0.7]{precession2.png}
  \caption{(a) Initial situation.
  The nuclear moment is fixed along Z (or z) direction and doesn't move both in laboratory system and in rotating system.
  (b) If the rotating field $H_1 (t)$ is applied, nuclear moment starts precession around X axis with angular velocity $\gamma H_1$.}
  \label{precession2}
\end{figure}

This corresponds to applying radio wave or oscillating field with the energy equivalent to Zeeman level splitting $\hbar\omega_0$.
Oscillating field is a linear combination of clockwise rortaing field and counterclockwise roatating field.
In clockwise rotating coordinate system, the counterclockwise rotating component has a frequency of $2\omega_0$, which is too large compared to Zeeman level splitting.
Thus, the counterclockwise component doesn't affect the motion of nuclear moment.
It means that we can regard oscillating field as clockwise rotating field.

Static field $\overrightarrow{H_0}$ is generated by superconducting magnet.
Oscillating field $\overrightarrow{H_1}(t)$ can be made by applying the high-frequncy current to coil.
Thus, if we insert sample into coil and set the ocillating field of coil perpendicular to static field of the magnet,
we can manipulate the direction of magnetization of selected nuclei in the sample.

\subsection{Free induction decay (FID) and NMR spectrum}
If we apply pulse of the oscillating filed with the energy corresponding to $\hbar\omega_0$ only within the time of $\pi/(2\omega_1)$ ($\pi/2$ pulse),
the nuclear moments rotate clockwise by $\pi/2$ from $Z$ direction to $Y$ direction (Fig.\ref{FID}(a)).
In the rotating system, the moment doesn't move any more.
In the laboraory system, it rotates in xy-plane with angular frequency $\omega_0$.
The rotation of the nuclear moment generates induced electromotive force on the coil and it is detected as a voltage signal with high frequancy of $\omega_0$.
If the field is perfectly uniform and static, the oscillating signal doesn't decay and its Fourier transformed spectrum is delta function $\delta(\omega-\omega_0)$ (Fig.\ref{FID}(a)).
However, in reality, spatial distribution and dynamical fluctuation are induced by local field.
Here, local field is a general term which means the microscopic field at nuclear site, including hyperfine field by electrons or dipolar field by other nuclei.
With the local field, the resonance condition for each nuclear moment can be different.
It gives a width around $\omega = \omega_0$ in the Fourier transformed spectrum (Fig.\ref{FID}(b)).
This means that some nuclear spins with the resonance of $\omega_0 + \delta\omega$ cannot be static in rotating system with angular frequency of $\omega_0$.
As time passed, they rotates and phase coherence of nuclear spins disappears (Fig.\ref{FID}(b)).
In signal, many oscillating components different from $\omega_0$ comes in and the signal decay (Fig.\ref{FID}(b)).
We can also say that the decay or the Fourier transformed spectrum reflects the information of local field.
The decay is called Free Induction Decay (FID).
The time constant for the decay is called $T^*_2$ and the Fourier transformed spectrum is called NMR spectrum.

\begin{figure}
  \centering
  \includegraphics[scale=0.7]{FID.png}
  \caption{The effect of $\pi/2$ pulse in the case of (a) uniform and static local field, and (b) realistic local field.
  Fourier transformed spectrum and evolution of nuclear moments for each case are also shown.
  In case of (a), there is no decay in signal, but in (b), Free Induction Decay (FID) is induced.
  From NMR spectrum of FID signal, we can know the local field distribution.
  If the field is given as a constant value, NMR spectrum is obtained as a frequency-swept data.}
  \label{FID}
\end{figure}

The weak point of FID is that it is immediately successive to $\pi/2$ pulse.
In general, there is a insensitive time after pulse injection.
If the insensitive time is longer than $T^*_2$, we cannot observe FID signal.
There is a technique to overcome such a situation called spin echo method (section\ref{spin_echo}).


\subsubsection{NMR spectrum}
%explanation of NMR spectrum
Here, we explain NMR spectrum with the simple example.
Let's consider about commensurate collinear AF transition.
Fig.\ref{NMRspectrum} shows the schematic picture for the example.
Here, we only consider the dipolar field as a hyperfine field.
Also, we suppose that we measure NMR for the nuclei which have the electronic spins and that the nuclear site is crystallographically one.
Above the transition point, $T_{AF}$, the system is in paramagntic state, so there is only one kind of hyperfine field (Fig.\ref{NMRspectrum}(a)).
It means one peak in NMR spectrum.
On the other hand, below $T_{AF}$, antiferromagnetic up-down spin array gives two kinds of hyperfine field (Fig.\ref{NMRspectrum}(b)).
It gives two peaks in NMR spectrum.
In the case of incommensurate AF order, hte hyperfine field is distributed in a certain range (Fig.\ref{NMRspectrum}(c)).
In this sense, in spin glass state, it gives the complex many kinds of hyperfine field (Fig.\ref{NMRspectrum}(d)).
As a reult, NMR spectrum become broadened.

\begin{figure}
  \centering
  \includegraphics[scale=0.7]{NMRspectrum.png}
  \caption{Simplified example for the concept of NMR spectrum
  in (a) paramagnetic state, (b) commensurate collinear AF order,  (c) incommensurate AF order, (d) spin glass.
  The number of peaks and the width of NMR spectrum indicate the distribution of hyperfine field.
  $P(H)$ is NMR spectrum and $H_0$ is the field corresponding to the resonance condition $H_0 = \omega_0/\gamma$,
  where $\omega_0$ is applied frequency of radio wave and $\gamma$ is gyromagnetic ratio of the nuclear spin.
  $H^z_{hf}$ and $H'^z_{hf}$ is a strength of hyperfine field.
  If the frequency is given, NMR spectrum is obtained as a field-swept data.}
  \label{NMRspectrum}
\end{figure}


\subsection{Spin echo and spin-spin relaxation time $T_2$}
\label{spin_echo}
Fig.\ref{spin_echo} shows the pulse sequence for spin echo method.
Applying $\pi/2$ pulse gives FID.
Then, after time $\tau$ passed, if we apply $\pi$ pulse, the direction of nuclear moments is reversed.
It leads the re-convergence of spin direction after the same time $\tau$ passed.
The same situation just after the $\pi/2$ pulse injection is reproduced at $t = 2\tau$.
The signal is called spin echo.
Fourier transformation of spin echo signal also gives NMR spectrum.
Thus, we can measure the signal which is apart from the insensitive time of pulses.

If the local field is static, spin echo signal doesn't decay.
However, in realtiy, the local field is time-dependent.
The value of local field in $0 < t < \tau$ and in $\tau < t < 2\tau$ can be different.
It disturb the complete phase recovery at $t = 2\tau$.
Also, interaction between nuclear spins, e.g. dipor interaction, distrubs the phase coherence of the total nuclear spin motion.
Thus, the spin echo signal decays as a function of $2\tau$: $\sim \exp(-2\tau/T_2)$.
The time constant $T_2$ is called spin-spin relaxation time.
If $T_2$ is small and the decay is rapid, we cannot measure the spin echo signal.

\begin{figure}
  \centering
  \includegraphics[scale=0.7]{spin_echo.png}
  \caption{Principle of spin echo method.
  The pulse sequence and the corresponding schematic evolution of nuclear moments are shown.
  (a) $t = 0$, before applying $\pi/2$ pulse, (b) $t = \pi/(2\omega_1)$, just after applying $\pi/2$ pulse.
  (c) $t = \pi/(2\omega_1) + \tau$, just before applying $\pi$ pulse.
  (d) $t = \pi/\omega_1 + \tau$, just after applying $\pi$ pulse.
  (e) $t = 2\tau$, spin echo signal.}
  \label{spin_echo}
\end{figure}

\subsection{Knight shift $K$ and spin-lattice relaxation rate $T^{-1}_1$}
In Hamiltonian, the situation for NMR measurement (Fig.\ref{nmr}) can be written as,
\begin{align}
\mathcal{H}_{tot} &=\mathcal{H}_{nuc} + \mathcal{H}_{hf}+\mathcal{H}_{env},
\end{align}
where $\mathcal{H}_{tot}$ is Hamiltonian for the total system,
$\mathcal{H}_{nuc}$ is for nuclear spins of the same species to measure, $\mathcal{H}_{env}$ is for the environment,
$\mathcal{H}_{hf}$ represents the interaction between the nuclear spins and the environment through hyperfine field.
Here, we call the source of hyperfine field as the environment, which is usually itinerant electrons or localized electronic spins.
Nuclear spins are treated as a sum of free independent spins under applied static field $\overrightarrow{H_0} = H_0\overrightarrow{e_z}$
and radio wave (pulses) $\overrightarrow{H_1}(t)$,
\begin{align}
\mathcal{H}_{nuc} = -\sum^{N_{nuc}}_{i = 1}\vec{\mu}\cdot\left(\overrightarrow{H_0}+\overrightarrow{H_1}(t)\right),
\end{align}
where $N_{nuc}$ is the number of the nuclear spins and $\vec{\mu}$ represents the magnetic momenet of the measuring nuclear spin.
The interaction is written as,
\begin{align}
\mathcal{H}_{hf} = -\sum^{N_{nuc}}_{i = 1}\vec{\mu}\cdot\overrightarrow{H_{hf}}(i),
\end{align}
where hyperfine field at a certain nuclear site $\overrightarrow{r_i}$ is represented as
$\overrightarrow{H_{hf}}(i) \left(= \overrightarrow{H_{hf}}\left(\overrightarrow{r_i}, \overrightarrow{H_0}\right) \right)$

As a result, the total Hamitonian can be described as,
\begin{align}
\mathcal{H}_{tot} &= -\gamma\hbar\sum^{N_{nuc}}_{i = 1} \left(I^zH_0 + I^zH^z_{hf}(i)
 + \frac{1}{2}(I^+H^-_{hf}(i)+I^-H^+_{hf}(i))+\vec{I}\cdot\overrightarrow{H_1}(t)\right) \label{Htot}\\
 & + \mathcal{H}_{env} \notag
\end{align}
where $\gamma$ is a gyromagnetic ratio of the nuclear spin, which connects magnetic moment $\vec{\mu}$ and nuclear spin $\hbar\vec{I}$: $\vec{\mu} = \gamma\hbar\vec{I}$.
The first and fifth term in Eq.(\ref{Htot}) are what the experimentalists manipulate for NMR measurement as we saw in preceding sections.
With the maniplulation, we can measure the effect of the second, third, and forth term.

\subsubsection{Knight shift}
%explanation of Knight shift
By the first term in Eq.(\ref{Htot}), we can form Zeeman level (Fig.\ref{Knight_shift}).
The second term in Eq.(\ref{Htot}) gives the deviation in the Zeeman level.
The difference in the levels is described as,
\begin{align}
\gamma (H_0 + H^z_{hf}(i)) = \gamma (1 + K_i)H_0,
\end{align}
in the unit of frequency.
Here, the Knight shift,
\begin{align}
\label{K}
K_i = H^z_{hf}(i)/H_0,
\end{align}
is defined as the strength of hyperfine field along the applied field at the nuclear site $i$.
The definition of $K_i$ (Eq.(\ref{K})) we adopted here is too detail and a bit unusual.
We usually define Knight shift $K$ in paramagnetic phase only for the dominant hyperfine field that gives a peak shift in NMR spectrum.
In that sense, usual definition is
\begin{align}
\label{K_usual}
K = H^z_{hf, peak}/H_0,
\end{align}
where $H^z_{hf, peak}$ is the dominant hyperfine field that gives a peak shift in NMR spectrum.
The definition of $K_i$ (Eq.(\ref{K})) contains $K$ in Eq.(\ref{K_usual}).
The Knight shift is detected as a peak shift from the resonance of $\omega_0 = \gamma H_0$ in NMR spectrum (Fig.\ref{NMRspectrum}).

The Kinght shift represents the static magnetic response because it is propotional to magnetic susceptibility in high-temperature paramagntic phase as shown below.
Suppose that the hyperfine field at nuclear site $i$ is given by localized electronic spins:
\begin{align}
\label{Hhf}
H^\alpha_{hf}(i) = \sum^{N_s}_{j = 1}\sum_{\beta = x,y,z}A^{\alpha\beta}(i, j)\mu^\beta_e(j),
\end{align}
where $A^{\alpha\beta}$ is hyperfine tensor, $N_s$ is the number of the spins, and $\vec{\mu_e}(j)$ is the magnetic momnent of localized electronic spin at site $j$.
In paramagnetic phase, each magnetic moment is parallel to applied field:
\begin{align}
\label{PM_mu}
\vec{\mu_e} = m\vec{e_z}
\end{align}
where $m$ is the size of the moment.
By substituting Eq. (\ref{PM_mu}) to Eq. (\ref{Hhf}), Knight shift $K_i$ in Eq. (\ref{K}) is rewritten as.
\begin{align}
K_i = \frac{A_{hf}(i)}{N_A\mu_B}\chi_{mol},
\label{Kchi}
\end{align}
where $N_A$ is Avogadro number, $\mu_B$ is Bohr magnetron, $\chi_{mol}$ is a molar magnetic susceptibility: $\chi_{mol} H_0 = M_{mol} = N_A m$,
and $A_{hf}(i)$ is hyperfine field at nuclear site $i$ generated by magnetic moments of 1 $\mu_B$: $A_{hf}(i) = \sum^{N_s}_{j = 1}A^{zz}(i, j)\mu_B$.
Eq.(\ref{Kchi}) tells that we can experimentally determine the strength of the hyperfine field, $A_{hf}(i)$.
Suppose that we have the Knight shift, $K$, and magnetic susceptibility, $\chi$, in paramagnetic state.
By plotting $K$ as a function of $\chi$ at the corresponding temperature, we can obtain the strength of the hyperfine field, $A_{hf}(i)$, as a slope of the linear plot.
The plot is called $K-\chi$ plot.

\begin{figure}
  \centering
  \includegraphics[scale=0.7]{Knight_shift.png}
  \caption{Formation of Zeeman levels of nuclear spins in the case of I = 1/2.
  The deviation in Zeeman level induced by hyperfine field gives Knight shift.}
  \label{Knight_shift}
\end{figure}



\subsubsection{Spin-lattice relaxation rate, $T^{-1}_1$}
%explanation of T^{-1}_1
In thermal equilibrium state, nuclear moments distribute in Zeeman levels with Boltzmann distribution.
It has a corresponding nuclear magnetization.
Here, if we apply the fifth term in Eq.(\ref{Htot}), the radio wave, we can temporarily make non-equilibirum state with the deviated nuclear magnetization.
As the time passed, the magnetization recovers to the value of thermal equilibrium state through hyperfine field of the third and forth term in Eq.(\ref{Htot}).
The time constant for the relaxation process is called spin-lattice relaxation time $T_1$.
The inverse $T^{-1}_1$ is called spin-lattice relaxation rate, which corresponds the transition rate between the levels.

$T^{-1}_1$ is measured as follows (Fig.\ref{T1measurement}).
First, we apply a pulse to make all the nuclear moments be in $XY$ plane,
In general, the number of pulses can be more than two (Comb pulses).
Then, after the time $T^*_2$ passed, the phase coherence of nuclear moments disappears and we lose the signal.
However, after the further time $T$ passed, the nuclear magnetization starts recover to the thermal equilibirum state via hyperfine field.
We can measure the recovering magnetization $M(T)$ by spin echo method.
By changing time $T$ and measuring $M(T)$ for each time, we can obtain relaxation curve $M(t)$.
Finally, we can obtain $T^{-1}_1$ by fitting to the relaxation curve.
The fitting function for $I = 1/2$ nuclei is,
\begin{equation}
M (t) = A + M_0 (1 - \mathrm{e}^{-\frac{t}{T_1}}).
\end{equation}
The fitting function can be more complex if there are many relaxation processes.

\begin{figure}
  \centering
  \includegraphics[scale=0.7]{T1measurement.png}
  \caption{Pulse sequence for $T^{-1}_1$ measurement.
  The corresponding schematic evolution of nuclear moments are also shown.
  (a) $t < 0$, before applying comb pulses.
  (b) $t = 0$, just after applying comb pulses.
  (c) $t\sim T^*_2$, signal decays.
  (d) $t = T$, total nuclear magnetizaztion recovers via hyperfine field.}
  \label{T1measurement}
\end{figure}

$T^{-1}_1$ represents dynamic magnetic fluctuation at the measuring nuclear site because it is related to dynamical magnetic susceptibility as shown below.
Fermi golden rule gives the general expression of $T^{-1}_1$ by hyperfine field $H^{\pm}_{hf}$,
\begin{align}
\label{T1_Hhf}
T^{-1}_1 = \left(\frac{\gamma}{2}\right)^2\frac{1}{N_{nuc}}\sum^{N_{nuc}}_{i = 1}\int^{\infty}_{-\infty}\mathrm{d}t\mathrm{e}^{i\omega_0t}\braket{\{H^+_{hf}(i,t),H^-_{hf}(i,0)\}},
\end{align}
where $\hbar\omega_0$ is the energy difference of the two Zeeman levels, $\{A, B\} = AB + BA$,
$Q (t) = \mathrm{e}^{\frac{i}{\hbar}\mathcal{H}_{env}t}Q\mathrm{e}^{-\frac{i}{\hbar}\mathcal{H}_{env}t}$ for arbitariry operator $Q$,
and thermal average $\braket{}$ is taken by the environment, i.e.,
\begin{align}
\braket{Q} = \sum_{\nu}\frac{\mathrm{e}^{-\beta E_{\nu}}}{Z_{env}}\bra{\nu}Q\ket{\nu},
\end{align}
where $E_{\nu}$ is the energy of the state of the environment $\ket{\nu}$, and $Z_{env} = \sum_{\nu} \mathrm{e}^{-\beta E_{\nu}}$.
\begin{comment}
NMR relaxation rate $T^{-1}_1$ is propotional to thermal average of time-correlation function of spin components perpendicular
to the applied external magnetic field.
It can be related to dynamical susceptibility through
\end{comment}
Again suppose that we have a hyperfine field of Eq.(\ref{Hhf}).
If we assume that orthogonal matrix to diagonalize $A^{\alpha\beta}$ is independent from magnetic site index $j$ and
that, for the eigenvalues of $A^{\alpha\beta}$, $\lambda_x, \lambda_y, \lambda_z$, $\lambda_x = \lambda_y (= A^{\perp})$ holds,
we can derive,
\begin{align}
\label{Hhf_S}
H^\pm_{hf}(i) = -\gamma_e\hbar\sum^{N_s}_{j = 1}A^{\perp}(i, j)S^\pm(j),
\end{align}
where $\gamma_e$ is a gyromagnetic ratio of the electronic spin, which connects magnetic moment $\vec{\mu_e}$ and electronic spin $\hbar\vec{S}$: $\vec{\mu_e} = -\gamma_e\hbar\vec{S}$,
and we used the same notation for the vectors before and after the change of the basis for diagonalization.
Substituting Eq.(\ref{Hhf_S}) to Eq.(\ref{T1_Hhf}), we obtain
\begin{align}
\label{T1_Sq}
T^{-1}_1 = \left(\frac{\gamma}{2}\right)^2(\gamma_e\hbar)^2\sum_{\bm{q}}A^{\perp}_{\bm{q}}A^\perp_{-\bm{q}}\int^{\infty}_{-\infty}\mathrm{d}t\mathrm{e}^{i\omega_0t}
\braket{\{S^+_{\bm{q}}(t),S^-_{-\bm{q}}(0)\}},
\end{align}
where we assume that $A^\perp (i, j)$ is dependent only on relative position $\bm{r}_i - \bm{r}_j$, i.e., $A^\perp (i, j) = A^\perp(\bm{r}_i - \bm{r}_j)$,
and conducted Fourier transformation,
\begin{align}
A^\perp (i, j) &= A^\perp(\bm{r}_i - \bm{r}_j) = \frac{1}{\sqrt{N_s}}\sum_{\bm{q}}A^\perp_{\bm{q}}\mathrm{e}^{-\bm{q}\cdot(\bm{r}_i - \bm{r}_j)},\\
S^\pm (j) &= \frac{1}{\sqrt{N_s}}\sum_{\bm{q}} S^\pm_{\bm{q}}\mathrm{e}^{-\bm{q}\cdot\bm{r}_j}.
\end{align}
Applying fluctuation-dissipation theorem \cite{Kitaoka},
\begin{align}
\frac{1}{4}\int^{\infty}_{-\infty}\mathrm{d}t\mathrm{e}^{i\omega_0t} \braket{\{S^+_{\bm{q}}(t),S^-_{-\bm{q}}(0)\}}
= \frac{2\chi^{\prime\prime}(\bm{q}, \omega_0)}{(\gamma_e\hbar)^2(1-\mathrm{e}^{-\hbar\omega_0/k_BT})},
\end{align}
Eq.(\ref{T1_Sq}) is further transformed to
\begin{align}
\label{T1_chi}
T^{-1}_1 = \frac{2\gamma^2k_BT}{\hbar}\sum_{\bm{q}}A^{\perp}_{\bm{q}}A^\perp_{-\bm{q}} \frac{\chi^{\prime\prime}(\bm{q}, \omega_0)}{\omega_0},
\end{align}
where we assume $\hbar\omega_0 \ll k_BT$.
Here, $\chi^{\prime\prime}(\bm{q}, \omega)$ is imaginary part of dynamic magnetic susceptibility.


\section{Measurement of NMR}
We measured ${}^7$Li Nuclear Magnetic Resonance (NMR) of $\beta$-Li$_2$IrO$_3$ single crystal
in opposed-anvil cell (Fig.\ref{kitagawa_cell} \cite{Kitagawa2010}) up to 3.9 GPa and down to 5 K  at 4 T.
The anvils are made of WC.
The type of lower and upper anvils were IVc and 3, respectively.
The gasket was NiCrAl and VIIb type.
For the gasket, the age hardening was done in the sequence of 1 hour up to 760 ${}^\circ$C, 12 hours for maintaining 760 ${}^\circ$C and 1 hour down to room temperature.
The cell was No. 6.
Pressure medium was Daphne 7575 (succcesor of Daphne 7474 \cite{Murata2008}).
We used ruby-fluorescene method to determine the pressure.

The setup for high-pressure NMR is shown in Fig.\ref{NMR_highP_setup}.
Here, we describe how to make the setup.
First, we have to make a coil by hand.
It is prepared by winding Cu wire of $\phi$ 0.02 mm around stretched teflon sheet.
Then, two kinds of instant adhesives (Aron Alpha, Toagosei Co., Ltd. and Araldite, Huntsman Japan) is painted on the coil.
The adhesive doesn't react with teflon, so we can glue coil only.
The coil is soldered with two-fold braided Cu wire of $\phi$ 0.1 mm.
Then, the available frequency range of the coil should be checked.
The soldered points is covered by Araldite to avoid vacuum discharge.

\begin{figure}[H]
  \centering
  \includegraphics[scale=0.7]{kitagawa_cell.png}
  \caption{The over all view of high-pressure NMR cell \cite{Kitagawa2010}.}
  \label{kitagawa_cell}
\end{figure}

\begin{figure}
  \centering
  \includegraphics[scale=0.7]{NMR_highP_setup.png}
  \caption{Schematic view of setup for high-pressure NMR measurement.}
  \label{NMR_highP_setup}
\end{figure}

Next, we have to prepare the setup to measure ruby fluorescence.
We need to make a constriction to optical fiber by instataneously heating.
The constriction enhances the stability against pressure.
The constricted fiber is connected to moissanite with optically transpareant glue (phthalic glue, NIKKA SEIKO).
Moissanite is optically transparent, which is imitaion of diamond.

The coil and optical fiber with moissanite is wrapped by capton tube.
The setup is inserted to lower anvil.
Then, we need to fill the space with 'cement', which is made of diamond powder and stycast 1266 in the ratio of 3.75 : 1 (weight).
If we fill 70 \% of the space, we need to push down the moissanite.
After filling the rest, we have to make the surface of the cement smooth and flat.
Some pieces of ruby are fixed on moissanite with Araldite.
Also, we put Aron Alpha to glue the capton tube to the anvil and to glue the fiber and Cu wire to capton tube.
The whole setup is put on the heater at 40 ${}^\circ$C for 14 hours.

After the cement become dried and hard, we have to paint the surface of the cement with coating, which is made of Araldite and appropriate amount of diamond powder.
The coating is also painted on the upper anvil and inside of the gasket.
Thus, we can obtain the setup in Fig.\ref{NMR_highP_setup}.

At this stage, we need to insert the $\beta$-Li$_2$IrO$_3$ crystal into the coil.
However, since the coil is larger than the crystal, we cannot confirm the crystal direction after inserting.
To overcome this, $\beta$-Li$_2$IrO$_3$ crystal was attached to small piece of weighing paper with GE varnish (Fig.\ref{crystal_setup}).
Weighing paper was cut in rectangular shape.
The direction parallel to shorter edge of rectangle is b-axis and the direction perpendicular to paper is c-axis.
Since the paper exceed the legth of coil, we can check the crystal direction through the paper even after the inserting to coil.

\begin{figure}
  \centering
  \includegraphics[scale=0.7]{crystal_setup.png}
  \caption{Schematic view of setup for crystal to insert the coil.}
  \label{crystal_setup}
\end{figure}

The rest procedure is to assemble the anvils and gasket, put Daphne 7575 oil as pressure medium and pressurize.
The photo of the setup just before pressurizing is shown in Fig.\ref{NMR_highP_setup_photo}.
It is important to put the marker on the cell, which indicates the direction of oscillating field of coil.

\begin{figure}
  \centering
  \includegraphics[scale=0.7]{NMR_highP_setup_photo.png}
  \caption{Photo of setup for high-pressure NMR measurement.}
  \label{NMR_highP_setup_photo}
\end{figure}

Pressurizig was done at room temperature with monitoring ruby fluorescence.
Then, the whole setup was gradually cooled down to liquid nitrogen temperature.
We confirmed the stability of pressure at liquid nitrogen temperature.
After that, we also checked the pressure at room temperature.
Then, high-pressure cell is ready to set NMR probe.

In the NMR probe, we have to check the motion of two-axis rotator.
The high-pressure cell should be set properly in order to avoid the breaking of the Cu wire and optical fiber during rotation.
After manually orienting the b-axis to be parallel to field direction, now it is ready to start the measurement.

Ambient-pressure NMR at 4 T and at 1 T were also measured.

To adjust crystal direction, we measured the angular dependence in a probe by two-axis rotator \cite{Kitagawa2010}.
The adjustment was done both for the measurement at ambient pressure and at high pressure.
The accuracy is roughly $\pm$ 5$^\circ$ (Appendix \ref{appendix_simu}).
The detail of the confirmation of crystal direction is described in Appendix \ref{appendix_simu}.
The fitting function for relaxation curve to obtain $T^{-1}_1$ was single exponential formula,
%formula,
\begin{equation}
M (t) = A + M_0 (1 - \mathrm{e}^{-\frac{t}{T_1}}),
\label{single}
\end{equation}
This is because the crystal is high-quality.
Since the quadrupolar splitting of ${}^7$Li nuclei ($I = 3/2$) is small, the single exponential formula (Eq.(\ref{single}) is also valid for ${}^7$Li.
If the crystal quality is poor, the fitting function become emprical stretch exponential formula,
\begin{equation}
M (t) = A + M_0 (1 - \mathrm{e}^{-\left(\frac{t}{T_1}\right)^\beta}),
\end{equation}
but it was not the case for our measurement.
