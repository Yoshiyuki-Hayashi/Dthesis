\chapter*{Abstract}
\addcontentsline{toc}{chapter}{Abstract}
Hyperhoneycomb $\beta$-Li$_2$IrO$_3$ is a candidate material for Kitaev spin liquid.
High pressure suppresses the spiral magnetic oreder without metallization, so pressure-induced spin liquid is expected.
In fact, $\mu$SR reported the appearance of liquid-like spin component at high pressure.
Our main motivation is to test the liquid component with NMR relaxation measurement. 

The magnetic properties of $\beta$-Li$_2$IrO$_3$ were investigated
by the ${}^7$Li NMR and the magnetization measurements under high pressures up to 3.9 GPa.
At ambient pressure, the NMR measurement confirmed the reported spiral ordering of $J_{\mathrm{eff}} = 1/2$ moments below $T_{\mathrm{mag}}$ = 38 K,
which is suppressed to become a ferromagnetically polarized paramagnet by applying a magnetic field parallel to the b-axis $B_b$ $>$ 2.7 T.
With application of pressure, the magnetic phase is suddenly suppressed at $P_c \sim$1 GPa.
Above $P_c$, a nonmagnetic spin-singlet state emerges below a first order transition temperature $T_\mathrm{d}$,
evidenced by the almost zero Knight shift and the suppressed relaxation rate $T^{-1}_1$.
The spin singlet phase is accompanied with the formation of the strong dimerized Ir$_2$ pairs within Ir zigzag chains previously observed by x-ray and neutron diffraction measurements.
We argue that the observed phase competition under pressure reflects the subtle balance between the orbital disordered $J_{\mathrm{eff}} = 1/2$ state
and the orbital selective valence bond state.

Similar pressure-induced dimerization is oberved among some Kitaev candidates.
The competition between spin-orbit coupling and dimerization might be ubiquitous phenomenon.
Our study indicates the pressure is not the suitable variable to tune the Kitaev candidates to be close to Kitaev spin liquid.
