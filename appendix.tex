\appendix
\chapter{Simulation in paramagnetic phase \& angular rotation}
\label{appendix_simu}
In this chapter, based on the calculation of Eq.(\ref{Li1}) and Eq.(\ref{Li2}), we simulate the peak splitting in ${}^7$Li NMR of $\beta$-Li$_2$IrO$_3$ in paramagnetic phase.
Since $+ 0 .04$ kOe is one order smaller than the values in Eq.(\ref{Li1}) and Eq.(\ref{Li2}), it is ignored in the calculation below.
$\pm$ in Eq.(\ref{Li1}) and Eq.(\ref{Li2}) means that there are four Li sites as a hyperfine tensor, even though crystallographically there are two sites.
Since Knight shift is a strength of internal field parallel to applied external field, it can be calculated as,
\begin{align}
K_i = \frac{\vec{n}\cdot\overrightarrow{H}_{hf}(i)}{H_0}
\end{align}
In paramagnetic phase, using Eq.(\ref{Bhf_n}), it is rewritten as,
\begin{align}
K_i = \frac{p}{B_0}\vec{n}\cdot\tilde{A}(i)\cdot\vec{n} = \frac{p}{B_0}M(i, \vec{n})
\end{align}
where $B_0 = \mu_0H_0$, $M (i,\vec{n})$ is a angular dependent part of Knight shift defined by the above.
$M (i,\vec{n})$ is plotted in Fig.\ref{angle_dep} (b),
where the angles are defined as the deviation from magnetic easy b axis (Fig.\ref{angle_dep}(a)).
%Fig. (a) definition of angles & (b) angle dependence
\begin{figure}
  \centering
  \includegraphics[scale=0.7]{angle_dep.png}
  \caption{(a) Angles $\theta$, $\phi$ are defined as the deviation from the magnetic easy b axis.
  (b) Angle dependent part of Knight shift, $M (i, \theta, \phi)$.
  Li1(2)$\pm$ correspond in $\pm$ in Eq.(\ref{Li1}) and Eq.(\ref{Li2}).}
  \label{angle_dep}
\end{figure}
If we apply field in arbitrary direction, spectrum splits to four lines in accordance with four hyperfine Li sites.
However if the field is in bc-plane or ac-plane, spectrum consists of two lines corresponding to two crystallographic Li sites.

This observation was utilized when we confirmed that the b-axis is parallel to applied field in NMR measurement.
First, I determined crystal direction by single crystal XRD and set it carefully to orient b-axis to be parallel to applied field.
In NMR measurement, spectrum with two-line structure was observed as expected for b-axis.
I named this original direction as b0 direction and correspondingly named the orthogonal directions as c0, a0.
Then, I conducted angular rotation with two-axis rotator from b0 to c0 axis to find the direction which shows the maximum splitting.
It is because b axis is a magnetic easy axis and the largest magnetization, i.e., splitting is expected.
I named this newly found direction as b1 direction, and correspondingly named the orthogonal directions as c1, a1.
(Note that a1 is the same as a0.)
Finally, I rotated from b1 to a1 axis and found the angle with maximum splitting again.
Larger angle rotation in a1b1 plane gives the further splitting of two-line structure as expected for ab-plane rotation.
As a result, I could confirm that the b axis is oriented to applied magnetic field.

The example of such an angular rotation is presented in Fig, \ref{b0c0} for b0-c0 rotation and in Fig. \ref{b1a1} for b1-a1 rotation.
In this case, I found b1 direction ($\theta = 95^\circ$) which shows slightly larger splitting than the original b0 ($\theta = 90^\circ$) direction in b0-c0 rotation (Fig.\ref{b0c0}).
Then the splitting is insensitive to a1b1 rotation in smaller angle range ($-10^\circ \leq \phi^\prime \leq 0^\circ$).
This indicates that b1 direction is close enough to angle region with maximum splitting.
On the other hand, at larger angle ($\phi^\prime \leq -15$), the Li2 line splits to two lines.
This confirms that the smaller angle region is closer to true b axis.
Therefore, for the measurement, I chose b1 as the direction close to b axis.
The accuracy should be roughly $\pm$ $5^\circ$ because the angular dependence was measured in the increment of $5^\circ$ around b axis.

%Fig. spectrum & splitting
\begin{figure}
  \centering
  \includegraphics[scale=0.7]{ang_dep_b0c0_2.png}
  \caption{(a) Angle dependence of spectrum at 0 GPa, 1 T, 50 K for the rotation in b0-c0 plane.
  (b) Angle dependence of the split in spectrum. The splitting was read by the eye.
  Note that the definition of the angle $\theta$ is not based on Fig. \ref{angle_dep} (a).
  It is defined as the angle within b0-c0 plane in reference of the real two-axis rotator geometry (Appendix.\ref{two-axis}).}
  \label{b0c0}
\end{figure}

\begin{figure}
  \centering
  \includegraphics[scale=0.7]{ang_dep_b1a1_2.png}
  \caption{(a) Angle dependence of spectrum at 0 GPa, 1 T, 50 K for the rotation in b1-a1 plane.
  (b) Angle dependence of the split in spectrum.
  The splitting was read by the eye.
  The splitting was measured from the leftmost peak.
  Note that the definition of the angle $\phi^\prime$ is not based on Fig. \ref{angle_dep} (a).
  It is defined as the angle within b1-a1 plane in reference of the real two-axis rotator geometry (Appendix.\ref{two-axis}).}
  \label{b1a1}
\end{figure}

\chapter{Two-axis rotation}
\label{two-axis}
To conduct the two-axis rotation of high-pressure cell (or ambient-pressure setup) in NMR measurement probe, we need to understand the mathematical geometry.
Let's define ($\hat{x}, \hat{y}, \hat{z}$) coordinate system for the probe and ($\hat{x}^\prime, \hat{y}^\prime, \hat{z}^\prime$) coordinate system for the cell
(Fig.\ref{two_axis}(a)).
Field is applied along $\hat{z}$ axis.
($\hat{x}, \hat{y}, \hat{z}$) coordinate system is fixed to the laboratory and never move.
($\hat{x}^\prime, \hat{y}^\prime, \hat{z}^\prime$) coordinate system rotates against ($\hat{x}, \hat{y}, \hat{z}$) coordinate system.
The rotation can be done around two axes.
One is fixed $\hat{x}$ axis ($\theta$ rotation) and  the other is moving $\hat{z}^\prime$ axis ($\phi$ rotation) (Fig.\ref{two_axis}(b)).
%Figure of probe and cell, definition of theta and phi
\begin{figure}
  \centering
  \includegraphics[scale=0.7]{two_axis.png}
  \caption{(a) Initial position without rotation.
  (b) The schematic view of the cell rotation of $\theta$ and $\phi$.}
  \label{two_axis}
\end{figure}
%The crystal to measure is in the cell and its position is fixed against the cell.
Suppose, at the initial cell position, those two coordinate systems correspond.
The way of two-axis rotation of the cell gives the relation between the coordinate systems,
%matrix R(theta, phi)
\begin{align}
\label{coordinates}
(\hat{x}^\prime, \hat{y}^\prime, \hat{z}^\prime) &= (\hat{x}, \hat{y}, \hat{z}) R(\theta, \phi),\\
R(\theta, \phi) &=
\left(
\begin{array}{ccc}
\cos\phi & -\sin\phi & 0 \\
\cos\theta\sin\phi & \cos\theta\cos\phi & \sin\theta \\
-\sin\theta\sin\phi & -\sin\theta\cos\phi & \cos\theta
\end{array}
\right).
\end{align}
If a, b, c axes of the crystal perfectly correspond to $\hat{x}^\prime, \hat{y}^\prime, \hat{z}^\prime$ direction of the cell, $(\theta,\phi) = (\frac{\pi}{2},0)$,
(0,0), $(\frac{\pi}{2}, \frac{\pi}{2})$ gives the measurement of b, c, a axes, respectively.
The $\theta$ and $\phi$ rotaition gives the rotation within bc plane and ab plane, respectively.

Suppose that there is a mismatch between crystal axes and $\hat{x}^\prime, \hat{y}^\prime, \hat{z}^\prime$ directions.
Then, suppose that we find b, c axis in the direction of $(\theta,\phi) = (\frac{\pi}{2}+\theta_0,0), (\theta_0,0)$.
Direction of a-axis is still $(\theta,\phi) = (\frac{\pi}{2}, \frac{\pi}{2})$.
We can conduct the measurement within bc plane by $\theta$ rotation,
However, we need to manipulate both of $\theta$ and $\phi$ for the measurement within ab plane.
The problem is how to find the values of $\theta$ and $\phi$ for the ab-plane measurement.

In general, if we set the rotation anlges to certain values of $(\theta,\phi)$, the direction,
%formula
\begin{align}
\label{general}
(\hat{x}^\prime, \hat{y}^\prime, \hat{z}^\prime) R^T (\theta, \phi)
\left(
\begin{array}{c}
0 \\
0 \\
1
\end{array}
\right),
\end{align}
is oriented to be parallel to probe $\hat{z}$ direction.
(Note that, for $R$, the inverse matrix is its transposed matrix: $R^{-1} = R^T$.)
It is because, from Eq. (\ref{coordinates}), direction of Eq. (\ref{general}) is nothing but a probe $\hat{z}$ direction,
%formula.
\begin{align}
(\hat{x}^\prime, \hat{y}^\prime, \hat{z}^\prime) R^T (\theta, \phi)
\left(
\begin{array}{c}
0 \\
0 \\
1
\end{array}
\right)
= (\hat{x}, \hat{y}, \hat{z})
\left(
\begin{array}{c}
0 \\
0 \\
1
\end{array}
\right).
\end{align}

Let's name the directions we found, $(\theta,\phi) = (\frac{\pi}{2}+\theta_0,0), (\theta_0,0). (\frac{\pi}{2}, \frac{\pi}{2})$ as b1, c1, a1 direction, respectively.
b1 direction is represented in ($\hat{x}^\prime, \hat{y}^\prime, \hat{z}^\prime$) coordinate system as,
%\hat{b1} = formula
\begin{align}
\widehat{b1} =
(\hat{x}^\prime, \hat{y}^\prime, \hat{z}^\prime) R^T (\theta_0 + \frac{\pi}{2}, 0)
\left(
\begin{array}{c}
0 \\
0 \\
1
\end{array}
\right).
\end{align}
Correspondingly, a1 direction is represented in ($\hat{x}^\prime, \hat{y}^\prime, \hat{z}^\prime$) coordinate system as
%\hat{a1} = formula
\begin{align}
\widehat{a1} =
(\hat{x}^\prime, \hat{y}^\prime, \hat{z}^\prime) R^T (\frac{\pi}{2}, \frac{\pi}{2})
\left(
\begin{array}{c}
0 \\
0 \\
1
\end{array}
\right).
\end{align}
For the measurement within a1b1 plane, we have to orient the arbitrary direction $\hat{m}$ within a1b1 plane,
%\hat{m} = formula
\begin{align}
\label{m}
\hat{m} = \cos\phi^\prime\widehat{b1} + \sin\phi^\prime\widehat{a1},
\end{align}
to be parallel to probe $\hat{z}$ direction by manipulating both of $\theta$ and $\phi$.
Here $\phi^\prime$ is a rotation angle within a1b1 plane.

Here let's generalize the problem and consider the way how to orient the arbitrary direction $\hat{n}$ on ($\hat{x}^\prime, \hat{y}^\prime, \hat{z}^\prime$) coordinate system
to be parallel to probe $\hat{z}$ direction.
We introduce polar coordinate for ($\hat{x}^\prime, \hat{y}^\prime, \hat{z}^\prime$) coordinate system (Fig.\ref{polar}).
\begin{figure}
  \centering
  \includegraphics[scale=0.7]{polar.png}
  \caption{Polar coordinates for ($\hat{x}^\prime, \hat{y}^\prime, \hat{z}^\prime$).}
  \label{polar}
\end{figure}
The arbitrary direction $\hat{n}$ can be described by two angles, ($\theta_n$, $\phi_n$),
%formula
\begin{align}
\hat{n} =
(\hat{x}^\prime, \hat{y}^\prime, \hat{z}^\prime)
\left(
\begin{array}{c}
\sin\theta_n\cos\phi_n \\
\sin\theta_n\sin\phi_n \\
\cos\theta_n
\end{array}
\right).
\end{align}
The problem is now reduced to find the rotation angle, ($\theta$, $\phi$), which satisfies,
%formula
\begin{align}
\hat{n} =
(\hat{x}^\prime, \hat{y}^\prime, \hat{z}^\prime)
\left(
\begin{array}{c}
\sin\theta_n\cos\phi_n \\
\sin\theta_n\sin\phi_n \\
\cos\theta_n
\end{array}
\right)
&= (\hat{x}, \hat{y}, \hat{z}) R(\theta, \phi)
\left(
\begin{array}{c}
\sin\theta_n\cos\phi_n \\
\sin\theta_n\sin\phi_n \\
\cos\theta_n
\end{array}
\right) \notag\\
&= (\hat{x}, \hat{y}, \hat{z})
\left(
\begin{array}{c}
0 \\
0 \\
1
\end{array}
\right),
\end{align}
for arbitrary ($\theta_n$, $\phi_n$).
Here, the second equation comes from Eq. (\ref{coordinates}) and the last equation is what we have to solve.
The solution is,
%formula
\begin{align}
\label{solution}
(\theta, \phi) = \left(-\theta_n, -\phi_n + \frac{\pi}{2}\right)
\end{align}
If ($\theta$, $\phi$) satisfies Eq. (\ref{solution}), arbitrary direction $\hat{n}$ on ($\hat{x}^\prime, \hat{y}^\prime, \hat{z}^\prime$) coordinate system
is oriented to be parallel to probe $\hat{z}$ direction.
Therefore, coming back to the original problem, we can orient $\hat{m}$ in Eq. (\ref{m}) to be parallel to probe $\hat{z}$ direction
by rewriting $\hat{m}$ as ($\theta_m$, $\phi_m$) in polar coordinate, and setting ($\theta$, $\phi$) = $\left(-\theta_m, -\phi_m+\frac{\pi}{2}\right)$.

To determine ($\theta$, $\phi$) for a1b1 rotation in programming, the above results were considered.

Finally, $(\theta,\phi)$ is controlled by two values (A, B) in KAME interface via
%formula.
\begin{align}
\theta = \frac{A-B}{80} \\
\phi = \frac{A+B}{160}
\end{align}

\chapter{Further NMR data}
\section{Measurement along c-axis}
We also measured $T^{-1}_1$ at 4 T along c-axis and at 2.6 GPa (Fig.\ref{c-axis}).
The value in high-$T$ paramagnetic phase along c-axis is larger than that along b-axis.
There is no significant difference in dimerization transition point, $T_d = 230$ K.

\begin{figure}
  \centering
  \includegraphics[scale=0.6]{c-axis.png}
  \caption{Field-direction dependence of $T^{-1}_1$ at 2.6 GPa.
  There is no significant difference in dimerization transition point, $T_d = 230$ K.}
  \label{c-axis}
\end{figure}

\section{Measurement for hysteresis}
Fig.\ref{hysteresis} shows $T^{-1}_1$ at 4 T along b-axis and at 3,5 GPa.
It is measured in the condition of decreasing temperature and increasing temperature around $T_d = 292$ K.
We didn't observe hysteresis by the increment of 5 K.
This means that the hysteresis is smaller than 5 K.

\begin{figure}
  \centering
  \includegraphics[scale=0.6]{hysteresis.png}
  \caption{$T^{-1}_1$ at 4 T along b-axis and at 3.5 GPa around $T_d = 292$ K.
  The hysteresis is smaller than 5 K.}
  \label{hysteresis}
\end{figure}

\section{Condition for measurement}
Detail pulse condition for the measurement is summarized in Table.\ref{pulse}.
%Table
\begin{table}
\begin{center}
\caption{Pulse condition we used for the measurement.}
\begin{tabular}{ccc} \hline
            & 1st pulse (width, intensity)& 2nd pulse (width, intensity)\\ \hline
 ${}^7$Li   & 2 $\mu$s, -3 dB         & 2 $\mu$s, 0 dB\\ \hline
 ${}^{63}$Cu& 2 or 5 $\mu$s, -3 or -1 dB      & 2 or 5 $\mu$s, 0 dB\\ \hline
\end{tabular}
\label{pulse}
\end{center}
\end{table}
For the low-temperature measurement, lower value of master level (-15 dB) with -10 dB attenuator was enough, but as increasing temperature, we needed to increase the master level
up to -12 dB with the attenuator.
Width of comb pulse to measure $T^{-1}_1$ was 6 or 7 $\mu$s with 0 dB and the number of comb pulses was one.

We used $\tau$ listed in Table.\ref{tau}.
\begin{table}
\begin{center}
\caption{$\tau$ we used for the measurement. The first and second row are the field and pressure when the $\tau$ was used.}
\begin{tabular}{ccccc} \hline
 $B$ (T)& $P$ (GPa)& $\tau$ ($\mu$s) for ${}^7$Li& $\tau$ ($\mu$s) for ${}^{63}$Cu\\ \hline
 4& 1.3& 20& 80&\\ \hline
 4& 2.6& 20& 80&\\ \hline
 4& 3.5& 70& 160&\\ \hline
 4& 3.9& 70& 160&\\ \hline
 4& 0& 70& 180&low $T$\\ \hline
 4& 0& 60& 180&For Li1, high $T$\\ \hline
 4& 0& 40, 50& 180&For Li2, high $T$\\ \hline
 1& 0& 20& 90&$T > T_{\mathrm{mag}} = 38$ K\\ \hline
 1& 0& 14& -&$T < T_{\mathrm{mag}} = 38$ K\\ \hline
\end{tabular}
\label{tau}
\end{center}
\end{table}

For the data collection, parameters listed in Table. \ref{trig} were used.
\begin{table}
\begin{center}
\caption{Parameters for data collection.}
\begin{tabular}{cccc} \hline
                  & ${}^7$Li& ${}^{63}$Cu& ${}^7$L (1 T)\\ \hline
 From Trigger (ms)& -0.01& -0.07& -0.005\\ \hline
 Width (ms)& 0.045& 0.2& 0.045\\ \hline
\end{tabular}
\label{trig}
\end{center}
\end{table}

All NMR measurement was based on KAME interface programmed by K. Kitagawa.
