\chapter{Result}
\label{result}
\section{Ambient pressure}
At ambient pressure, the magnetic susceptibility $\chi_b(T)=M_b/B_b$ with applied field $B_b$ = 1 T parallel to the b-axis, shown in Fig. \ref{0GPa_Tdep_Kchi}(a),
agrees well with those reported previously \cite{takayama2015hyperhoneycomb}.
At high temperatures, a Curie-Weiss behavior with effective moment close to $p_\mathrm{eff}$ = 1.73 $\mu_B$, expected for pure $J_\mathrm{eff}$ = 1/2 moment,
and Weiss temperature $\theta_{CW} \sim +40$ K (ferromagnetic) is observed.
With further lowering temperature, $\chi_b(T)$ shows a rapid increase below $\sim$ 50 K,
followed by almost temperature independent behavior below the well-defined kink at $T_\mathrm{mag} = 38$ K.
The kink corresponds to a spiral ordering of $J_\mathrm{eff}$ = 1/2 moments.
At a high field $B_b$ = 4 T, the kink corresponding to the magnetic transition is gone, and system remains paramagnetic with field induced moment of 0.35 $\mu_B$
\cite{takayama2015hyperhoneycomb}.

In Fig. \ref{0GPa_spectrum}(a,b), we show ${}^7$Li NMR spectra at ambient pressure with applied magnetic field parallel to b-axis.
Two NMR peaks are observed, consistent with the presence of the two Li sites.
Those two peaks as a function temperature shifts to the opposite direction, meaning that the two sites have the opposite sign of hyper fine coupling.
The temperature dependence is well scaled by the temperature dependent magnetic susceptibility $\chi_b(T)$ as shown in Fig. \ref{0GPa_Tdep_Kchi}(a,b),
except for the high temperature region above $\sim 180$ K where the separation of the two peaks and hence the determination of the Knight shift for Li1 and Li2
is subject of substantial ambiguity (Fig.\ref{0GPa_spectrum}(d)).
The $K-\chi$ plot in Fig. \ref{0GPa_Tdep_Kchi}(d) yields a hyperfine field $A_{hf}$= -0.513 $\pm$ 0.007 kOe/$\mu_B$ and 1.67 $\pm$ 0.02 kOe/$\mu_B$, respectively.
If we only consider the classical dipolar field, the calculation gives $A_{hf}$ = -0.24 kOe/$\mu_B$ for Li1 and 0.19 kOe/$\mu_B$ for Li2.
The detail of calculation is as follows.

In order to estimate hyperfine coupling constant of $\beta$-Li$_2$IrO$_3$ in paramagnetic phase,
we simulated the internal hyperfine field on Li site $\bm{r}_i$, $\overrightarrow{H_{hf}}(i)$, generated by dipolar field of magnetic moments on Ir sites.
It is given by
\begin{align}
\label{Hhf_dip}
H^\alpha_{hf}(i) &= \sum^{N_s}_{j = 1}\sum_{\beta = x,y,z}A^{\alpha\beta}(i, j)\mu^\beta_e(j),\\
A^{\alpha\beta} (i, j) &= \frac{1}{4\pi}\sum^{N_s}_{i = 1}\left(3\frac{(r^\alpha_j-r^\alpha_i)(r^\beta_j-r^\beta_i)}{|\bm{r}_j - \bm{r}_i|^5}
  - \frac{\delta^{\alpha\beta}}{|\bm{r}_j - \bm{r}_i|^3}\right),
\end{align}
where $\vec{\mu}_e(j)$ is a magnetic moment of localized electronic spin at Ir site $\bm{r}_j$, $N_s$ is the number of the localized spins, and $A^{\alpha\beta}$ is hyperfine tensor.
In paramagnetic phase, magnetic moments are parallel to the applied field:
\begin{align}
\label{PM}
\vec{\mu_e} = p\mu_B \frac{\vec{H_0}}{H_0}= p\mu_B\vec{n},
\end{align}
where $p\mu_B$ is the size of the moment, $\vec{H_0}$ is the applied field, and $\vec{n}$ is the direction of the applied field.
Eq.(\ref{Hhf_dip}) and (\ref{PM}) gives,
\begin{align}
\label{Bhf_n}
B^\alpha_{hf} = \mu_0 H^\alpha_{hf}(i) = p\sum_{\beta = x,y,z}\tilde{A}^{\alpha\beta}(i)n^\beta,
\end{align}
where
\begin{align}
\label{A_tilde}
\tilde{A}^{\alpha\beta} (i) = \sum^{N_s}_{j = 1} A^{\alpha\beta}(i, j)\mu_0\mu_B.
\end{align}
The result of $\tilde{A}(i)$ (kOe) is,
\begin{align}
\label{Li1}
\tilde{A}(\mathrm{Li(1)site}) =
\begin{pmatrix}
0.99 & \pm0.53 & 0\\
\pm0.53 & -0.28 & 0\\
0 & 0 & -0.70
\end{pmatrix},
\end{align}

\begin{align}
\label{Li2}
\tilde{A}(\mathrm{Li(2)site}) =
\begin{pmatrix}
-0.56 & \pm1.2 & 0\\
\pm1.2 & 0.15 & 0\\
0 & 0 & 0.41
\end{pmatrix},
\end{align}
in the basis of $(\hat{a}, \hat{b}, \hat{c})$, where $\hat{a} = \vec{a}/a$ for unit cell vector along a-axis, $\vec{a}$, and lattice constant of a-axis, $a$, etc.
The summation in Eq. (\ref{A_tilde}) was conducted spherically, taking the site $i$ as a center position \cite{Kanamori}.
Demagnetic coefficients are assumed to be zero.
$+ 0 .04$ kOe should be added as a contribution from the surface of the sphere for each diagonal component.
Therefore, for b-axis measurement, calculated hyperfine coupling constant is $A_{hf} = -0.28 + 0.04 = -0.24$ kOe/$\mu_B$ for Li1 and $A_{hf} = 0.15 + 0.04 = 0.19$ kOe/$\mu_B$ for Li2.

Thus, we assign the peak with negative hyperfine field as Li1 and the other as Li2.
The deviation from the calculated value is due to Fermi contact term, which comes from the finite overlap of electronic wave function at ${}^7$Li nuclear site.
The effect of Fermi contact term is more significant for Li2 than Li1.
It is probably because Li2 site locates in the pseudo-honeycomb ring of Ir and has larger overlap of $J_\mathrm{eff}$ = 1/2 wave function than Li1.

Fig. \ref{0GPa_Tdep_Kchi}(c) shows the temperature dependence of $T^{-1}_1$ for Li2.
It gradually increases as lowering temperature in high-T paramagnetic phase.
The behavior corresponds to the enhancement of dynamic susceptibility.
The difference between 1 T and 4 T appears below $\sim$ 60 K.
Both of $K$ and $T^{-1}_1$ of 4 T are suppressed from the values of 1 T (Fig. \ref{0GPa_Tdep_Kchi}(b,c)).
This corresponds to the suppression of magnetic order at 4 T.
Below $T_{\mathrm{mag}}$ = 38 K, spectrum of 1 T is broadened and we cannot see the paramagnetic two-line structure any more (Fig.\ref{0GPa_spectrum}(a,c)).
This is consistent with the appearance of non-coplanar incommensurate AF (ICAF) order observed in neutron scattering
because the ICAF order should diffuse the distribution of hyperfine field.
On the other hand, the spectrum of 4 T maintains the paramagnetic Li1 and Li2 even below 38 K (Fig. \ref{0GPa_spectrum}(b)).
This means that the field of 4 T destroys ICAF order and makes the system ferromagnetically polarized paramagnet.
It is further confirmed by the saturated $K$ and almost zero $T^{-1}_1$ below 38 K (Fig. \ref{0GPa_Tdep_Kchi}(b,c))
because, in the polarized state, the magnetization is saturated and the fluctuation is suppressed.
We couldn't observe the signature for the mixing of zigzag component in the high-field state \cite{ruiz2017correlated}.

\begin{figure}
  \centering
  \includegraphics[scale=0.6]{0GPa_Tdep_Kchi.png}
  \caption{Ambient-pressure magnetic data of $\beta$-Li$_2$IrO$_3$.
  (a) magnetic susceptibility,$\chi = M/B$, at 1 T, (b) Knight shift $K$ of Li2, (c) spin-lattice relaxation rate $T^{-1}_1$ of Li2, (d) $K$-$\chi$ plot.
  Field is applied along b-axis for all measurement.
  The temperature dependence of the shifts is well scaled by $\chi$ below 160 K [(a,b,d)],
  where there is a uncertainity to determine high-$T$ Knight shift above 180 K because of the closeness of two lines [Fig.\ref{0GPa_spectrum}(d)].
  Thus, $K$-$\chi$ plot was done in the temperature range of 40 K $\leq$ $T$ $\leq$ 160 K, where shifts in 60 K $<$ $T$ $\leq$ 160 K are from 4 T measurement and
  shifts in 40 K $\leq$ $T$ $\leq$ 60 K are from 1 T measurement.
  Both of $K$ and $T^{-1}_1$ of 4 T are suppressed from the values of 1 T below 60 K [(f,g)].
  It corresponds the suppression of the ICAF order by high field of 4 T.
  Saturated $K$ and almost zero $T^{-1}_1$ at 4 T below 38 K indicates the ferromagnetic polarization of $J_{\mathrm{eff}} = 1/2$ moments [(f,g)].}
  \label{0GPa_Tdep_Kchi}
\end{figure}

\begin{figure}
  \centering
  \includegraphics[scale=0.4]{0GPa_spectrum2.png}
  \caption{${}^7$Li NMR spectra at ambient-pressure.
  (a) ${}^7$Li NMR spectrum at 1 T, (b) spectrum at 4 T, (c) spectrum at 1 T in wider $K$ region, (d) spectrum at 4 T around 180 K.
   Field is applied along b-axis.
   Dotted lines are gaussian fitting results.
  Two lines corresponds to two Li sites, Li1 and Li2 [(a, b)].
  Below $T_{\mathrm{mag}}$ = 38 K, spectrum of 1 T is broadened [(a,c)], while two-line structure is preserved in spectrum of 4 T.
  This corresponds the appearance of ICAF order and the suppression of the order by high field of 4 T.
  There is a uncertainity to determine high-$T$ Knight shift above 180 K because it is difficult to precisely separate the close two lines as we can see the failure of double
  gaussian fitting [(d)].}
  \label{0GPa_spectrum}
\end{figure}

\newpage
\section{High pressure}
High-pressure magnetic susceptibility at 1 T along b-axis, $\chi_b=M_b/B_b$ is shown in Fig. \ref{mag_suscep} with the ambient pressure data.
The magnetic transition point, $T_{\mathrm{mag}}$ = 38 K doesn't change up to 1.4 GPa (Fig. 2(a,b)).
This means that ICAF ordered moments robustly exist under the pressure up to 1.4 GPa at 1 T.
On the other hand, at the pressure of 1.1 and 1.4 GPa, we observed the anomalous drop of $\chi_b$ at $T_{\mathrm{d}}$ = 78 K and 154 K, respectively (Fig. \ref{mag_suscep}(b)).
This indicates the appearance of the new high-pressure phase at $T_{\mathrm{d}}$.
It has a smaller magnetization than paramagnetic phase.
At 1.1 GPa and 1.4 GPa below $T_{\mathrm{d}}$, the new high-pressure phase and paramagnetic phase (or ICAF order phase below $T_{\mathrm{mag}}$) coexists.
The pressure-dependence of $\chi_b$ at 6 K (Fig. \ref{mag_suscep} (a) inset) is consistent with reported XMCD \cite{takayama2015hyperhoneycomb}
and $\chi$ for polycrystalline sample \cite{Majumder2018},
including the initial enhancement up to 0.9 GPa.

\begin{figure}
  \centering
  \includegraphics[scale=0.7]{mag_suscep4.png}
  \caption{(a) Magentic susceptibility $\chi$ = M/B at 1 T along b axis under various pressure.
  (b) the enlarged view. The magnetic transition at $T_{\mathrm{mag}}$ = 38 K is unchanged and exists up to 1.4 GPa, implying the robust ICAF order phase under pressure.
  The anomalous decrease of $\chi$ at $T_{\mathrm{d}}$ means the emergence of new high-pressure phase.
  At 1.1 GPa and 1.4 GPa, the new phase and the paramagnetic phase (or ICAF order phase below $T_{\mathrm{mag}}$) coexists below $T_{\mathrm{d}}$.}
  \label{mag_suscep}
\end{figure}

To clarify the new high-pressure phase observed in $\chi_b$ measurement, we conducted high-pressure ${}^7$Li NMR measurement at 4 T along b-axis.
Such a high field destroys ICAF order, but it gives us a larger intensity in NMR.
The results at 3.5 GPa is shown in Fig. \ref{3.5GPa}.
Above 300 K, the pseudo-spins are in paramagnetic state, evidenced by the spectrum distinguishable as paramagnetic Li1 and Li2 lines (Fig. \ref{3.5GPa} (a))
and the value of $T^{-1}_1$ comparable to that of paramagnetic phase at ambient pressure (Fig.\ref{3.5GPa} (c)).
(In high-pressure spectra at high temperature, there is no failure of separating two peaks, which was observed at ambient pressure data.)
However, at $T_{\mathrm{d}}$ = 292 K, two lines merges to one line (Li$_\mathrm{d}$) with the almost zero value of $K$ (Fig. \ref{3.5GPa} (a), (b)).
This means the transition to the non-magnetic state with almost zero susceptibility.
Also, $T^{-1}_1$ suddenly decrease at $T_{\mathrm{d}}$ = 292 K (Fig. \ref{3.5GPa} (c)).
The behavior agrees with the non-magnetic transition.
In the temperature range of 200 K $<$ $T$ $<$ $T_{\mathrm{d}}$ = 292 K, $K$ and $T^{-1}_1$ still have finite values.
It is considered that some defects in the non-magnetic phase produce magnetic moments around them in that temperature region.
They can give finite $K$ and $T^{-1}_1$ to non-magnetic domain and make the transition step-wise.
Below 200 K, both of $K$ and $T^{-1}_1$ are almost zero.
This confirms the non-magnetic ground state at 3.5 GPa.

\begin{figure}[H]
  \centering
  \includegraphics[scale=0.7]{3p5GPa3.png}
  \caption{7Li NMR at 3.5 GPa. (a) spectrum, (b) Knight shift $K$, (c) spin-lattice relaxation rate $T^{-1}_1$.
  In (a), dotted lines are gaussian fits for each line (Li1, Li2, Li$_\mathrm{d}$).
  In (c), $T^{-1}_1$ at ambient pressure for Li2 is also plotted for comparison.
  $T^{-1}_1$ was measured for main Li$_\mathrm{d}$ line below $T_\mathrm{d}$ at 3.5 GPa.
  Above $T_\mathrm{d}$, it was measured for the average of Li1 and Li2 lines.
  At $T_{\mathrm{d}}$ = 292 K, paramagnetic Li1 and Li2 lines merge to one line (Li$_\mathrm{d}$)
  with almost zero $K$ and sudden drop of $T^{-1}_1$ occurs, indicating non-magnetic transition. }
  \label{3.5GPa}
\end{figure}

Fig. \ref{spectrum_highP} shows the spectrum at 1.3 GPa, 2.6 GPa, 3.5 GPa plotted in the same scale of $K$.
(a), (b), (c) are the spectra around transition point, $T_\mathrm{d}$, and (d), (e), (f) are the spectra in the whole temperature range.
The sharp Li$_\mathrm{d}$ line at 3.5 GPa resides at around $K$ = 0 (Fig.\ref{spectrum_highP}(f)).
This means that hyperfine field for Li$_\mathrm{d}$ line at 3.5 GPa is less distributed around the zero value.
It further confirms the non-magnetic ground state at 3.5 GPa.

For 2.6 GPa, the abrupt drop of $T^{-1}_1$ from the paramagnetic value occurs at $T_{\mathrm{d}}$ = 230 K (Fig. \ref{T1_highP}(a)).
Correspondingly, the main line below $T_{\mathrm{d}}$ = 230 K has almost zero $K$ down to the lowest temperature (Fig. \ref{spectrum_highP} (b,e)).
These suggest that the main line below $T_{\mathrm{d}}$ = 230 K represents the non-magnetic domain, same as one at 3.5 GPa.
Thus, we name it as Li$_\mathrm{d}$ line.
The Li$_\mathrm{d}$ line is broader than one at 3.5 GPa.
This is probably caused by defect-induced moments in non-magnetic domain.
They can diffuse the hyperfine field around zero value.
There are the additional lines with positive $K$.
They are considered as coexisting paramagnetic component (note that at 4 T, ICAF order is suppressed.) or defect-induced magnetic moments in non-magnetic domain.
For the latter case, the concentration of defects is probably larger than one for the Li$_\mathrm{d}$ line enough to give a finite $K$.

For 1.3 GPa, the third line different from paramagnetic Li1 and Li2 splits from Li1 line at $T_{\mathrm{d}}$ = 90 K (Fig. \ref{spectrum_highP}(a)).
$T^{-1}_1$ for the new main line start to decrease from the paramagnetic value of Li1 below $T_{\mathrm{d}}$ = 90 K (Fig. \ref{T1_highP}(a)).
The transition temperatures, $T_{\mathrm{d}}$, are monotonously dependent on pressure.
Also, it is common with 2.6, 3.5 GPa measurement that there is anomaly in spectrum at $T_{\mathrm{d}}$ and that $T^{-1}_1$ decrease below $T_{\mathrm{d}}$.
Thus, we can assign the new main line as Li$_\mathrm{d}$ line, which represents the non-magnetic domain.
However, the Li$_\mathrm{d}$ line is broad and has a finite $K$ (Fig. \ref{spectrum_highP}(d)).
It is considered that the concentration of defect-induced moments in the non-magnetic domain is large enough to give the broad spectrum and finite $K$.
Below $T_{\mathrm{d}}$ = 90 K, paramagnetic Li1 and Li2 coexists with Li$_\mathrm{d}$ line.
This is consistent with the coexistence of paramagnetic component (or ICAF order moments below $T_\mathrm{mag}$ = 38 K)
and the new high-pressure phase observed in $\chi_b$ at 1.1, 1.4 GPa below $T_{\mathrm{d}}$ = 78 K and 154 K, respectively (Fig. \ref{mag_suscep}(b)).
Thus, we identify the new high-pressure phase found in 1 T $\chi_b$ measurement as non-magnetic phase confirmed in 4 T NMR measurement.
Here, different from ICAF order, which is field-dependent, the non-magnetic phase should be field-independent.
It is because the transition temperature $T_{\mathrm{d}}$ reaching to room temperature is much larger than the energy scale of 4 T.
Finally, not only Li$_\mathrm{d}$ line but also Li1 and Li2 lines are broadened at low temperature.
It is considered that some defects also exist in paramagnetic domain, diffuse the hyperfine field and make the paramagnetic state close to the glassy state.
At 5 K, we couldn't observe Li1 line because of the broadening and the small $T_2$.

For 3.9 GPa, $T^{-1}_1$ (Fig. \ref{T1_highP}(a)), $K$ and spectrum (not shown) behaves similarly to those at 3.5 GPa.
The only difference is the higher transition point, $T_{\mathrm{d}}$ = 315 K.

For all pressure, even below $T_{\mathrm{d}}$, there are temperature regions with finite $T^{-1}_1$ (Fig. \ref{T1_highP}(a)).
For 1.3, 2.6 GPa, the regions extend to the lowest temperature.
In these temperature regions, it is considered that defect-induced magnetic moments exist in non-magnetic phase and give finite dynamic magnetic response.

Fig. \ref{T1_highP}(b) is a log-scale plot of $T^{-1}_1$.
As indicated by arrow, temperature dependence in the almost zero value region at 3.5 and 3.9 GPa in linear-scale plot is not monotonous.
There is a pressure-independent peak structure.
It is possibly caused by defect.

\begin{figure}[H]
  \centering
  \includegraphics[scale=0.7]{spectrum_highP2.png}
  \caption{Spectra at 4 T along b-axis and at high pressure around transition temperature, $T_\mathrm{d}$, (a) 1.3 GPa, (b) 2.6 GPa, (c) 3.5 GPa,
  and in the whole temperature range, (d) 1.3 GPa, (e) 2.6 GPa, (f) 3.5 GPa.
  Li1, Li2 denote two Li sites in paramagnetic state.
  Li$_\mathrm{d}$ denotes the line for non-magnetic state.
  Dotted lines are gaussian fits for each line (Li1, Li2, Li$_\mathrm{d}$).
  At 3.5 GPa, non-magnetic state with sharp line at $K$ = 0 \% was observed.
  As lowering pressure, at 1.3, 2.6 GPa, the non-magnetic state (Li$_\mathrm{d}$) coexists with other components.
  Also, at 1.3, 2.6 GPa, possibly, some defects produce the magnetic moments in the non-magnetic domain and make Li$_\mathrm{d}$ line broader.
  At 1.3 GPa, they give even finite $K$ to Li$_\mathrm{d}$ line.}
  \label{spectrum_highP}
\end{figure}

\begin{figure}[H]
  \centering
  \includegraphics[scale=1.0]{T1_highP2.png}
  \caption{Spin-lattice relaxation rate, $T^{-1}_1$, at 4 T along b-axis and at each pressure.
  Non-magnetic transition point, $T_\mathrm{d}$, is defined as a onset temperature of significant drop of $T^{-1}_1$.
  $T_\mathrm{d}$ monotonically increases as applying pressure.
  $T^{-1}_1$ was measured for main Li$_\mathrm{d}$ line below $T_\mathrm{d}$.
  Above $T_\mathrm{d}$, it was measured for Li1 line at 1.3 GPa and for the average of Li1 and Li2 lines at other pressure.
  (a) linear scale, (b) log scale.
  In (b), we can see the peak structure at 3.5 and 3.9 GPa as indicated by arrow, which is possibly caused by defect.}
  \label{T1_highP}
\end{figure}

\newpage
\section{Phase diagram}
The transition temperatures we obtained, $T_{\mathrm{mag}}$ and $T_\mathrm{d}$ for 1 T $\chi_b$ measurement, and $T_\mathrm{d}$ for 4 T NMR measurement,
are plotted in $P-T$ phase diagram ignoring the difference of the field (Fig. \ref{phase}).
Note that the high field above $\sim$ 2.7 T along b-axis switches the ICAF order phase to paramagnetic phase.
The magnetic transition points, $T_{\mathrm{mag}}$, in $\chi$ of polycrystalline sample \cite{Majumder2018}
and the structural transition point, $T_s$, in single crystal XRD \cite{veiga2017pressure} are also plotted.
Consistent with the reported ones, the magnetic transition points, $T_{\mathrm{mag}}$, are almost pressure-independent.
We didn't observe $T_{\mathrm{mag}}$ in high-pressure NMR measurement because it was done under 4 T.
\begin{comment}
However, the coexisting component at 1.3 GPa is paramagnetic one.
It implies that, at least at 1.3 GPa, $T_{\mathrm{mag}}$ should appear if the field is lowered below 2.7 T because the paramagnetic phase should become ICAF order moments below 2.7 T.
\end{comment}
However, for 2.6 GPa, there is a possibility that the coexisting component is remaining paramagnetic phase and that $T_{\mathrm{mag}}$ can appear if the field is lowered below 2.7 T.
For 3.5 GPa, there is no coexisting component and no possibility for the appearnce of ICAF order even if the field is lowered.
Thus, we drew the horizontal dotted line for $T_{\mathrm{mag}}$ up to just below 3.5 GPa.

We can smoothly connect $T_\mathrm{d}$ in $\chi_b$, $T_\mathrm{d}$ in NMR measurement and $T_s$ in single crystal XRD with one line.
This means that the paramagnetic-to-non-magnetic transition we observed coincides with $F_{ddd}$-to-$C/2c$ structural transition.
In the $C/2c$ phase, only one kind of Ir-Ir bond, emphasized in Fig. \ref{phase}, is 12 \% shorter than other bonds.
Therefore, we conclude that the non-magnetic state is a singlet dimer state.

In the phase diagram, we denoted the coexisting region with dashed lines.
It is composed of paramagnetic spins (or ICAF ordered momenets below $T_\mathrm{mag}$) and singlet dimers.
Our high-pressure $\chi$ data and NMR at 1.3 GPa indicates such a coexistence.
From NMR at 2.6 GPa, there are two possibilities for the coexisting component.
One is paramagnetic component and  the other is defect-induced moments in singlet dimer domain.
In the plot (Fig. \ref{phase}), we included the $P-T$ region at 2.6 GPa into dashed coexisting phase because of the ambiguity and in order to simplify the phase diagram.
The behavior of $T^{-1}_1$ below $T_\mathrm{d}$ is successive for 2.6 GPa, 3.5 GPa and 3.9 GPa, especially if we focus on the temperature region with the finite $T^{-1}_1$ value
(Fig. \ref{T1_highP}).
This indicates that those corresponding $P-T$ regions are the same phase.
Thus, those $P-T$ regions are included into dashed coexisting region.
As we saw in previous section, the effect of defects is prominent in the dashed region.

\begin{figure}[h]
  \centering
  \includegraphics[scale=1.0]{phase19.png}
  \caption{Pressure-temperature phase diagram for $\beta$-Li$_2$IrO$_3$.
  PM, ICAF means paramagnetic phase, incommensurate non-coplanar antiferromagnetic phase, respectively.
  Dashed region is coexisting phase.
  $\times$ is from chi at 1 T along b-axis and $\bullet$ is from 4 T NMR.
  $\circ$ is from chi at 1 T for polycrystalline sample \cite{Majumder2018} and $\blacktriangledown$ comes from XRD \cite{veiga2017pressure}.
  Note that ICAF order is destroyed and substituted to PM at 4 T. (Inset) crystal structures for each phase.
  In $F_{ddd}$ phase at ambient pressure, all bonds have almost equivalent lengths \cite{takayama2015hyperhoneycomb}.
  In $C2/c$ phase, the emphasized bonds are 12 \% shorter than other bonds \cite{veiga2017pressure}.}
  \label{phase}
\end{figure}
