
\documentclass[a4,10.5pt]{report}
\usepackage[backend=bibtex,style=phys,articletitle=false,biblabel=brackets,chaptertitle=false,pageranges=false]{biblatex}
\bibliography{docter_thesis_ref}
\usepackage{amsmath,amsthm,amssymb}
\usepackage{epic, eepic}
\usepackage[dvipdfmx]{hyperref,graphicx}
\usepackage{pxjahyper}
\usepackage{braket}
\usepackage{comment}
\usepackage{bm}
\usepackage{here}

\renewcommand{\bibname}{References}

\newcommand{\mycomment}[1]{%
}%


\begin{document}
\large
\tableofcontents

\chapter*{Abstract}
\addcontentsline{toc}{chapter}{Abstract}
Hyperhoneycomb $\beta$-Li$_2$IrO$_3$ is a candidate material for Kitaev spin liquid.
High pressure suppresses the spiral magnetic oreder without metallization, so pressure-induced spin liquid is expected.
In fact, $\mu$SR reported the appearance of liquid-like spin component at high pressure.
Our main motivation is to test the liquid component with NMR relaxation measurement. 

The magnetic properties of $\beta$-Li$_2$IrO$_3$ were investigated 
by the ${}^7$Li NMR and the magnetization measurements under high pressures up to 3.9 GPa.  
At ambient pressure, the NMR measurement confirmed the reported spiral ordering of $J_{\mathrm{eff}} = 1/2$ moments below $T_{\mathrm{mag}}$ = 38 K, 
which is suppressed to become a ferromagnetically polarized paramagnet by applying a magnetic field parallel to the b-axis $B_b$ $>$ 2.7 T. 
With application of pressure, the magnetic phase is suddenly suppressed at $P_c \sim$1 GPa. 
Above $P_c$, a nonmagnetic spin-singlet state emerges below a first order transition temperature $T_\mathrm{d}$, 
evidenced by the almost zero Knight shift and the suppressed relaxation rate $T^{-1}_1$. 
The spin singlet phase is accompanied with the formation of the strong dimerized Ir$_2$ pairs within Ir zigzag chains previously observed by x-ray and neutron diffraction measurements. 
We argue that the observed phase competition under pressure reflects the subtle balance between the orbital disordered $J_{\mathrm{eff}} = 1/2$ state 
and the orbital selective valence bond state.

Similar pressure-induced dimerization is oberved among some Kitaev candidates.
The competition between spin-orbit coupling and dimerization might be ubiquitous phenomenon.
Our study indicates the pressure is not the suitable variable to tune the Kitaev candidates to be close to Kitaev spin liquid.

\chapter{Introduction}
Historically, band theory succeeded to explain the physics of weakly correlated $s$ or $p$ electron system like alkhali metal or semiconducting behavior of Si or Ge.
The interest of physicists pursuing the exotic quantum phenomena, then, focused on surveying the effect of strong Coulomb correlation.
The target was $3d$ electron system, where strong correlation originated from well-localized $3d$ orbital was expected ($U \sim 5$eV for NiO \cite{bengone2000implementation}).
For example, Mott/charge-transfer insulator \cite{arima1993variation}, colossal magnetoresistance in manganites \cite{dagotto2005complexity} 
and high-$T_C$ superconducting cuprate \cite{dagotto2005complexity} are the topics in 3d electron system.
These phenomena are beyond the conventional band theory or mean-field theory.
They stimulated the theoretical development to correctly include strong Coulomb $U$ such as dynamical mean-field theory \cite{georges1996dynamical}.
In that sense, $4d, 5d$ electron systems had been forgot because the effect of correlation was considered to be weaker in the extended $4d, 5d$ orbitals compared to $3d$.


The situation is drastically changed from the discovery of layered perovskite Sr$_2$IrO$_4$ \cite{kim2008novel, kim2009phase}.
In Sr$_2$IrO$_4$, octahedral crystal field split $5d$ orbital into $t_{2g}$ and $e_g$ orbitals.
Metallic state was expected since five electrons fill the widely-spread $t_{2g}$ orbital.
However, it is actually a magnetic insulator.
The insulating state is induced by the interplay of strong spin-orbit coupling (SOC) ($\lambda_{SO} \sim 0.3$ eV \cite{kim2008novel}, 0.01 eV for 3d ion \cite{Kanamori})
and modest Coulomb $U$($U \sim 2$eV for Sr$_2$IrO$_4$ \cite{arita2012ab}).
Strong SOC mixes up $S = 1/2$ of the hole and the effective angular momentum of $t_{2g}$ orbital, $L_\mathrm{eff} = -1$ (Fig.\ref{five_t2g}).
It gives the splitting of $t_{2g}$ orbitals into completely filled $J_{\mathrm{eff}} = 3/2$ quartet and half-filled $J_{\mathrm{eff}} = 1/2$ doublet.
The band width of $t_{2g}$ orbitals is reduced to that of $J_{\mathrm{eff}} = 1/2$ state by SOC.
In the narrow $J_{\mathrm{eff}} = 1/2$ state, the modest value of Coulomb $U$ is enough to open the Mott gap.
Thus, by reducing the band width, strong SOC enhances the effect of Coulomb $U$, which makes $4d$, $5d$ systems nontrivial spin-orbit Mott insulators.

The existence of $J_{\mathrm{eff}} = 1/2$ state in Sr$_2$IrO$_4$ was established by resonant X-ray scattering (RXS) \cite{kim2009phase}.
Fig.\ref{Sr2IrO4} shows the RXS intensity for L$_2$ and L$_3$ edge of Ir.
In $S = 1/2$ model, both of L$_2$ and L$_3$ edge should be enhanced, while in $J_\mathrm{eff} = 1/2$ model, only L$_3$ edge is enhanced.
The enhancement in RXS intensity is only observed in L$_3$ edge, indicating the validity of $J_\mathrm{eff} = 1/2$ model in Sr$_2$IrO$_4$.

%Fig. Energy level splitting of Ir$^{4+}$ or Ru$^{3+} ion.
\begin{figure}
  \centering
  \includegraphics[scale=0.7]{five_t2g.png}
  \caption{Energy level splitting of Ir$^{4+}$ or Ru$^{3+}$ ion under octahedral crystal field.
  $\Delta_{\mathrm{cubic}}$ is the strength of crystal field splitting.
  Angular momentum of $d$ orbital ($L$ = 2) reduces to $L_{\mathrm{eff}}$ = -1 in $t_{2g}$ manifold.
  Strong SOC mixes $L_{\mathrm{eff}}$ = -1 and $S = \frac{1}{2}$ of one hole and gives $J_{\mathrm{eff}} = \frac{1}{2}$ state.
  $\lambda_{\mathrm{SO}}$ is the strength of SOC.
  $\Delta_{\mathrm{cubic}}$ is of the order of 3.3 eV for Sr$_2$IrO$_4$ \cite{ishii2011momentum}, 2.2 eV for RuCl$_3$ \cite{sandilands2016spin}
  and 3.5 eV for $\beta$-Li$_2$IrO$_3$ \cite{takayama2018pressure}.
  Splitting $\frac{3}{2}\lambda_{\mathrm{SOC}}$ is of the order of 0.4 eV for Sr$_2$IrO$_4$ \cite{kim2008novel} 0.14 eV for RuCl$_3$ \cite{sandilands2016spin} 
  and 0.7 eV for $\beta$-Li$_2$IrO$_3$ \cite{takayama2018pressure}.}
  \label{five_t2g}
\end{figure}

\begin{figure}
  \centering
  \includegraphics[scale=0.7]{Sr2IrO4.png}
  \caption{(A) Solid lines are X-ray absorption spectra. The dotted red lines is the intensity of magnetic (1 0 22) peak.
  (B) Equal resonant enhancement for L$_2$ and L$_3$ edge is expected in $S = 1/2$ model. In contrast, the enhancement occurs only in L$_3$ edge in $J_\mathrm{eff} = 1/2$ model
  \cite{kim2009phase}.}
  \label{Sr2IrO4}
\end{figure}

In conventional $S = 1/2$ Mott insulator, the interaction between spins is isotropic AF Heisenberg interaction.
However, in the spin-orbit $J_\mathrm{eff} = 1/2$ Mott insulator, the interaction between $J_\mathrm{eff} = 1/2$ pseudo-spins can be strongly anisotropic \cite{jackeli2009mott}
and it depends on how the octahedra are connected.
First, let's consider two $J_\mathrm{eff} = 1/2$ spins in corner-shared octahedra (Fig.\ref{180_90bond}(a)).
In this case, isotropic Heisenberg interaction remains.
Interation between $J_\mathrm{eff} = 1/2$ spins is 
\begin{equation}
\mathcal{H}_{ij} = J_1\bm{S}_i\cdot\bm{S}_j + J_2(\bm{S}_i\cdot\bm{r}_{ij})(\bm{r}_{ij}\cdot\bm{S}_j),
\end{equation}
where $\bm{S}_{i(j)}$ is  $J_{\mathrm{eff}} = 1/2$ spin at site $i(j)$, $\bm{r}_{ij}$ is a unit vector along $ij$ bond.
$J_1$, $J_2$ is a coupling constant and the isotropic Heisenberg interaction, $J_1$, is more dominant.
On the other hand, let's consider two $J_\mathrm{eff} = 1/2$ pseudo-spins in edge-shared anion octahedra (Fig.\ref{180_90bond}(b)).
In this case, as shown in Fig.\ref{180_90bond}, we can consider two indirect virtual hopping paths mediated by anion, 
$d_{yz}$-$p_z$ (upper anion)-$d_{xz}$ and $d_{xz}$-$p_z$ (lower anion)-$d_{yz}$.
It is because the $J_{\mathrm{eff}} = 1/2$ state is a spin-orbital-entangled wavefunction including $d_{xz}$ and $d_{yz}$ \cite{jackeli2009mott}:
\begin{align}
\ket{J^z_{eff} = \pm \frac{1}{2}} = \frac{1}{\sqrt{3}}(\ket{d_{xy}, \pm}\pm\ket{d_{yz}, \mp}+i\ket{d_{xz}, \pm}),
\label{Jeff} 
\end{align}
where signs in $\ket{}$ of the right-hand side denotes the spin state.
The two contribution interfere in destructive manner and the isotropic exchange interaction vanishes.
This is a striking difference from 180$^\circ$ bond situation.
Only the anisotropic FM Ising interaction called Kitaev interaction,
\begin{equation}
\mathcal{H}^{(\gamma)}_{ij} = -KS^\gamma_iS^\gamma_j,
\label{Kitaev}
\end{equation}
remains, where $\bm{S}_{i(j)}$ is  $J_{\mathrm{eff}} = 1/2$ spin at site $i(j)$, $\gamma$ is a direction perpendicular to $ij$ bond and sharing edge, and $K$ is positive.
The magnetic easy axis of this FM Ising interaction, $\gamma$ direction, is bond-dependent.
This feature is key ingredient to realize Kitaev model in $J_{\mathrm{eff}} = 1/2$ pseudo-spins.

\begin{figure}
  \centering
  \includegraphics[scale=0.7]{180_90bond.png}
  \caption{Virtual hopping path for two $J_{\mathrm{eff}} = \frac{1}{2}$ spins in (a) corner-shared anion octahedra and (b) edge-shared anion octahedra \cite{jackeli2009mott}.
  Site $i, j$ is the position of two $J_{\mathrm{eff}} = \frac{1}{2}$ spins and small circle is anion $X$.
  In (a), Heisenberg interaction reamains, but in (b), two virtual $dpd$ hoppings via upper or lower anion gives Kitaev interaction.}
  \label{180_90bond}
\end{figure}

Here, Kitaev model is the model where one-half spins are aligned in honeycomb lattice and the Kitaev interaction (Eq.\ref{Kitaev}) is given for each bond (Fig.\ref{Kitaev_model})
\cite{kitaev2006anyons}.
In Hamiltonian, it can be written as,
%formula.
\begin{equation}
\mathcal{H} = -K_x\sum_\mathrm{X-bond} S^x_iS^x_j - K_y\sum_\mathrm{Y-bond} S^y_iS^y_j - K_z\sum_\mathrm{Z-bond} S^z_iS^z_j,
\label{Kitaev_H}
\end{equation}
where $K_x$, $K_y$, $K_z$ is positive, $\bm{S}$ is one-half spin, and X, Y, Z bonds are denoted in Fig.\ref{Kitaev_model}.
Since the interactions on three bonds try to orient the spin to three different direction, it gives the frustration.
The frustration disturbs the magnetic order and gives spin liquid in the ground state.

We can exactly solve Kitaev model as follows.
First, let's rewrite one-half spin in Kitaev model as,
%formula
\begin{equation}
S^\gamma_j = \frac{i}{2}b^\gamma_jc_j
\end{equation}
Here, $\gamma = x,y,z$ and $b^\gamma_j$, $c_j$ is Majorana fermions.
$j$ represents the site index.
This means that $\bm{S}_j$ is represented by four Majorana fermions.
One is itinerant Majorana fermion, $c_j$, and the other are localized ones, $b^x_j, b^y_j, b^z_j$.
Generally, Majorana fermions $a_j$ is an operator which satisfies,
\begin{equation}
\{a_i, a_j\} = 2\delta_{ij}, \vspace{4mm} a^\dag_i = a_i.
\end{equation}
$S^\gamma_j$ actually behaves one-half spins if we assume one constrain for Majorana fermions,
\begin{equation}
D_j = b^x_jb^y_jb^z_jc_j = 1.
\end{equation}
Then, Kitaev model (Eq.\ref{Kitaev_H}) can be rewritten as,
\begin{equation}
\mathcal{H} = \frac{i}{4}\sum_{j,k}A_{jk}c_jc_k.
\label{hopping}
\end{equation}
Here, $A_{jk}$ is defined as,
\begin{equation}
A_{jk} = \begin{cases}
2K_{\alpha_{jk}}u_{jk} & \text{if j and k are connected}\\
0 & \text{otherwise}
\end{cases}
\end{equation}
$u_{jk}$ is defined as,
\begin{equation}
u_{jk} = ib^{\alpha_{jk}}_jb^{\alpha_{jk}}_k,
\end{equation}
where $\alpha_{jk} = x, y, z$ if $jk$ is X, Y, Z-bond. 
$u_{jk}$ is a constant ($\pm 1$) since it commutes with $\mathcal{H}$ and satisfies $u^2_{jk} = 1$.
Thus, Kitaev model (Eq.\ref{Kitaev_H}) is mapped to nearest neighbor hopping model of Majorana fermions coupled with $Z_2$ variables.
The product of $u_{jk}$ around hexagon, $W$, is also conserved quantity with the value of $\pm1$.
Numerical calculation or Lieb's theorem \cite{kitaev2006anyons} tell that the ground state is obtained if $W = 1$ for all hexagons.
This means that one of the ground state is achieved if all $u_{jk}$ are equal to one.
Thus, the model (Eq.\ref{hopping}) reduces to just a nearest neighbor hopping model on honeycomb lattice. 
It gives the Dirac cone dispersion of itinerant Majorana fermion $c_j$, similarly to graphene.

For the Kitaev spin liquid, variety of novel phenomena have been predicted: application to topologaical quantum computing \cite{kitaev2006anyons}, 
realization of exotic superconductor by hole-doping \cite{you2012doping}, and fractional excitation by Majorana fermions \cite{nasu2015thermal, yoshitake2017temperature}.

The essential ingredients for Kitaev model is three 120$^\circ$ bond structure.
This means that we can consider the extension of Kitaev model on the lattice different from honeycomb at least if the lattice has three 120$^\circ$ bond structure. 
Theorists invented a lot of such extension of Kitaev model \cite{o2016classification}.
One of them is 3D hyperhoneycomb lattice (Fig.\ref{hyperhoneycomb}).
The hyperhoneycomb structure can be constructed from the three-dimensional stack of two-dimensional honeycomb layers, 
by dividing the honeycomb lattice into zigzag chains and the bridging bonds, rotating the zigzag chains alternatively and reconnecting them with those in the upper and lower planes.
Kitaev model on hyperhoneycomb lattice also shows spin liquid ground state with nodal line of itinerant Majorana fermion \cite{mandal2009exactly}.
Because of the three dimensionality, it shows topological phase transition to spin liquid ground state \cite{nasu2014vaporization}.

\begin{figure}
  \centering
  \includegraphics[scale=0.7]{Kitaev_model.png}
  \caption{Kitaev model on honeyomcb lattice \cite{kitaev2006anyons, kitagawa2018spin}.
  For each X, Y, Z bond, Kitaev interaction $\mathcal{H}^{(x)}$,$\mathcal{H}^{(y)}$,$\mathcal{H}^{(z)}$ (e.q.(\ref{Kitaev})) is defined.
  Each bond has different magnetic easy axis and frustrates.}
  \label{Kitaev_model}
\end{figure}

\begin{figure}
  \centering
  \includegraphics[scale=0.7]{hyperhoneycomb.png}
  \caption{Hyperhoneycomb lattice \cite{nasu2014vaporization}. a, b, c represent primitive translation vectors.}
  \label{hyperhoneycomb}
\end{figure}

Kiteav model was originally toy model.
However, since it was suggested that $J_{\mathrm{eff}} = 1/2$ pseudo-spins in edge-shared anion octahedra can show the Kitaev interaction \cite{jackeli2009mott}, 
the materialzation of Kitaev model become realistic.
It triggered the intensive material reseach on honeycomb-based Ir$^{4+}$ or Ru$^{3+}$ compounds pursuing Kitaev spin liquid.

Na$_2$IrO$_3$ has 2D honeycomb-layered structure of Ir$^{4+}$.
It is an insulator with optical gap $\sim 340$ meV, which can be explained only by LDA + SO + $U$, not LDA + SO \cite{comin20122}.
This means that Na$_2$IrO$_3$ is spin-orbit Mott insulator.
The Ir$^{4+}$ ions are surrounded by edge-shared O$^{2-}$ octahedra, so the building blocks for Kiteav model are provided.
However, it has AF interaction with Curie-Weiss temperature $\theta_{\mathrm{CW}} = - 116$ K \cite{singh2010antiferromagnetic} 
and shows zigzag order of $J_{\mathrm{eff}} = 1/2$ pseudo-spins \cite{ye2012direct}.
It is considered that additional AF interaction exists and makes the iridate far from ideal FM Kitaev spin liquid.
On the other hand, the dominance of bond-directional interaction was revealed by diffuse magnetic X-ray scattering \cite{chun2015direct}.
For Na$_2$IrO$_3$, instead of $J_{\mathrm{eff}} = 1/2$ picture, quasi-molecular orbital (QMO) picture is proposed \cite{mazin20122}.
In QMO picture, electrons delocalize within Ir hexagon, which is in contrast to atomically localizing $J_{\mathrm{eff}} = 1/2$ picture.
Other calculation indicates that anisotropic interaction induced by trigonal distortion of O$^{2-}$ octahedra as well as SOC play a key role to stabilze zigzag order 
\cite{yamaji2014first}.
In that sense, uniaxial strain to reduce trigonal distortion might be helpful to reach quantum spin liquid in Na$_2$IrO$_3$.


$\alpha$-Li$_2$IrO$_3$ has the same structure as Na$_2$IrO$_3$.
Again, it has AF interaction ($\theta_{\mathrm{CW}} = - 33$ K \cite{singh2012relevance}),
and shows non-collinear order \cite{williams2016incommensurate}.
However, if we substitute Li between the honeycomb layers of $\alpha$-Li$_2$IrO$_3$ to H, 
the resulting compound, $($H$_{\frac{3}{4}}$Li$_{\frac{1}{4}}$)$_2$IrO$_3$ 
or namely H$_3$LiIr$_2$O$_6$, shows no magnetic order.
The spin liquid behavior was confirmed down to 50 mK by specific heat measurement. 
The interaction is still AF for H$_3$LiIr$_2$O$_6$ ($\theta_{\mathrm{CW}} = -105$ K) \cite{kitagawa2018spin}, suggesting the deviation from ideal FM Kitaev spin liquid.
In fact, the characteristic behavior for Kitaev spin liquid like fractional excitation have not been reported yet.
Instead, fermionic excitation based on impurities was captured by NMR relaxation measurement \cite{kitagawa2018spin}.
Theorists revealed that configuration of interlayer hydrogen modifies the interaction between pseudo-spins via H-O bonding \cite{li2018role}.
Zero-point fluctution of hydrogen is considered as a key to understand spin-liquid behavior in H$_3$LiIr$_2$O$_6$ \cite{li2018role}.

$\alpha$-RuCl$_3$ is one of the most intensively studied Kitaev candidates.
It has Ru$^{3+}$ honeycomb-layered structure surrounded by edge-shared Cl$^-$ octahedra.
Photoemission spectroscopy confirmed that it is Mott insulator with substaintial SOC \cite{koitzsch2016j}.
The sign of Curie-Weiss temperature is dependent on field direction: $\theta_{\mathrm{CW}} = 37$ K ($H \parallel$ ab-plane) and 
$\theta_{\mathrm{CW}} = -150$ K ($H \parallel$ c-axis) \cite{majumder2015anisotropic}.
In calculation, Kitaev term is FM and 3-5 times larger than AF Heisenberg \cite{yadav2016kitaev}.
However, surpringly, inelastic neutron scattering revealed the existence of AF Kitaev interaction, not FM Kitaev interaction \cite{banerjee2017neutron}.
It shows zigzag AF order at low temperature, but the order can be suppressed under the field within ab-plane above $\sim 7$ T \cite{kasahara2018majorana}. 
In the high-field regime, the spin liquid behavior is confirmed by NMR measurement \cite{baek2017evidence}.
Anomolous thermal quantum Hall effect, which can be regarded as emergence of Majorana fermions, is also reported in the high-field spin liquid state of $\alpha$-RuCl$_3$ 
\cite{kasahara2018majorana}.

$\gamma$-Li$_2$IrO$_3$ \cite{modic2014realization} has 3D stripy honeycomb structure of Ir$^{4+}$.
The FM Curie-Weiss behavior in high temperature region is reported.
It shows the complex spiral order \cite{biffin2014noncoplanar}.
Also for $\gamma$-Li$_2$IrO$_3$, field above $\sim 3$ T suppresses the magnetic order and liquid state is proposed \cite{modic2017robust}, 
while the microscopic confirmation like NMR has not been done yet.
On the other hand, high-pressure $\sim 1.4$ GPa suppress the order without structural phase transition \cite{breznay2017resonant}.
The nature of high-pressure phase still remains unclear.

%Fig. crystal structure (a) honeyocmb (b) stripy (c) beta-Li2IrO3
\begin{figure}
  \centering
  \includegraphics[scale=0.7]{lattices2.png}
  \caption{(a) Crystal structure of 2D honeycomb-layered H$_3$LiIr$_2$O$_6$ \cite{kitagawa2018spin}.
  (b) 3D stripy honeycomb structure of $\gamma$-Li$_2$IrO$_3$ \cite{modic2014realization}.
  (c) Crystal structure of 3D hyperhoneycomb $\beta$-Li$_2$IrO$_3$ \cite{takayama2015hyperhoneycomb}.}
  \label{lattices}
\end{figure}

Our material, $\beta$-Li$_2$IrO$_3$ \cite{takayama2015hyperhoneycomb}, is one of such Kitaev candidates.
$\beta$-Li$_2$IrO$_3$ crystallizes in orthorhombic structure with the space group $F_{ddd}$ at ambient pressure. 
Ir$^{4+}$ ions have a spin-orbital-entangled $J_\mathrm{eff}$ = 1/2 pseudo-spin due to the octahedral crystal field from surrounding O$^{2-}$ ions 
as well as the strong spin-orbit coupling. 
The Ir ions form hyperhoneycomb sub-lattice, which can be viewed as a three-dimensional analogue of two-dimensional honeycomb lattice (Fig. \ref{lattices}(c)).  
The hyperhoneycomb lattice is surrounded by edge-shared O$^{2-}$ octahedra.
It means the existence of Kitaev interaction between Ir$^{4+}$ ions.
Since the hyperhoneycomb lattice shares 120$^\circ$ bond structure with honeycomb lattice, the frustration of Kitaev interactions also arises.
It leads spin liquid state theoretically \cite{mandal2009exactly}.
There are two crystallographically inequivalent Li sites. 
Li2 is surrounded by pseudo-honeycomb five-membered ring of Ir, while Li1 is located outside the pseudo-honeycomb ring.

Fig.\ref{band_cal} shows band calculation for $\beta$-Li$_2$IrO$_3$ under ambient pressure \cite{veiga2017pressure}.
We can see the splitting of $J_\mathrm{eff} = 1/2$ band and $J_\mathrm{eff} = 3/2$ band.
$J_\mathrm{eff} = 1/2$ band lies at fermi level, which means the physics of $\beta$-Li$_2$IrO$_3$ is governed by $J_\mathrm{eff} = 1/2$ pseudo-spins.

\begin{figure}
  \centering
  \includegraphics[scale=0.7]{band_cal.png}
  \caption{Band calculation for $\beta$-Li$_2$IrO$_3$ under ambient pressure \cite{veiga2017pressure}. $U_\mathrm{eff} = U - J_H = 2.5$ eV, where $U$ is on-site Coulomb repulsion
  and $J_H$ is Hund coupling.}
  \label{band_cal}
\end{figure}

Fig. \ref{beta0GPa} (a) shows the magnetic susceptibility $\chi = M/B$ at 1 T for polycrystalline sample of $\beta$-Li$_2$IrO$_3$ \cite{takayama2015hyperhoneycomb}.
The high-$T$ Curie-Weiss fit yields the effective moment $p_{\mathrm{eff}} \sim$ 1.61 $\mu_B$, which is close to 1.73 $\mu_B$ of ideal $J_{\mathrm{eff}} = 1/2$ moment,  
and the ferromagnetic Curie-Weiss constant $\theta_{CW} \sim 40$ K.
The ferromagnetically interacting $J_{\mathrm{eff}} = 1/2$ spins indicate the proximity to ideal Kitaev model in $\beta$-Li$_2$IrO$_3$.
Also, Raman spectroscopy suggests the emergence of Majorana fermionic excitation originating from Kitaev model \cite{glamazda2016raman}.
However, $\chi$ shows steep increase below 50 K and have a kink at $T_\mathrm{mag} = 38$ K.
The kink corresponds to the appearance of incommensurate non-coplanar antiferromagnetic (ICAF) order, confirmed by neutron scattering \cite{Biffin2014}. 
Fig. \ref{beta0GPa}(b) shows the magnetic structure for ICAF order phase.
The theoretical investigations suggest that the existence of additional interaction different from nearest neighbor FM Kitaev interaction 
can explain the realization of the spiral order in $\beta$-Li$_2$IrO$_3$ \cite{Lee2015}.
For example, direct $dd$ hoppings between the Ir$^{4+}$ ions gives isotropic Heisenberg interaction,
%formula
\begin{equation}
\mathcal{H}_{ij} = J\bm{S}_i\cdot\bm{S}_j,
\end{equation}
where $\bm{S}_{i(j)}$ is  $J_{\mathrm{eff}} = 1/2$ spin at site $i(j)$.
The combination of $dd$ and $dpd$ hopping gives symmetric off-diagonal interaction \cite{Nasu2014},
%formula
\begin{equation}
\mathcal{H}_{ij} = \Gamma_{xy} (S^x_iS^y_j + S^y_iS^x_j).
\label{symoff}
\end{equation}
Also, the importance of next nearest neighbor AF Heisenberg interaction is pointed out \cite{Katukuri2016}.
These interactions disturb Kitaev spin liquid state and stabilize the ICAF order phase in $\beta$-Li$_2$IrO$_3$.

%Fig. magnetic susceptibilty
\begin{figure}
  \centering
  \includegraphics[scale=0.7]{beta0GPa.png}
  \caption{Physical property of $\beta$-Li$_2$IrO$_3$ under ambient pressure.
  (a) Temperature dependence of magnetic susceptibility $\chi$ and $\chi^{-1}$ (inset) at 1 T for polycrystalline sample\cite{takayama2015hyperhoneycomb}.
  It shows the magnetic order below $T_{\mathrm{AF}} = 38$ K.
  (b) Magnetic structure in the ordered phase revealed by neutron scattering \cite{Biffin2014}}
  \label{beta0GPa}
\end{figure}

However, there are two ways to suppress the ICAF order --- applying high field along b-axis above $\sim$ 2.7 T 
and applying high pressure above $\sim$ 2 GPa \cite{takayama2015hyperhoneycomb}.
In the case of high field, the kink corresponding to ICAF order vanishes (Fig. \ref{Bdep_mag} (inset)), 
and system remains paramagnetic with field induced moment of $\sim$ 0.35 $\mu_B$ (Fig. \ref{Bdep_mag}).
The size of magnetic moment in ICAF order phase is 0.47 $\mu_B$, which is close to the value of the field-induced moment \cite{Biffin2014}.
Thus, the high field destroys the ICAF order by ferromagnetically orienting $J_{\mathrm{eff}} = 1/2$ moment along the field direction.
On the other hand, magnetic x-ray scattering revealed that high-field paramagnetic state admixes zigzag component along a-axis \cite{ruiz2017correlated}. 
There is also a possibility that the high-field state can be a spin liquid state.
Thus, our first motivation is to explore the high-field state of $\beta$-Li$_2$IrO$_3$ by NMR measurement.

\begin{figure}
  \centering
  \includegraphics[scale=0.7]{magnetization.png}
  \caption{Field dependence of magnetization of $\beta$-Li$_2$IrO$_3$ under ambient pressure.
  (inset) Field depnedence of magnetic susceptibility $\chi$ \cite{takayama2015hyperhoneycomb}.
  The kink of ICAF order vanishes at high field.}
  \label{Bdep_mag}
\end{figure}

In the case of high pressure, the ICAF order is suppressed without metallization.
Fig. \ref{beta2GPa} (a) shows the pressure dependence of X-ray Magnetic Circular Dichroism (XMCD) at 5 K, 4 T.
XMCD corresponds to bulk magnetization.
The field-induced moment is suppressed around 2 GPa.
On the other hand, the resistivity shows the insulating state (Fig. \ref{beta2GPa} (a) (inset)).
The disappearance of ICAF order without metallization indicates the possible pressure-induced Kitaev spin liquid in $\beta$-Li$_2$IrO$_3$. 
Also, $\mu$SR found the coexistence of liquid-like dynamic spins and glass-like frozen spins at $\sim$ 2 GPa (Fig. \ref{beta2GPa} (b)) \cite{Majumder2018}.
First principle calculation suggested the enhancement of symmetric off-diagonal term (Eq.\ref{symoff}) as applying pressure in $\beta$-Li$_2$IrO$_3$ \cite{Yadav2018, Kim2016}.
In the model with the dominance of symmetric off-diagonal term, classical spin liquid instability is proposed \cite{Rousochatzakis2017}.
Therefore, our second motivation is to test the spin-liquid component with different time scale, NMR relaxation measurement. 

\begin{figure}
  \centering
  \includegraphics[scale=0.7]{beta2GPa.png}
  \caption{Evolution of field-induced moment or magnetic order in $\beta$-Li$_2$IrO$_3$ as pressurizing.
  (a) Pressure dependence of X-ray Magnetic Circular Dichroism (XMCD) at 5 K, 4 T and resistivity (inset) \cite{takayama2015hyperhoneycomb}.
  XMCD reflects bulk magnetization.
  Around 2 GPa, the ICAF order is surppressed, but the state is insulating, which excludes metallization.
  (b) Phase diagram based on $\mu$SR \cite{Majumder2018}.
  Coexistence of frozen glassy spins and dynamic spins is proposed in the order-surppressed pressure region around 2 GPa.}
  \label{beta2GPa}
\end{figure}

On the other hand, the larger pressure around 4 GPa induces a structural phase transition at room temperature for $\beta$-Li$_2$IrO$_3$.
Fig. \ref{beta4GPa} (a) is pressure dependence of lattice parameter $\beta$ obtained by powder and single-crystal XRD at room temperature \cite{veiga2017pressure}.
Abrupt decrease of the lattice parameter is observed around 4 GPa.
It corresponds to structural phase transition from $F_{ddd}$ struccture to $C2/c$ structure. 
The high-pressure $C2/c$ structure has dimerized Ir-Ir bonds, which is 12 \% shorter than other bonds (Fig. \ref{beta4GPa}(b)). 

There are three possibilities for the non-magnetic state of dimer in $J_{\mathrm{eff}} = 1/2$ system (Fig. \ref{three}). 
The first one is $J_{\mathrm{eff}} = 1/2$ spin singlet based on the spin-orbital-entangled wave function.
This is conventional spin singlet, 
\begin{equation}
\frac{1}{\sqrt{2}}(\ket{\uparrow\downarrow}-\ket{\downarrow\uparrow}),
\end{equation}
where $\uparrow, \downarrow$ means $J_{\mathrm{eff}} = 1/2$ up state and down state.
It is realized when direct $dd$ hopping, $t_{dd}$, is larger than indirect $dpd$ hopping, $t'_{dpd}$ 
as long as $J_{\mathrm{eff}} = 1/2$ picture is valid.
The singlet-triplet excitation gap $\sim t^2_{dd}/U$ is relatively small.
Here $U$ is on-site Coulomb repulsion.

The second one is quadrupolar state of $J_{\mathrm{eff}} = 1/2$ spins \cite{Nasu2014}. 
It can be written as,
%formula
\begin{equation}
\ket{\psi^-_Q} = \frac{1}{\sqrt{2}}(\ket{\uparrow\uparrow}-i\ket{\downarrow\downarrow}),
\label{quad_m}
\end{equation}
where $\uparrow, \downarrow$ means $J_{\mathrm{eff}} = 1/2$ up state and down state.
It is stabilized when the symmetric off-diagonal interaction (Eq.(\ref{symoff})) exists and the $t'_{dpd}$ is larger than $t_{dd}$.
The state has a finite van Vleck term.
If the symmetric off-diagonal interaction (Eq.(\ref{symoff})) doesn't exist, the state (Eq.\ref{quad_m}) is degenerate with another quadrupolar state,
\begin{equation}
\ket{\psi^+_Q} = \frac{1}{\sqrt{2}}(\ket{\uparrow\uparrow}+i\ket{\downarrow\downarrow}).
\label{quad_p}
\end{equation}

The last one is band-insulator-like non-magnetic dimer based on molecular orbital.
This is a singlet of $S = 1/2$.
It is realized when $t_{dd}$ is larger than $U$.
The $J_{\mathrm{eff}} = 1/2$ wavefunction collapses in this limit.
The bonding-antibonding excitation gap $\sim t$ is relatively large.
Thus, our third motivation is to judge which non-magnetic state is realzied in the high-pressure dimer phase via NMR measurement.

\begin{figure}
  \centering
  \includegraphics[scale=0.7]{beta4GPa2.png}
  \caption{Structural property of $\beta$-Li$_2$IrO$_3$ at high pressure \cite{veiga2017pressure}.
  (a) Pressure dependence of lattice parameter $\beta$.
  Structural phase transition occurs around 4 GPa at room temperature
  (b) Structure in the high pressure $C2/c$ phase.
  Emphasized Y bond become 12 \% shorter than X, Z bond (X : 3.0139(12) \AA, Y : 2.6620(13) \AA, Z : 3.0122(15) \AA). }
  \label{beta4GPa}
\end{figure}

\begin{figure}
  \centering
  \includegraphics[scale=0.7]{three_possibilities.eps}
  \caption{Three possiblities for the non-magnetic state of dimer in $J_{\mathrm{eff}} = 1/2$ system.}
  \label{three}
\end{figure}

Here, we completed the $P$-$T$ phase diagram of $\beta$-Li$_2$IrO$_3$ (Fig. \ref{phase_pre}). 
We revealed that the high-$P$ phase is non-magnetic singlet dimer. 
It is probably based on molecular orbital. 
The high-$T$ high-$P$ singlet dimer formation is connected to low-$T$ low-$P$ magnetic collapse. 
From the continuity of the phase, we concluded that spin-liquid component observed in $\mu$SR is dynamical defect-induced moments embedded in singlet dimer phase. 
Our results suggest that the pressure-induced competition between orbitally-disordered $J_{\mathrm{eff}} = 1/2$ magnetism and orbital-selective molecular dimer 
is a ubiquitous phenomenon among some Kitaev magnets.

\begin{figure}[h]
  \centering
  \includegraphics[scale=1.0]{phase19.png}
  \caption{Pressure-temperature phase diagram for $\beta$-Li2IrO3. 
  PM, ICAF means paramagnetic phase, incommensurate non-coplanar antiferromagnetic phase, respectively. 
  Dashed region is coexisting phase. 
  $\times$ is from chi at 1 T along b-axis and $\bullet$ is from 4 T NMR. 
  $\circ$ is from chi at 1 T for polycrystalline sample \cite{Majumder2018} and $\blacktriangledown$ comes from XRD \cite{veiga2017pressure}. 
  Note that ICAF order is destroyed and substituted to PM at 4 T. (Inset) crystal structures for each phase. 
  In $F_{ddd}$ phase at ambient pressure, all bonds have almost equivalent lengths \cite{takayama2015hyperhoneycomb}. 
  In $C2/c$ phase, the emphasized bonds are 12 \% shorter than other bonds \cite{veiga2017pressure}.}
  \label{phase_pre}
\end{figure}

\chapter{Method}
\label{method}
\section{Synthesis}
We prepared two samples with different quality.
The first one was revealed to be low-quality with substaintial defects after intensive measurements (Fig.\ref{lowQ}).
Thus, we started the synthesis by ourselves to clarify the physics of $\beta$-Li$_2$IrO$_3$ with high-quality crystals.

There is a suggestion that crystal growth of $\alpha, \beta$-Li$_2$IrO$_3$ is reaction between IrO$_3$ gas and LiOH gas \cite{Freund2016}.
The reaction is as follows.
First, the oxidization, 
\begin{align}
\mathrm{Ir} + \mathrm{O_2} &\rightarrow \mathrm{IrO_2},\\
\mathrm{2Li} + \mathrm{\frac{1}{2}O_2} &\rightarrow \mathrm{Li_2O},
\end{align}
is necessary.
Then, IrO$_3$ gas and LiOH gas are generated in the equivalnece of 
\begin{align}
\mathrm{IrO_2 (s)} + \mathrm{\frac{1}{2}O_2} &\rightleftharpoons \mathrm{IrO_3 (g)},\\
\mathrm{Li_2O (s)} + \mathrm{H_2O (g)} &\rightleftharpoons \mathrm{2LiOH (g)}.
\end{align}
Finally, the gases reaact,
\begin{align}
\mathrm{IrO_3 (g)} + \mathrm{2LiOH (g)} &\rightarrow \mathrm{Li_2IrO_3 (s)} + \mathrm{H_2O (g)} + \mathrm{\frac{1}{2}O_2}.
\end{align}

Considering it, we adopted a vapor-transport-like method using H$_2$O and O$_2$ as agent.
In the very early stage for synthesis, we used Li and Ir.
It was a simple synthesis where Li and Ir are put together without mixing and heated.
Since the initial low-quality crystal had hexagonal shape, we tried to grow the hexagonal crystal.
However, later, the hexagonal crystal was revealed to be $\alpha$-Li$_2$IrO$_3$.
(Initial hexagonal low-quality crystal was at least $\beta$-Li$_2$IrO$_3$.)
Then, we switched the Li supplyer from Li to LiOH$\cdot$H$_2$O.
We obtained rhombic crystals and they were $\beta$-Li$_2$IrO$_3$.
Since the difference between Li and LiOH$\cdot$H$_2$O is considered as the amount of water vapor, we induced the humid air flow into furnace.
The rhombic $\beta$-Li$_2$IrO$_3$ crystal become bigger, but the quality of crystal was still low.
It shows the magnetic transition point at $T_\mathrm{mag} = 38$ K, but the low-$T$ Curie tail coming from defects was still substaintially large.
Then, we optimized the time for inducing humid air flow.
With the optimization, low-$T$ Curie tail was minimized.
However, the size of crystal was still too small $\sim 0.2$ mm.
Then, we switched the Li supplyer from LiOH$\cdot$H$_2$O to Li$_2$CO$_3$.
Thus, finally, after 60 trials, we could obtain the large high-quality crystal of $\beta$-Li$_2$IrO$_3$.

The maximum dimension of $\beta$-Li$_2$IrO$_3$ single crystal was 0.69 $\times$ 0.45 $\times$ 0.56 mm (b $\times$ a$\times$ c-axes) (Fig. \ref{crystal}).

%Fig. photo of crystal
\begin{figure}
  \centering
  \includegraphics[scale=0.7]{crystal.png}
  \caption{$\beta$-Li$_2$IrO$_3$ single crystal.
  (a) Top view from c axis.
  (b) Side view.}
  \label{crystal}
\end{figure}

The detail of optimized synthesis condition is as follows.
Ir (99.9\%, 100 $\mu m$ mesh, rare metallic) of 0.5 g was placed in a alumina crucible (SSA-S) of $\phi$ 14 mm $\times$ 38 mm and 
Li$_2$CO$_3$ (99.999\%, rare metallic) with 5 \% shortage of Li from the stoichiometric ratio of Li/Ir 
was just put on the Ir without mixing by mortar (Fig. \ref{synthesis} (a)). 
The crucible with loose lid was further placed in the larger crucible with loose lid.
The setup was heated in humid air with the sequence of 
4 h up to 880$^\circ$C, 1 h up to 1000$^\circ$C, 24 h up to 1050$^\circ$C, 24 h down to 1000$^\circ$C (Fig. \ref{synthesis} (b)).
From 1000$^\circ$C, the crucible was in dry air with the sequence of 24 h down to 950$^\circ$C and 5 h down to room temperature.
Humid air was made by flowing the dry air through de-ionized water (Fig.\ref{flow}).
The air flowed from the bottom to the top in the order-made airtight vertical tube furnace.
%aim of the each sequence
The original aim of each sequence in Fig. \ref{synthesis} (b) is as follows;
The one-hour sequence from 880$^\circ$C to 1000$^\circ$C is done for the oxidization of Ir.
The 24-hour sequence from 1000$^\circ$C to 1050$^\circ$C is done for the creation of seed crystals of $\beta$-Li$_2$IrO$_3$.
The 48-hour sequence from 1050$^\circ$C to 950$^\circ$C is for the crystal growth.
However, especially, the meanings of the one-hour sequence is almost nothing but the tracing of the initial success.

%picture of setup and sequence
\begin{figure}
  \centering
  \includegraphics[scale=0.7]{synthesis.png}
  \caption{(a) Setup for the synthesis. 
  The reagents are placed in alumina crucible doubly.
  The dimension is in mm.
  Air flows from the bottom to the top.
  (b) Sequence of furnace for the synthesis.
  RT means room temperature.}
  \label{synthesis}
\end{figure}

\begin{figure}
  \centering
  \includegraphics[scale=0.7]{air_condition.png}
  \caption{Flow chart for (a) humid air condition
  and (b) dry air condition.}
  \label{flow}
\end{figure}

The large crystals are mainly obtained from the top surface of the resulting compound with surrounded by hard Li-Ir oxide,
as if they were buried fossils.

Although the shape of the crystal, magnetic anisotropy and magnetic transition point are similar to $\gamma$-Li$_2$IrO$_3$ \cite{modic2014realization}, 
there are differences in extinction rule for (hh0) Bragg peaks and in the absolute value of magnetic susceptibility.
We confirmed that our crystal is truly $\beta$-Li$_2$IrO$_3$ through single-crystal X-ray diffraction with Bruker D8 Discover (Bruker AXS) 
and magnetic susceptibility measurement with Magnetic Property Measurement System (MPMS) (Quantum Design).

The magnetic transition point of our crystal is 38 K, which is the same as reported \cite{ruiz2017correlated}.

\section{Measurement of magnetic susceptibility}
We measured magnetic susceptibility along b-axis of $\beta$-Li$_2$IrO$_3$ single crystals (0.587 mg as a whole)
by MPMS in opposed-anvil-type high-pressure cell (mCAC cell, Fig.\ref{tateiwa_cell} \cite{Tateiwa2011}) up to 1.4 GPa and down to 6 K at 1 T. 
Pressure medium was Daphne 7575 (succesor of Daphne7474 \cite{Murata2008}). 

The advantages of this method are mainly three points \cite{Tateiwa2011}.
(i) The cell has low background magnetization.
It enables us to measure not only high-pressure ferromagnetism but also antiferromagnetism under pressure.
(ii) The cost to manufacture the cell is lower than diamond anvil cell.
(iii) It is easy to use for the beginners of high-pressure technique.

The detail technique is as follows.
For the background measurement, first we attached the lower zilconia anvil and CuBe gasket 
with small amount of GE varnish in high-pressure cell for the later easier treatment of setup.
Then we put tiny piece of Pb as manometer and Daphne 7575 as pressure medium.
The setup was cloesd with upper zilconia anvil and the background was measured (Fig.\ref{chi_meas_setup} (a)).

After the background measurement, setup was decomposed.
The three crystals were attached to be oriented to b-axis with small amount of GE varnish and the crystals are fixed at the center of lower anvil with GE varnish 
(Fig. \ref{chi_setup}).
Then again we attached the lower anvil and gasket with GE varnish, put the same Pb piece and Daphne 7575 and measured the magnetization (Fig.\ref{chi_meas_setup} (b)).
The background was directly substracted in the scan data for each temperature.

With the above method, the ambiguity arises in the background substraction for the difference in small amount of GE varnish, but it can be neglected in our measurement.

The gasket was broken at 1.7 GPa. 
After the measurement, the top edge of b-axis was scratched, but the orientation was maintained.

We summarized the dimension of the setup in Fig.\ref{chi_setup} (a).

\begin{figure}
  \centering
  \includegraphics[scale=0.6]{tateiwa_cell.png}
  \caption{The over all view of the high-pressure cell \cite{Tateiwa2011}.}
  \label{tateiwa_cell}
\end{figure}

\begin{figure}
  \centering
  \includegraphics[scale=0.6]{mag_suscep_meas.png}
  \caption{(a) The setup for background measurement
  (b) The setup for the magnetic susceptibility measurement}
  \label{chi_meas_setup}
\end{figure}

%photo of setup
\begin{figure}
  \centering
  \includegraphics[scale=0.6]{chi_setup.png}
  \caption{(a) The dimension of the anvil and gasket. The unit is in mm.
  (b) Photos of the setup for the magnetic susceptibility measurement under high pressure (top view).
  Three crystals are attached with GE varnish and it is glued on lower zilconia anvil.
  Field $B$ is parallel to b axis.
  The dimension is in mm.
  (c) Side view of (b).}
  \label{chi_setup}
\end{figure}

\newpage
\section{Principles of nuclear magnetic resonance (NMR)}
%explanation of NMR

\begin{figure}[h]
  \centering
  \includegraphics[scale=0.7]{nmr.png}
  \caption{Schematic image for the situation of NMR measurement. 
  By measuring the Zeeman levels of nuclear spins, we can know the magnetism of electrons through hyperfine field.
  $H_0$ is applied static field for the Zeeman level formation.
  $H_1(t)$ is radio wave with the energy of $\sim\hbar\omega = \gamma\hbar H_0$ to manipulate the direction of nuclear moments.
  Here, $\gamma$ is gyromagnetic ratio of the nuclear spin.}
  \label{nmr}
\end{figure}

Nuclear spins in solid interact with electrons.
The interaction is called hyperfine interaction.
It can be described as the field applied at the nuclear site, which is called hyperfine field (Fig.\ref{nmr}).
It consists of dipolar field of electron spin, the field induced by orbital motion of electrons, and Fermi contact term.
Here, Fermi contact term comes from the finite overlap of electron wave function at the nuclear spin site.

By applying magnetic field, the degenerate levels of the nuclear spin splits to Zeeman levels.
The hyperfine field gives the deviation in Zeeman levels or induce the transition between Zeeman levels. 

If we apply the radio wave (pulse) whose frequency corresponds to the Zeeman level splitting, the absorption of radio wave occurs (Fig.\ref{nmr}).
In Nuclear Magnetic Resonance (NMR) measurement, we utilized this phenomenon and manipulate the coherent motion of the total nuclear magnetization.
With the manipulation, we can obtain the information about the Zeeman level.
Since it contains the effect of the hyperfine field, we can know the magnetism of electrons via NMR.

First, we will present the motion of free nuclear spin under magnetic field in order to explain the experimental manipulation of total nuclear magnetization.
Then, we will explain what we can measure in NMR --- NMR spectrum, spin-spin relaxation time $T_2$, Knight shift $K$, and spin-lattice relaxation rate $T^{-1}_1$.

The overall discussion in this section is based on Ref.\cite{Kitaoka, Asayama, Takigawa}.

\subsection{Free nuclear spin under field}
Suppose we have a free nucleus with spin $\hbar\vec{I}$ and gyromagnetic ratio $\gamma$.
The magnetic moment $\vec{\mu}$ of nucleus is written as,
\begin{equation}
\vec{\mu} = \gamma\hbar\vec{I}.
\end{equation}
If we apply the static field along $z$ direction, $\overrightarrow{H_0} = H_0\vec{e_z}$, the nuclear moment interacts with the field.
The Hamiltonian for the Zeeman interaction is,
\begin{equation}
\mathcal{H} = -\vec{\mu}\cdot\overrightarrow{H_0}.
\end{equation}
The motion of the nuclear moment is described by equation of motion in Heisenberg picture,
\begin{equation}
\frac{\mathrm{d}\vec{\mu}}{\mathrm{d}t} = \frac{i}{\hbar}[\mathcal{H},\vec{\mu}] = \gamma\vec{\mu}\times\overrightarrow{H_0}.
\label{eom_H0}
\end{equation}
Here, commutation relations for spin components, $[I^x, I^y] = iI^z$ etc. were used.
The solution of Eq.(\ref{eom_H0}) is 
\begin{align}
\mu^x (t) &= \mu^x(0)\cos\omega_0t + \mu^y(0)\sin\omega_0t,\\
\mu^y (t) &= \mu^y(0)\cos\omega_0t - \mu^x(0)\sin\omega_0t,\\
\mu^z (t) &= \mu^z(0),
\end{align}
where we set 
\begin{equation}
\omega_0 = \gamma H_0
\end{equation}
This is a precession motion (Fig.\ref{precession1}(a)). 
If we view it from $+z$ direction , it rotates clockwise. 

\begin{figure}
  \centering
  \includegraphics[scale=0.7]{precession1.png}
  \caption{(a) Precession motion of nuclear moment under static field $H_0$ in laboratory coordinate system.
  It is clockwise with the angular velocity $\gamma H_0$
  (b) Nuclear moments doesn't move in the clockwise rotating coordinate system with angular velocity $\gamma H_0$.}
  \label{precession1}
\end{figure}

Then, let's change the coordinate system to the rotating coordinate system.
The rotating coordinate system shares the origin and $z$ axis with the original laborarory coordinate system and it rotates 
around $z$ axis with angular velocity of $\vec{\omega}$.
We name the coordinate axes fixed to the rotating system as $X, Y, Z$ axes.
$Z$ axis is the same as $z$ axis.
In the rotating system, the equation of motion of $\vec{\mu}$ can be written as,
\begin{equation}
\frac{\delta\vec{\mu}}{\delta t} = \gamma\vec{\mu}\times\left(\overrightarrow{H_0} + \frac{\vec{\omega}}{\gamma}\right),
\end{equation}
where $\frac{\delta}{\delta t}$ is a derivative in rotating coordinate system.
If we choose 
\begin{equation}
\vec{\omega} = -\gamma\overrightarrow{H_0}
\end{equation}
the equation of motion becomes 
\begin{equation}
\frac{\delta\vec{\mu}}{\delta t} = 0
\end{equation}
It means that nuclear moment doesn't move in the rotating coordinate with $\vec{\omega} = -\gamma\overrightarrow{H_0}$ (Fig.\ref{precession1}(b)).
The rotation of cooridnate system is clockwise if we view it from $+z$ direction.

Suppose that the initial condition of spin component is 
\begin{align}
\mu^x(0) = 0\\
\mu^y(0) = 0\\
\mu^z(0) = m.
\end{align}
In this case, the nuclear moment is fixed along $z$ direction and doesn't move under $\overrightarrow{H_0}$ 
both in the laboratory coordinate system and in the rotating coordinate system (Fig.\ref{precession2}(a)).
In this situation, let's apply the clockwise rotating magnetic field with the angular frequency $\omega_0$ perpendicular to $\overrightarrow{H_0}$,
\begin{equation}
\overrightarrow{H_1}(t) = H_1(\cos\omega_0t\vec{e_x} - \sin\omega_0t\vec{e_y}).
\end{equation}
In rotating coordinate system with $\vec{\omega} = -\gamma\overrightarrow{H_0}$ (clockwise), $\overrightarrow{H_1}(t)$ doesn't move.
Thus, we can set the direction of $\overrightarrow{H_1}(t)$ as $X$ axis.
The equation of motion becomes
\begin{equation}
\frac{\delta\vec{\mu}}{\delta t} = \gamma\vec{\mu}\times\overrightarrow{H_1}.
\label{eom_H1}
\end{equation}
Here, we omit the argument $t$ for $\overrightarrow{H_1}$ because it is time independent in the rotating coordinate system.
Eq.(\ref{eom_H1}) is equivalent to eq.(\ref{eom_H0}).
Therefore, the nuclear moment starts precession motion around $X$ direction with the angular frequency of $\omega_1 = \gamma H_1$ (Fig.\ref{precession2}(b)).
It is rotation within ZY plane and clockwise if we view it from $+X$ direction.

\begin{figure}
  \centering
  \includegraphics[scale=0.7]{precession2.png}
  \caption{(a) Initial situation.
  The nuclear moment is fixed along Z (or z) direction and doesn't move both in laboratory system and in rotating system.
  (b) If the rotating field $H_1 (t)$ is applied, nuclear moment starts precession around X axis with angular velocity $\gamma H_1$.}
  \label{precession2}
\end{figure}

This corresponds to applying radio wave or oscillating field with the energy equivalent to Zeeman level splitting $\hbar\omega_0$.
Oscillating field is a linear combination of clockwise rortaing field and counterclockwise roatating field.
In clockwise rotating coordinate system, the counterclockwise rotating component has a frequency of $2\omega_0$, which is too large compared to Zeeman level splitting.
Thus, the counterclockwise component doesn't affect the motion of nuclear moment.
It means that we can regard oscillating field as clockwise rotating field.

Static field $\overrightarrow{H_0}$ is generated by superconducting magnet.
Oscillating field $\overrightarrow{H_1}(t)$ can be made by applying the high-frequncy current to coil.
Thus, if we insert sample into coil and set the ocillating field of coil perpendicular to static field of the magnet,
we can manipulate the direction of magnetization of selected nuclei in the sample.

\subsection{Free induction decay (FID) and NMR spectrum}
If we apply pulse of the oscillating filed with the energy corresponding to $\hbar\omega_0$ only within the time of $\pi/(2\omega_1)$ ($\pi/2$ pulse), 
the nuclear moments rotate clockwise by $\pi/2$ from $Z$ direction to $Y$ direction (Fig.\ref{FID}(a)).
In the rotating system, the moment doesn't move any more.
In the laboraory system, it rotates in xy-plane with angular frequency $\omega_0$.
The rotation of the nuclear moment generates induced electromotive force on the coil and it is detected as a voltage signal with high frequancy of $\omega_0$.
If the field is perfectly uniform and static, the oscillating signal doesn't decay and its Fourier transformed spectrum is delta function $\delta(\omega-\omega_0)$ (Fig.\ref{FID}(a)).
However, in reality, spatial distribution and dynamical fluctuation are induced by local field.
Here, local field is a general term which means the microscopic field at nuclear site, including hyperfine field by electrons or dipolar field by other nuclei.
With the local field, the resonance condition for each nuclear moment can be different.
It gives a width around $\omega = \omega_0$ in the Fourier transformed spectrum (Fig.\ref{FID}(b)).
This means that some nuclear spins with the resonance of $\omega_0 + \delta\omega$ cannot be static in rotating system with angular frequency of $\omega_0$.
As time passed, they rotates and phase coherence of nuclear spins disappears (Fig.\ref{FID}(b)).
In signal, many oscillating components different from $\omega_0$ comes in and the signal decay (Fig.\ref{FID}(b)).
We can also say that the decay or the Fourier transformed spectrum reflects the information of local field.
The decay is called Free Induction Decay (FID).
The time constant for the decay is called $T^*_2$ and the Fourier transformed spectrum is called NMR spectrum.

\begin{figure}
  \centering
  \includegraphics[scale=0.7]{FID.png}
  \caption{The effect of $\pi/2$ pulse in the case of (a) uniform and static local field, and (b) realistic local field.
  Fourier transformed spectrum and evolution of nuclear moments for each case are also shown.
  In case of (a), there is no decay in signal, but in (b), Free Induction Decay (FID) is induced.
  From NMR spectrum of FID signal, we can know the local field distribution.
  If the field is given as a constant value, NMR spectrum is obtained as a frequency-swept data.}
  \label{FID}
\end{figure}

The weak point of FID is that it is immediately successive to $\pi/2$ pulse.
In general, there is a insensitive time after pulse injection.
If the insensitive time is longer than $T^*_2$, we cannot observe FID signal.
There is a technique to overcome such a situation called spin echo method (section\ref{spin_echo}).


\subsubsection{NMR spectrum}
%explanation of NMR spectrum
Here, we explain NMR spectrum with the simple example.
Let's consider about commensurate collinear AF transition. 
Fig.\ref{NMRspectrum} shows the schematic picture for the example.
Here, we only consider the dipolar field as a hyperfine field.
Also, we suppose that we measure NMR for the nuclei which have the electronic spins and that the nuclear site is crystallographically one.
Above the transition point, $T_{AF}$, the system is in paramagntic state, so there is only one kind of hyperfine field (Fig.\ref{NMRspectrum}(a)).
It means one peak in NMR spectrum.
On the other hand, below $T_{AF}$, antiferromagnetic up-down spin array gives two kinds of hyperfine field (Fig.\ref{NMRspectrum}(b)). 
It gives two peaks in NMR spectrum.
In the case of incommensurate AF order, hte hyperfine field is distributed in a certain range (Fig.\ref{NMRspectrum}(c)).
In this sense, in spin glass state, it gives the complex many kinds of hyperfine field (Fig.\ref{NMRspectrum}(d)).
As a reult, NMR spectrum become broadened.

\begin{figure}
  \centering
  \includegraphics[scale=0.7]{NMRspectrum.png}
  \caption{Simplified example for the concept of NMR spectrum
  in (a) paramagnetic state, (b) commensurate collinear AF order,  (c) incommensurate AF order, (d) spin glass.
  The number of peaks and the width of NMR spectrum indicate the distribution of hyperfine field.
  $P(H)$ is NMR spectrum and $H_0$ is the field corresponding to the resonance condition $H_0 = \omega_0/\gamma$,
  where $\omega_0$ is applied frequency of radio wave and $\gamma$ is gyromagnetic ratio of the nuclear spin.
  $H^z_{hf}$ and $H'^z_{hf}$ is a strength of hyperfine field.
  If the frequency is given, NMR spectrum is obtained as a field-swept data.}
  \label{NMRspectrum}
\end{figure}


\subsection{Spin echo and spin-spin relaxation time $T_2$}
\label{spin_echo}
Fig.\ref{spin_echo} shows the pulse sequence for spin echo method.
Applying $\pi/2$ pulse gives FID.
Then, after time $\tau$ passed, if we apply $\pi$ pulse, the direction of nuclear moments is reversed.
It leads the re-convergence of spin direction after the same time $\tau$ passed.
The same situation just after the $\pi/2$ pulse injection is reproduced at $t = 2\tau$.
The signal is called spin echo.
Fourier transformation of spin echo signal also gives NMR spectrum.
Thus, we can measure the signal which is apart from the insensitive time of pulses.

If the local field is static, spin echo signal doesn't decay.
However, in realtiy, the local field is time-dependent.
The value of local field in $0 < t < \tau$ and in $\tau < t < 2\tau$ can be different.
It disturb the complete phase recovery at $t = 2\tau$. 
Also, interaction between nuclear spins, e.g. dipor interaction, distrubs the phase coherence of the total nuclear spin motion.
Thus, the spin echo signal decays as a function of $2\tau$: $\sim \exp(-2\tau/T_2)$.
The time constant $T_2$ is called spin-spin relaxation time.
If $T_2$ is small and the decay is rapid, we cannot measure the spin echo signal.

\begin{figure}
  \centering
  \includegraphics[scale=0.7]{spin_echo.png}
  \caption{Principle of spin echo method.
  The pulse sequence and the corresponding schematic evolution of nuclear moments are shown.
  (a) $t = 0$, before applying $\pi/2$ pulse, (b) $t = \pi/(2\omega_1)$, just after applying $\pi/2$ pulse.
  (c) $t = \pi/(2\omega_1) + \tau$, just before applying $\pi$ pulse.
  (d) $t = \pi/\omega_1 + \tau$, just after applying $\pi$ pulse.
  (e) $t = 2\tau$, spin echo signal.}
  \label{spin_echo}
\end{figure}

\subsection{Knight shift $K$ and spin-lattice relaxation rate $T^{-1}_1$}
In Hamiltonian, the situation for NMR measurement (Fig.\ref{nmr}) can be written as,
\begin{align}
\mathcal{H}_{tot} &=\mathcal{H}_{nuc} + \mathcal{H}_{hf}+\mathcal{H}_{env},
\end{align}
where $\mathcal{H}_{tot}$ is Hamiltonian for the total system, 
$\mathcal{H}_{nuc}$ is for nuclear spins of the same species to measure, $\mathcal{H}_{env}$ is for the environment, 
$\mathcal{H}_{hf}$ represents the interaction between the nuclear spins and the environment through hyperfine field.
Here, we call the source of hyperfine field as the environment, which is usually itinerant electrons or localized electronic spins. 
Nuclear spins are treated as a sum of free independent spins under applied static field $\overrightarrow{H_0} = H_0\overrightarrow{e_z}$ 
and radio wave (pulses) $\overrightarrow{H_1}(t)$,
\begin{align}
\mathcal{H}_{nuc} = -\sum^{N_{nuc}}_{i = 1}\vec{\mu}\cdot\left(\overrightarrow{H_0}+\overrightarrow{H_1}(t)\right),
\end{align}
where $N_{nuc}$ is the number of the nuclear spins and $\vec{\mu}$ represents the magnetic momenet of the measuring nuclear spin. 
The interaction is written as,
\begin{align}
\mathcal{H}_{hf} = -\sum^{N_{nuc}}_{i = 1}\vec{\mu}\cdot\overrightarrow{H_{hf}}(i),
\end{align}
where hyperfine field at a certain nuclear site $\overrightarrow{r_i}$ is represented as 
$\overrightarrow{H_{hf}}(i) \left(= \overrightarrow{H_{hf}}\left(\overrightarrow{r_i}, \overrightarrow{H_0}\right) \right)$

As a result, the total Hamitonian can be described as,
\begin{align}
\mathcal{H}_{tot} &= -\gamma\hbar\sum^{N_{nuc}}_{i = 1} \left(I^zH_0 + I^zH^z_{hf}(i) 
 + \frac{1}{2}(I^+H^-_{hf}(i)+I^-H^+_{hf}(i))+\vec{I}\cdot\overrightarrow{H_1}(t)\right) \label{Htot}\\
 & + \mathcal{H}_{env} \notag
\end{align}
where $\gamma$ is a gyromagnetic ratio of the nuclear spin, which connects magnetic moment $\vec{\mu}$ and nuclear spin $\hbar\vec{I}$: $\vec{\mu} = \gamma\hbar\vec{I}$.
The first and fifth term in Eq.(\ref{Htot}) are what the experimentalists manipulate for NMR measurement as we saw in preceding sections.
With the maniplulation, we can measure the effect of the second, third, and forth term. 

\subsubsection{Knight shift}
%explanation of Knight shift
By the first term in Eq.(\ref{Htot}), we can form Zeeman level (Fig.\ref{Knight_shift}).
The second term in Eq.(\ref{Htot}) gives the deviation in the Zeeman level.
The difference in the levels is described as,
\begin{align}
\gamma (H_0 + H^z_{hf}(i)) = \gamma (1 + K_i)H_0,
\end{align}
in the unit of frequency.
Here, the Knight shift, 
\begin{align}
\label{K}
K_i = H^z_{hf}(i)/H_0,
\end{align}
is defined as the strength of hyperfine field along the applied field at the nuclear site $i$.
The definition of $K_i$ (Eq.(\ref{K})) we adopted here is too detail and a bit unusual.
We usually define Knight shift $K$ in paramagnetic phase only for the dominant hyperfine field that gives a peak shift in NMR spectrum.
In that sense, usual definition is 
\begin{align}
\label{K_usual}
K = H^z_{hf, peak}/H_0,
\end{align}
where $H^z_{hf, peak}$ is the dominant hyperfine field that gives a peak shift in NMR spectrum.
The definition of $K_i$ (Eq.(\ref{K})) contains $K$ in Eq.(\ref{K_usual}).
The Knight shift is detected as a peak shift from the resonance of $\omega_0 = \gamma H_0$ in NMR spectrum (Fig.\ref{NMRspectrum}).

The Kinght shift represents the static magnetic response because it is propotional to magnetic susceptibility in high-temperature paramagntic phase as shown below.
Suppose that the hyperfine field at nuclear site $i$ is given by localized electronic spins:
\begin{align}
\label{Hhf}
H^\alpha_{hf}(i) = \sum^{N_s}_{j = 1}\sum_{\beta = x,y,z}A^{\alpha\beta}(i, j)\mu^\beta_e(j),
\end{align}
where $A^{\alpha\beta}$ is hyperfine tensor, $N_s$ is the number of the spins, and $\vec{\mu_e}(j)$ is the magnetic momnent of localized electronic spin at site $j$.
In paramagnetic phase, each magnetic moment is parallel to applied field: 
\begin{align}
\label{PM_mu}
\vec{\mu_e} = m\vec{e_z}
\end{align}
where $m$ is the size of the moment.
By substituting Eq. (\ref{PM_mu}) to Eq. (\ref{Hhf}), Knight shift $K_i$ in Eq. (\ref{K}) is rewritten as.
\begin{align}
K_i = \frac{A_{hf}(i)}{N_A\mu_B}\chi_{mol},
\label{Kchi}
\end{align}
where $N_A$ is Avogadro number, $\mu_B$ is Bohr magnetron, $\chi_{mol}$ is a molar magnetic susceptibility: $\chi_{mol} H_0 = M_{mol} = N_A m$,
and $A_{hf}(i)$ is hyperfine field at nuclear site $i$ generated by magnetic moments of 1 $\mu_B$: $A_{hf}(i) = \sum^{N_s}_{j = 1}A^{zz}(i, j)\mu_B$.
Eq.(\ref{Kchi}) tells that we can experimentally determine the strength of the hyperfine field, $A_{hf}(i)$.
Suppose that we have the Knight shift, $K$, and magnetic susceptibility, $\chi$, in paramagnetic state.
By plotting $K$ as a function of $\chi$ at the corresponding temperature, we can obtain the strength of the hyperfine field, $A_{hf}(i)$, as a slope of the linear plot.
The plot is called $K-\chi$ plot.

\begin{figure}
  \centering
  \includegraphics[scale=0.7]{Knight_shift.png}
  \caption{Formation of Zeeman levels of nuclear spins in the case of I = 1/2.
  The deviation in Zeeman level induced by hyperfine field gives Knight shift.}
  \label{Knight_shift}
\end{figure}



\subsubsection{Spin-lattice relaxation rate, $T^{-1}_1$}
%explanation of T^{-1}_1
In thermal equilibrium state, nuclear moments distribute in Zeeman levels with Boltzmann distribution.
It has a corresponding nuclear magnetization.
Here, if we apply the fifth term in Eq.(\ref{Htot}), the radio wave, we can temporarily make non-equilibirum state with the deviated nuclear magnetization.
As the time passed, the magnetization recovers to the value of thermal equilibrium state through hyperfine field of the third and forth term in Eq.(\ref{Htot}).
The time constant for the relaxation process is called spin-lattice relaxation time $T_1$.
The inverse $T^{-1}_1$ is called spin-lattice relaxation rate, which corresponds the transition rate between the levels.

$T^{-1}_1$ is measured as follows (Fig.\ref{T1measurement}). 
First, we apply a pulse to make all the nuclear moments be in $XY$ plane,
In general, the number of pulses can be more than two (Comb pulses).
Then, after the time $T^*_2$ passed, the phase coherence of nuclear moments disappears and we lose the signal.
However, after the further time $T$ passed, the nuclear magnetization starts recover to the thermal equilibirum state via hyperfine field.
We can measure the recovering magnetization $M(T)$ by spin echo method.
By changing time $T$ and measuring $M(T)$ for each time, we can obtain relaxation curve $M(t)$.
Finally, we can obtain $T^{-1}_1$ by fitting to the relaxation curve.
The fitting function for $I = 1/2$ nuclei is,
\begin{equation}
M (t) = A + M_0 (1 - \mathrm{e}^{-\frac{t}{T_1}}).
\end{equation}
The fitting function can be more complex if there are many relaxation processes.

\begin{figure}
  \centering
  \includegraphics[scale=0.7]{T1measurement.png}
  \caption{Pulse sequence for $T^{-1}_1$ measurement.
  The corresponding schematic evolution of nuclear moments are also shown.
  (a) $t < 0$, before applying comb pulses.
  (b) $t = 0$, just after applying comb pulses.
  (c) $t\sim T^*_2$, signal decays.
  (d) $t = T$, total nuclear magnetizaztion recovers via hyperfine field.}
  \label{T1measurement}
\end{figure}

$T^{-1}_1$ represents dynamic magnetic fluctuation at the measuring nuclear site because it is related to dynamical magnetic susceptibility as shown below.
Fermi golden rule gives the general expression of $T^{-1}_1$ by hyperfine field $H^{\pm}_{hf}$,
\begin{align}
\label{T1_Hhf}
T^{-1}_1 = \left(\frac{\gamma}{2}\right)^2\frac{1}{N_{nuc}}\sum^{N_{nuc}}_{i = 1}\int^{\infty}_{-\infty}\mathrm{d}t\mathrm{e}^{i\omega_0t}\braket{\{H^+_{hf}(i,t),H^-_{hf}(i,0)\}},
\end{align}
where $\hbar\omega_0$ is the energy difference of the two Zeeman levels, $\{A, B\} = AB + BA$,
$Q (t) = \mathrm{e}^{\frac{i}{\hbar}\mathcal{H}_{env}t}Q\mathrm{e}^{-\frac{i}{\hbar}\mathcal{H}_{env}t}$ for arbitariry operator $Q$,
and thermal average $\braket{}$ is taken by the environment, i.e.,
\begin{align}
\braket{Q} = \sum_{\nu}\frac{\mathrm{e}^{-\beta E_{\nu}}}{Z_{env}}\bra{\nu}Q\ket{\nu},
\end{align}
where $E_{\nu}$ is the energy of the state of the environment $\ket{\nu}$, and $Z_{env} = \sum_{\nu} \mathrm{e}^{-\beta E_{\nu}}$.
\begin{comment}
NMR relaxation rate $T^{-1}_1$ is propotional to thermal average of time-correlation function of spin components perpendicular 
to the applied external magnetic field.
It can be related to dynamical susceptibility through 
\end{comment}
Again suppose that we have a hyperfine field of Eq.(\ref{Hhf}).
If we assume that orthogonal matrix to diagonalize $A^{\alpha\beta}$ is independent from magnetic site index $j$ and
that, for the eigenvalues of $A^{\alpha\beta}$, $\lambda_x, \lambda_y, \lambda_z$, $\lambda_x = \lambda_y (= A^{\perp})$ holds,
we can derive,
\begin{align}
\label{Hhf_S}
H^\pm_{hf}(i) = -\gamma_e\hbar\sum^{N_s}_{j = 1}A^{\perp}(i, j)S^\pm(j),
\end{align}
where $\gamma_e$ is a gyromagnetic ratio of the electronic spin, which connects magnetic moment $\vec{\mu_e}$ and electronic spin $\hbar\vec{S}$: $\vec{\mu_e} = -\gamma_e\hbar\vec{S}$,
and we used the same notation for the vectors before and after the change of the basis for diagonalization.
Substituting Eq.(\ref{Hhf_S}) to Eq.(\ref{T1_Hhf}), we obtain
\begin{align}
\label{T1_Sq}
T^{-1}_1 = \left(\frac{\gamma}{2}\right)^2(\gamma_e\hbar)^2\sum_{\bm{q}}A^{\perp}_{\bm{q}}A^\perp_{-\bm{q}}\int^{\infty}_{-\infty}\mathrm{d}t\mathrm{e}^{i\omega_0t}
\braket{\{S^+_{\bm{q}}(t),S^-_{-\bm{q}}(0)\}},
\end{align}
where we assume that $A^\perp (i, j)$ is dependent only on relative position $\bm{r}_i - \bm{r}_j$, i.e., $A^\perp (i, j) = A^\perp(\bm{r}_i - \bm{r}_j)$,
and conducted Fourier transformation,
\begin{align}
A^\perp (i, j) &= A^\perp(\bm{r}_i - \bm{r}_j) = \frac{1}{\sqrt{N_s}}\sum_{\bm{q}}A^\perp_{\bm{q}}\mathrm{e}^{-\bm{q}\cdot(\bm{r}_i - \bm{r}_j)},\\
S^\pm (j) &= \frac{1}{\sqrt{N_s}}\sum_{\bm{q}} S^\pm_{\bm{q}}\mathrm{e}^{-\bm{q}\cdot\bm{r}_j}.
\end{align}
Applying fluctuation-dissipation theorem \cite{Kitaoka}, 
\begin{align}
\frac{1}{4}\int^{\infty}_{-\infty}\mathrm{d}t\mathrm{e}^{i\omega_0t} \braket{\{S^+_{\bm{q}}(t),S^-_{-\bm{q}}(0)\}} 
= \frac{2\chi^{\prime\prime}(\bm{q}, \omega_0)}{(\gamma_e\hbar)^2(1-\mathrm{e}^{-\hbar\omega_0/k_BT})},
\end{align}
Eq.(\ref{T1_Sq}) is further transformed to
\begin{align}
\label{T1_chi}
T^{-1}_1 = \frac{2\gamma^2k_BT}{\hbar}\sum_{\bm{q}}A^{\perp}_{\bm{q}}A^\perp_{-\bm{q}} \frac{\chi^{\prime\prime}(\bm{q}, \omega_0)}{\omega_0},
\end{align}
where we assume $\hbar\omega_0 \ll k_BT$. 
Here, $\chi^{\prime\prime}(\bm{q}, \omega)$ is imaginary part of dynamic magnetic susceptibility.


\section{Measurement of NMR}
We measured ${}^7$Li Nuclear Magnetic Resonance (NMR) of $\beta$-Li$_2$IrO$_3$ single crystal 
in opposed-anvil cell (Fig.\ref{kitagawa_cell} \cite{Kitagawa2010}) up to 3.9 GPa and down to 5 K  at 4 T. 
The anvils are made of WC.
The type of lower and upper anvils were IVc and 3, respectively.
The gasket was NiCrAl and VIIb type.
For the gasket, the age hardening was done in the sequence of 1 hour up to 760 ${}^\circ$C, 12 hours for maintaining 760 ${}^\circ$C and 1 hour down to room temperature.
The cell was No. 6. 
Pressure medium was Daphne 7575 (succcesor of Daphne 7474 \cite{Murata2008}). 
We used ruby-fluorescene method to determine the pressure.

The setup for high-pressure NMR is shown in Fig.\ref{NMR_highP_setup}.
Here, we describe how to make the setup.
First, we have to make a coil by hand.
It is prepared by winding Cu wire of $\phi$ 0.02 mm around stretched teflon sheet.
Then, two kinds of instant adhesives (Aron Alpha, Toagosei Co., Ltd. and Araldite, Huntsman Japan) is painted on the coil.
The adhesive doesn't react with teflon, so we can glue coil only.
The coil is soldered with two-fold braided Cu wire of $\phi$ 0.1 mm.
Then, the available frequency range of the coil should be checked.
The soldered points is covered by Araldite to avoid vacuum discharge.

\begin{figure}[H]
  \centering
  \includegraphics[scale=0.7]{kitagawa_cell.png}
  \caption{The over all view of high-pressure NMR cell \cite{Kitagawa2010}.}
  \label{kitagawa_cell}
\end{figure}

\begin{figure}
  \centering
  \includegraphics[scale=0.7]{NMR_highP_setup.png}
  \caption{Schematic view of setup for high-pressure NMR measurement.}
  \label{NMR_highP_setup}
\end{figure}

Next, we have to prepare the setup to measure ruby fluorescence.
We need to make a constriction to optical fiber by instataneously heating.
The constriction enhances the stability against pressure.
The constricted fiber is connected to moissanite with optically transpareant glue (phthalic glue, NIKKA SEIKO).
Moissanite is optically transparent, which is imitaion of diamond.

The coil and optical fiber with moissanite is wrapped by capton tube.
The setup is inserted to lower anvil.
Then, we need to fill the space with 'cement', which is made of diamond powder and stycast 1266 in the ratio of 3.75 : 1 (weight).
If we fill 70 \% of the space, we need to push down the moissanite.
After filling the rest, we have to make the surface of the cement smooth and flat.
Some pieces of ruby are fixed on moissanite with Araldite.
Also, we put Aron Alpha to glue the capton tube to the anvil and to glue the fiber and Cu wire to capton tube.
The whole setup is put on the heater at 40 ${}^\circ$C for 14 hours.

After the cement become dried and hard, we have to paint the surface of the cement with coating, which is made of Araldite and appropriate amount of diamond powder.
The coating is also painted on the upper anvil and inside of the gasket.
Thus, we can obtain the setup in Fig.\ref{NMR_highP_setup}.

At this stage, we need to insert the $\beta$-Li$_2$IrO$_3$ crystal into the coil.
However, since the coil is larger than the crystal, we cannot confirm the crystal direction after inserting.
To overcome this, $\beta$-Li$_2$IrO$_3$ crystal was attached to small piece of weighing paper with GE varnish (Fig.\ref{crystal_setup}).
Weighing paper was cut in rectangular shape.
The direction parallel to shorter edge of rectangle is b-axis and the direction perpendicular to paper is c-axis.
Since the paper exceed the legth of coil, we can check the crystal direction through the paper even after the inserting to coil.

\begin{figure}
  \centering
  \includegraphics[scale=0.7]{crystal_setup.png}
  \caption{Schematic view of setup for crystal to insert the coil.}
  \label{crystal_setup}
\end{figure}

The rest procedure is to assemble the anvils and gasket, put Daphne 7575 oil as pressure medium and pressurize.
The photo of the setup just before pressurizing is shown in Fig.\ref{NMR_highP_setup_photo}.
It is important to put the marker on the cell, which indicates the direction of oscillating field of coil.

\begin{figure}
  \centering
  \includegraphics[scale=0.7]{NMR_highP_setup_photo.png}
  \caption{Photo of setup for high-pressure NMR measurement.}
  \label{NMR_highP_setup_photo}
\end{figure}

Pressurizig was done at room temperature with monitoring ruby fluorescence.
Then, the whole setup was gradually cooled down to liquid nitrogen temperature.
We confirmed the stability of pressure at liquid nitrogen temperature.
After that, we also checked the pressure at room temperature.
Then, high-pressure cell is ready to set NMR probe.

In the NMR probe, we have to check the motion of two-axis rotator.
The high-pressure cell should be set properly in order to avoid the breaking of the Cu wire and optical fiber during rotation.
After manually orienting the b-axis to be parallel to field direction, now it is ready to start the measurement.

Ambient-pressure NMR at 4 T and at 1 T were also measured.

To adjust crystal direction, we measured the angular dependence in a probe by two-axis rotator \cite{Kitagawa2010}. 
The adjustment was done both for the measurement at ambient pressure and at high pressure.
The accuracy is roughly $\pm$ 5$^\circ$ (Appendix \ref{appendix_simu}).
The detail of the confirmation of crystal direction is described in Appendix \ref{appendix_simu}.
The fitting function for relaxation curve to obtain $T^{-1}_1$ was single exponential formula,
%formula,
\begin{equation}
M (t) = A + M_0 (1 - \mathrm{e}^{-\frac{t}{T_1}}),
\label{single}
\end{equation}
This is because the crystal is high-quality.
Since the quadrupolar splitting of ${}^7$Li nuclei ($I = 3/2$) is small, the single exponential formula (Eq.(\ref{single}) is also valid for ${}^7$Li.
If the crystal quality is poor, the fitting function become emprical stretch exponential formula,
\begin{equation}
M (t) = A + M_0 (1 - \mathrm{e}^{-\left(\frac{t}{T_1}\right)^\beta}),
\end{equation}
but it was not the case for our measurement.

\chapter{Result}
\label{result}
\section{Ambient pressure}
At ambient pressure, the magnetic susceptibility $\chi_b(T)=M_b/B_b$ with applied field $B_b$ = 1 T parallel to the b-axis, shown in Fig. \ref{0GPa_Tdep_Kchi}(a), 
agrees well with those reported previously \cite{takayama2015hyperhoneycomb}. 
At high temperatures, a Curie-Weiss behavior with effective moment close to $p_\mathrm{eff}$ = 1.73 $\mu_B$, expected for pure $J_\mathrm{eff}$ = 1/2 moment, 
and Weiss temperature $\theta_{CW} \sim +40$ K (ferromagnetic) is observed. 
With further lowering temperature, $\chi_b(T)$ shows a rapid increase below $\sim$ 50 K, 
followed by almost temperature independent behavior below the well-defined kink at $T_\mathrm{mag} = 38$ K. 
The kink corresponds to a spiral ordering of $J_\mathrm{eff}$ = 1/2 moments. 
At a high field $B_b$ = 4 T, the kink corresponding to the magnetic transition is gone, and system remains paramagnetic with field induced moment of 0.35 $\mu_B$ 
\cite{takayama2015hyperhoneycomb}.

In Fig. \ref{0GPa_spectrum}(a,b), we show ${}^7$Li NMR spectra at ambient pressure with applied magnetic field parallel to b-axis. 
Two NMR peaks are observed, consistent with the presence of the two Li sites. 
Those two peaks as a function temperature shifts to the opposite direction, meaning that the two sites have the opposite sign of hyper fine coupling. 
The temperature dependence is well scaled by the temperature dependent magnetic susceptibility $\chi_b(T)$ as shown in Fig. \ref{0GPa_Tdep_Kchi}(a,b), 
except for the high temperature region above $\sim 180$ K where the separation of the two peaks and hence the determination of the Knight shift for Li1 and Li2 
is subject of substantial ambiguity (Fig.\ref{0GPa_spectrum}(d)).
The $K-\chi$ plot in Fig. \ref{0GPa_Tdep_Kchi}(d) yields a hyperfine field $A_{hf}$= -0.513 $\pm$ 0.007 kOe/$\mu_B$ and 1.67 $\pm$ 0.02 kOe/$\mu_B$, respectively. 
If we only consider the classical dipolar field, the calculation gives $A_{hf}$ = -0.24 kOe/$\mu_B$ for Li1 and 0.19 kOe/$\mu_B$ for Li2.
The detail of calculation is as follows.

In order to estimate hyperfine coupling constant of $\beta$-Li$_2$IrO$_3$ in paramagnetic phase,
we simulated the internal hyperfine field on Li site $\bm{r}_i$, $\overrightarrow{H_{hf}}(i)$, generated by dipolar field of magnetic moments on Ir sites.
It is given by
\begin{align}
\label{Hhf_dip}
H^\alpha_{hf}(i) &= \sum^{N_s}_{j = 1}\sum_{\beta = x,y,z}A^{\alpha\beta}(i, j)\mu^\beta_e(j),\\
A^{\alpha\beta} (i, j) &= \frac{1}{4\pi}\sum^{N_s}_{i = 1}\left(3\frac{(r^\alpha_j-r^\alpha_i)(r^\beta_j-r^\beta_i)}{|\bm{r}_j - \bm{r}_i|^5}
  - \frac{\delta^{\alpha\beta}}{|\bm{r}_j - \bm{r}_i|^3}\right),
\end{align}
where $\vec{\mu}_e(j)$ is a magnetic moment of localized electronic spin at Ir site $\bm{r}_j$, $N_s$ is the number of the localized spins, and $A^{\alpha\beta}$ is hyperfine tensor.
In paramagnetic phase, magnetic moments are parallel to the applied field:
\begin{align}
\label{PM}
\vec{\mu_e} = p\mu_B \frac{\vec{H_0}}{H_0}= p\mu_B\vec{n},
\end{align}
where $p\mu_B$ is the size of the moment, $\vec{H_0}$ is the applied field, and $\vec{n}$ is the direction of the applied field.
Eq.(\ref{Hhf_dip}) and (\ref{PM}) gives,
\begin{align}
\label{Bhf_n}
B^\alpha_{hf} = \mu_0 H^\alpha_{hf}(i) = p\sum_{\beta = x,y,z}\tilde{A}^{\alpha\beta}(i)n^\beta,
\end{align}
where
\begin{align}
\label{A_tilde}
\tilde{A}^{\alpha\beta} (i) = \sum^{N_s}_{j = 1} A^{\alpha\beta}(i, j)\mu_0\mu_B.
\end{align}
The result of $\tilde{A}(i)$ (kOe) is, 
\begin{align}
\label{Li1}
\tilde{A}(\mathrm{Li(1)site}) =
\begin{pmatrix}
0.99 & \pm0.53 & 0\\
\pm0.53 & -0.28 & 0\\
0 & 0 & -0.70
\end{pmatrix},
\end{align}

\begin{align}
\label{Li2}
\tilde{A}(\mathrm{Li(2)site}) =
\begin{pmatrix}
-0.56 & \pm1.2 & 0\\
\pm1.2 & 0.15 & 0\\
0 & 0 & 0.41
\end{pmatrix},
\end{align}
in the basis of $(\hat{a}, \hat{b}, \hat{c})$, where $\hat{a} = \vec{a}/a$ for unit cell vector along a-axis, $\vec{a}$, and lattice constant of a-axis, $a$, etc.
The summation in Eq. (\ref{A_tilde}) was conducted spherically, taking the site $i$ as a center position \cite{Kanamori}.
Demagnetic coefficients are assumed to be zero. 
$+ 0 .04$ kOe should be added as a contribution from the surface of the sphere for each diagonal component.
Therefore, for b-axis measurement, calculated hyperfine coupling constant is $A_{hf} = -0.28 + 0.04 = -0.24$ kOe/$\mu_B$ for Li1 and $A_{hf} = 0.15 + 0.04 = 0.19$ kOe/$\mu_B$ for Li2.
 
Thus, we assign the peak with negative hyperfine field as Li1 and the other as Li2. 
The deviation from the calculated value is due to Fermi contact term, which comes from the finite overlap of electronic wave function at ${}^7$Li nuclear site. 
The effect of Fermi contact term is more significant for Li2 than Li1. 
It is probably because Li2 site locates in the pseudo-honeycomb ring of Ir and has larger overlap of $J_\mathrm{eff}$ = 1/2 wave function than Li1.

Fig. \ref{0GPa_Tdep_Kchi}(c) shows the temperature dependence of $T^{-1}_1$ for Li2. 
It gradually increases as lowering temperature in high-T paramagnetic phase. 
The behavior corresponds to the enhancement of dynamic susceptibility. 
The difference between 1 T and 4 T appears below $\sim$ 60 K. 
Both of $K$ and $T^{-1}_1$ of 4 T are suppressed from the values of 1 T (Fig. \ref{0GPa_Tdep_Kchi}(b,c)). 
This corresponds to the suppression of magnetic order at 4 T. 
Below $T_{\mathrm{mag}}$ = 38 K, spectrum of 1 T is broadened and we cannot see the paramagnetic two-line structure any more (Fig.\ref{0GPa_spectrum}(a,c)). 
This is consistent with the appearance of non-coplanar incommensurate AF (ICAF) order observed in neutron scattering 
because the ICAF order should diffuse the distribution of hyperfine field. 
On the other hand, the spectrum of 4 T maintains the paramagnetic Li1 and Li2 even below 38 K (Fig. \ref{0GPa_spectrum}(b)). 
This means that the field of 4 T destroys ICAF order and makes the system ferromagnetically polarized paramagnet. 
It is further confirmed by the saturated $K$ and almost zero $T^{-1}_1$ below 38 K (Fig. \ref{0GPa_Tdep_Kchi}(b,c)) 
because, in the polarized state, the magnetization is saturated and the fluctuation is suppressed.
We couldn't observe the signature for the mixing of zigzag component in the high-field state \cite{ruiz2017correlated}.

\begin{figure}
  \centering
  \includegraphics[scale=0.6]{0GPa_Tdep_Kchi.png}
  \caption{Ambient-pressure magnetic data of $\beta$-Li$_2$IrO$_3$.
  (a) magnetic susceptibility,$\chi = M/B$, at 1 T, (b) Knight shift $K$ of Li2, (c) spin-lattice relaxation rate $T^{-1}_1$ of Li2, (d) $K$-$\chi$ plot.
  Field is applied along b-axis for all measurement. 
  The temperature dependence of the shifts is well scaled by $\chi$ below 160 K [(a,b,d)], 
  where there is a uncertainity to determine high-$T$ Knight shift above 180 K because of the closeness of two lines [Fig.\ref{0GPa_spectrum}(d)].
  Thus, $K$-$\chi$ plot was done in the temperature range of 40 K $\leq$ $T$ $\leq$ 160 K, where shifts in 60 K $<$ $T$ $\leq$ 160 K are from 4 T measurement and 
  shifts in 40 K $\leq$ $T$ $\leq$ 60 K are from 1 T measurement.
  Both of $K$ and $T^{-1}_1$ of 4 T are suppressed from the values of 1 T below 60 K [(f,g)].
  It corresponds the suppression of the ICAF order by high field of 4 T.
  Saturated $K$ and almost zero $T^{-1}_1$ at 4 T below 38 K indicates the ferromagnetic polarization of $J_{\mathrm{eff}} = 1/2$ moments [(f,g)].}
  \label{0GPa_Tdep_Kchi}
\end{figure}

\begin{figure}
  \centering
  \includegraphics[scale=0.4]{0GPa_spectrum2.png}
  \caption{${}^7$Li NMR spectra at ambient-pressure.
  (a) ${}^7$Li NMR spectrum at 1 T, (b) spectrum at 4 T, (c) spectrum at 1 T in wider $K$ region, (d) spectrum at 4 T around 180 K.
   Field is applied along b-axis.
   Dotted lines are gaussian fitting results.
  Two lines corresponds to two Li sites, Li1 and Li2 [(a, b)]. 
  Below $T_{\mathrm{mag}}$ = 38 K, spectrum of 1 T is broadened [(a,c)], while two-line structure is preserved in spectrum of 4 T.
  This corresponds the appearance of ICAF order and the suppression of the order by high field of 4 T.
  There is a uncertainity to determine high-$T$ Knight shift above 180 K because it is difficult to precisely separate the close two lines as we can see the failure of double
  gaussian fitting [(d)].}
  \label{0GPa_spectrum}
\end{figure}

\newpage
\section{High pressure}
High-pressure magnetic susceptibility at 1 T along b-axis, $\chi_b=M_b/B_b$ is shown in Fig. \ref{mag_suscep} with the ambient pressure data. 
The magnetic transition point, $T_{\mathrm{mag}}$ = 38 K doesn't change up to 1.4 GPa (Fig. 2(a,b)). 
This means that ICAF ordered moments robustly exist under the pressure up to 1.4 GPa at 1 T. 
On the other hand, at the pressure of 1.1 and 1.4 GPa, we observed the anomalous drop of $\chi_b$ at $T_{\mathrm{d}}$ = 78 K and 154 K, respectively (Fig. \ref{mag_suscep}(b)). 
This indicates the appearance of the new high-pressure phase at $T_{\mathrm{d}}$. 
It has a smaller magnetization than paramagnetic phase. 
At 1.1 GPa and 1.4 GPa below $T_{\mathrm{d}}$, the new high-pressure phase and paramagnetic phase (or ICAF order phase below $T_{\mathrm{mag}}$) coexists. 
The pressure-dependence of $\chi_b$ at 6 K (Fig. \ref{mag_suscep} (a) inset) is consistent with reported XMCD \cite{takayama2015hyperhoneycomb} 
and $\chi$ for polycrystalline sample \cite{Majumder2018}, 
including the initial enhancement up to 0.9 GPa.

\begin{figure}
  \centering
  \includegraphics[scale=0.7]{mag_suscep4.png}
  \caption{(a) Magentic susceptibility $\chi$ = M/B at 1 T along b axis under various pressure. 
  (b) the enlarged view. The magnetic transition at $T_{\mathrm{mag}}$ = 38 K is unchanged and exists up to 1.4 GPa, implying the robust ICAF order phase under pressure. 
  The anomalous decrease of $\chi$ at $T_{\mathrm{d}}$ means the emergence of new high-pressure phase. 
  At 1.1 GPa and 1.4 GPa, the new phase and the paramagnetic phase (or ICAF order phase below $T_{\mathrm{mag}}$) coexists below $T_{\mathrm{d}}$.}
  \label{mag_suscep}
\end{figure}

To clarify the new high-pressure phase observed in $\chi_b$ measurement, we conducted high-pressure ${}^7$Li NMR measurement at 4 T along b-axis. 
Such a high field destroys ICAF order, but it gives us a larger intensity in NMR. 
The results at 3.5 GPa is shown in Fig. \ref{3.5GPa}. 
Above 300 K, the pseudo-spins are in paramagnetic state, evidenced by the spectrum distinguishable as paramagnetic Li1 and Li2 lines (Fig. \ref{3.5GPa} (a)) 
and the value of $T^{-1}_1$ comparable to that of paramagnetic phase at ambient pressure (Fig.\ref{3.5GPa} (c)). 
(In high-pressure spectra at high temperature, there is no failure of separating two peaks, which was observed at ambient pressure data.)
However, at $T_{\mathrm{d}}$ = 292 K, two lines merges to one line (Li$_\mathrm{d}$) with the almost zero value of $K$ (Fig. \ref{3.5GPa} (a), (b)). 
This means the transition to the non-magnetic state with almost zero susceptibility. 
Also, $T^{-1}_1$ suddenly decrease at $T_{\mathrm{d}}$ = 292 K (Fig. \ref{3.5GPa} (c)). 
The behavior agrees with the non-magnetic transition. 
In the temperature range of 200 K $<$ $T$ $<$ $T_{\mathrm{d}}$ = 292 K, $K$ and $T^{-1}_1$ still have finite values. 
It is considered that some defects in the non-magnetic phase produce magnetic moments around them in that temperature region. 
They can give finite $K$ and $T^{-1}_1$ to non-magnetic domain and make the transition step-wise. 
Below 200 K, both of $K$ and $T^{-1}_1$ are almost zero. 
This confirms the non-magnetic ground state at 3.5 GPa.

\begin{figure}[H]
  \centering
  \includegraphics[scale=0.7]{3p5GPa3.png}
  \caption{7Li NMR at 3.5 GPa. (a) spectrum, (b) Knight shift $K$, (c) spin-lattice relaxation rate $T^{-1}_1$.
  In (a), dotted lines are gaussian fits for each line (Li1, Li2, Li$_\mathrm{d}$).
  In (c), $T^{-1}_1$ at ambient pressure for Li2 is also plotted for comparison.
  $T^{-1}_1$ was measured for main Li$_\mathrm{d}$ line below $T_\mathrm{d}$ at 3.5 GPa.
  Above $T_\mathrm{d}$, it was measured for the average of Li1 and Li2 lines.
  At $T_{\mathrm{d}}$ = 292 K, paramagnetic Li1 and Li2 lines merge to one line (Li$_\mathrm{d}$)
  with almost zero $K$ and sudden drop of $T^{-1}_1$ occurs, indicating non-magnetic transition. }
  \label{3.5GPa}
\end{figure}

Fig. \ref{spectrum_highP} shows the spectrum at 1.3 GPa, 2.6 GPa, 3.5 GPa plotted in the same scale of $K$.
(a), (b), (c) are the spectra around transition point, $T_\mathrm{d}$, and (d), (e), (f) are the spectra in the whole temperature range.
The sharp Li$_\mathrm{d}$ line at 3.5 GPa resides at around $K$ = 0 (Fig.\ref{spectrum_highP}(f)). 
This means that hyperfine field for Li$_\mathrm{d}$ line at 3.5 GPa is less distributed around the zero value. 
It further confirms the non-magnetic ground state at 3.5 GPa.

For 2.6 GPa, the abrupt drop of $T^{-1}_1$ from the paramagnetic value occurs at $T_{\mathrm{d}}$ = 230 K (Fig. \ref{T1_highP}(a)). 
Correspondingly, the main line below $T_{\mathrm{d}}$ = 230 K has almost zero $K$ down to the lowest temperature (Fig. \ref{spectrum_highP} (b,e)). 
These suggest that the main line below $T_{\mathrm{d}}$ = 230 K represents the non-magnetic domain, same as one at 3.5 GPa. 
Thus, we name it as Li$_\mathrm{d}$ line. 
The Li$_\mathrm{d}$ line is broader than one at 3.5 GPa. 
This is probably caused by defect-induced moments in non-magnetic domain. 
They can diffuse the hyperfine field around zero value. 
There are the additional lines with positive $K$. 
They are considered as coexisting paramagnetic component (note that at 4 T, ICAF order is suppressed.) or defect-induced magnetic moments in non-magnetic domain. 
For the latter case, the concentration of defects is probably larger than one for the Li$_\mathrm{d}$ line enough to give a finite $K$.

For 1.3 GPa, the third line different from paramagnetic Li1 and Li2 splits from Li1 line at $T_{\mathrm{d}}$ = 90 K (Fig. \ref{spectrum_highP}(a)). 
$T^{-1}_1$ for the new main line start to decrease from the paramagnetic value of Li1 below $T_{\mathrm{d}}$ = 90 K (Fig. \ref{T1_highP}(a)). 
The transition temperatures, $T_{\mathrm{d}}$, are monotonously dependent on pressure. 
Also, it is common with 2.6, 3.5 GPa measurement that there is anomaly in spectrum at $T_{\mathrm{d}}$ and that $T^{-1}_1$ decrease below $T_{\mathrm{d}}$. 
Thus, we can assign the new main line as Li$_\mathrm{d}$ line, which represents the non-magnetic domain. 
However, the Li$_\mathrm{d}$ line is broad and has a finite $K$ (Fig. \ref{spectrum_highP}(d)). 
It is considered that the concentration of defect-induced moments in the non-magnetic domain is large enough to give the broad spectrum and finite $K$. 
Below $T_{\mathrm{d}}$ = 90 K, paramagnetic Li1 and Li2 coexists with Li$_\mathrm{d}$ line. 
This is consistent with the coexistence of paramagnetic component (or ICAF order moments below $T_\mathrm{mag}$ = 38 K) 
and the new high-pressure phase observed in $\chi_b$ at 1.1, 1.4 GPa below $T_{\mathrm{d}}$ = 78 K and 154 K, respectively (Fig. \ref{mag_suscep}(b)). 
Thus, we identify the new high-pressure phase found in 1 T $\chi_b$ measurement as non-magnetic phase confirmed in 4 T NMR measurement. 
Here, different from ICAF order, which is field-dependent, the non-magnetic phase should be field-independent. 
It is because the transition temperature $T_{\mathrm{d}}$ reaching to room temperature is much larger than the energy scale of 4 T. 
Finally, not only Li$_\mathrm{d}$ line but also Li1 and Li2 lines are broadened at low temperature. 
It is considered that some defects also exist in paramagnetic domain, diffuse the hyperfine field and make the paramagnetic state close to the glassy state. 
At 5 K, we couldn't observe Li1 line because of the broadening and the small $T_2$.

For 3.9 GPa, $T^{-1}_1$ (Fig. \ref{T1_highP}(a)), $K$ and spectrum (not shown) behaves similarly to those at 3.5 GPa. 
The only difference is the higher transition point, $T_{\mathrm{d}}$ = 315 K.

For all pressure, even below $T_{\mathrm{d}}$, there are temperature regions with finite $T^{-1}_1$ (Fig. \ref{T1_highP}(a)). 
For 1.3, 2.6 GPa, the regions extend to the lowest temperature. 
In these temperature regions, it is considered that defect-induced magnetic moments exist in non-magnetic phase and give finite dynamic magnetic response.

Fig. \ref{T1_highP}(b) is a log-scale plot of $T^{-1}_1$.
As indicated by arrow, temperature dependence in the almost zero value region at 3.5 and 3.9 GPa in linear-scale plot is not monotonous.
There is a pressure-independent peak structure.
It is possibly caused by defect.

\begin{figure}[H]
  \centering
  \includegraphics[scale=0.7]{spectrum_highP2.png}
  \caption{Spectra at 4 T along b-axis and at high pressure around transition temperature, $T_\mathrm{d}$, (a) 1.3 GPa, (b) 2.6 GPa, (c) 3.5 GPa,
  and in the whole temperature range, (d) 1.3 GPa, (e) 2.6 GPa, (f) 3.5 GPa.
  Li1, Li2 denote two Li sites in paramagnetic state. 
  Li$_\mathrm{d}$ denotes the line for non-magnetic state. 
  Dotted lines are gaussian fits for each line (Li1, Li2, Li$_\mathrm{d}$).  
  At 3.5 GPa, non-magnetic state with sharp line at $K$ = 0 \% was observed. 
  As lowering pressure, at 1.3, 2.6 GPa, the non-magnetic state (Li$_\mathrm{d}$) coexists with other components. 
  Also, at 1.3, 2.6 GPa, possibly, some defects produce the magnetic moments in the non-magnetic domain and make Li$_\mathrm{d}$ line broader. 
  At 1.3 GPa, they give even finite $K$ to Li$_\mathrm{d}$ line.}
  \label{spectrum_highP}
\end{figure}

\begin{figure}[H]
  \centering
  \includegraphics[scale=1.0]{T1_highP2.png}
  \caption{Spin-lattice relaxation rate, $T^{-1}_1$, at 4 T along b-axis and at each pressure. 
  Non-magnetic transition point, $T_\mathrm{d}$, is defined as a onset temperature of significant drop of $T^{-1}_1$.
  $T_\mathrm{d}$ monotonically increases as applying pressure.
  $T^{-1}_1$ was measured for main Li$_\mathrm{d}$ line below $T_\mathrm{d}$.
  Above $T_\mathrm{d}$, it was measured for Li1 line at 1.3 GPa and for the average of Li1 and Li2 lines at other pressure.
  (a) linear scale, (b) log scale.
  In (b), we can see the peak structure at 3.5 and 3.9 GPa as indicated by arrow, which is possibly caused by defect.}
  \label{T1_highP}
\end{figure}

\newpage
\section{Phase diagram}
The transition temperatures we obtained, $T_{\mathrm{mag}}$ and $T_\mathrm{d}$ for 1 T $\chi_b$ measurement, and $T_\mathrm{d}$ for 4 T NMR measurement, 
are plotted in $P-T$ phase diagram ignoring the difference of the field (Fig. \ref{phase}). 
Note that the high field above $\sim$ 2.7 T along b-axis switches the ICAF order phase to paramagnetic phase. 
The magnetic transition points, $T_{\mathrm{mag}}$, in $\chi$ of polycrystalline sample \cite{Majumder2018} 
and the structural transition point, $T_s$, in single crystal XRD \cite{veiga2017pressure} are also plotted. 
Consistent with the reported ones, the magnetic transition points, $T_{\mathrm{mag}}$, are almost pressure-independent. 
We didn't observe $T_{\mathrm{mag}}$ in high-pressure NMR measurement because it was done under 4 T. 
\begin{comment}
However, the coexisting component at 1.3 GPa is paramagnetic one. 
It implies that, at least at 1.3 GPa, $T_{\mathrm{mag}}$ should appear if the field is lowered below 2.7 T because the paramagnetic phase should become ICAF order moments below 2.7 T. 
\end{comment}
However, for 2.6 GPa, there is a possibility that the coexisting component is remaining paramagnetic phase and that $T_{\mathrm{mag}}$ can appear if the field is lowered below 2.7 T. 
For 3.5 GPa, there is no coexisting component and no possibility for the appearnce of ICAF order even if the field is lowered. 
Thus, we drew the horizontal dotted line for $T_{\mathrm{mag}}$ up to just below 3.5 GPa. 

We can smoothly connect $T_\mathrm{d}$ in $\chi_b$, $T_\mathrm{d}$ in NMR measurement and $T_s$ in single crystal XRD with one line. 
This means that the paramagnetic-to-non-magnetic transition we observed coincides with $F_{ddd}$-to-$C/2c$ structural transition. 
In the $C/2c$ phase, only one kind of Ir-Ir bond, emphasized in Fig. \ref{phase}, is 12 \% shorter than other bonds. 
Therefore, we conclude that the non-magnetic state is a singlet dimer state.

In the phase diagram, we denoted the coexisting region with dashed lines. 
It is composed of paramagnetic spins (or ICAF ordered momenets below $T_\mathrm{mag}$) and singlet dimers.
Our high-pressure $\chi$ data and NMR at 1.3 GPa indicates such a coexistence.
From NMR at 2.6 GPa, there are two possibilities for the coexisting component.
One is paramagnetic component and  the other is defect-induced moments in singlet dimer domain.
In the plot (Fig. \ref{phase}), we included the $P-T$ region at 2.6 GPa into dashed coexisting phase because of the ambiguity and in order to simplify the phase diagram.
The behavior of $T^{-1}_1$ below $T_\mathrm{d}$ is successive for 2.6 GPa, 3.5 GPa and 3.9 GPa, especially if we focus on the temperature region with the finite $T^{-1}_1$ value
(Fig. \ref{T1_highP}).
This indicates that those corresponding $P-T$ regions are the same phase.
Thus, those $P-T$ regions are included into dashed coexisting region.
As we saw in previous section, the effect of defects is prominent in the dashed region. 

\begin{figure}[h]
  \centering
  \includegraphics[scale=1.0]{phase19.png}
  \caption{Pressure-temperature phase diagram for $\beta$-Li$_2$IrO$_3$. 
  PM, ICAF means paramagnetic phase, incommensurate non-coplanar antiferromagnetic phase, respectively. 
  Dashed region is coexisting phase. 
  $\times$ is from chi at 1 T along b-axis and $\bullet$ is from 4 T NMR. 
  $\circ$ is from chi at 1 T for polycrystalline sample \cite{Majumder2018} and $\blacktriangledown$ comes from XRD \cite{veiga2017pressure}. 
  Note that ICAF order is destroyed and substituted to PM at 4 T. (Inset) crystal structures for each phase. 
  In $F_{ddd}$ phase at ambient pressure, all bonds have almost equivalent lengths \cite{takayama2015hyperhoneycomb}. 
  In $C2/c$ phase, the emphasized bonds are 12 \% shorter than other bonds \cite{veiga2017pressure}.}
  \label{phase}
\end{figure}

\chapter{Discussion}
\section{Quantum critical point at ambient pressure}
The size of the low-$T$ magnetic moment at 4 T is $\sim$ 0.4 $\mu_B$ (Fig.\ref{Bdep_mag}), which is still less than 1.6 $\mu_B$ at paramagnetic phase.
The magnetic moments are not fully polarized at 4 T.
This means that there is still a magnetically fluctuating component in the high-field regime.
In fact, the magnetization is not saturated at 4 T (Fig.\ref{Bdep_mag}).
The problem is whether the fluctuating component can be understood as spin liquid or not.
In the case of field-induced spin liquid, $\alpha$-RuCl$_3$, the size of the low-$T$ magnetic moment at 8 T is $\sim$ 0.55 $\mu_B$ \cite{Zheng2017}.
In that sense, there is a possibility that the unpolarized component in $\beta$-Li$_2$IrO$_3$ can be a spin liquid.
It has a spin gap of 90 $\pm$ 20 K (Fig.\ref{spin_gap}(b)).

Here, spin gap fitting was done with the function,
\begin{equation}
T^{-1}_1 = b\mathrm{e}^{-\Delta/T},
\end{equation}
where $b$ and spin gap $\Delta$ are fitting parameters, and $T$ is temperature.
Temperature range used for fitting is summarized in Table.\ref{Trange}.
For the pressure of 2.6, 3.5, 3.9 GPa, we estimated the spin gap by the fitting to the decrease from the intermidiate value to almost zero value.

For $\alpha$-RuCl$_3$, there are two reports that one claim the gapless quantum spin liquid \cite{Zheng2017} and the other claim the gapful quantum spin liquid \cite{baek2017evidence}.
However, the common feature is that it become gapless around the quantum critical point.
This might mean that the gapped behavior we observed at 4 T is too far away from the quantum critical point.
In $\alpha$-RuCl$_3$, exotic behavior like quantized thermal Hall effect is only reported around quantum critical point $\sim 8$ T (Fig.\ref{phase_BT}(b)) \cite{kasahara2018majorana}.
Our result might suggest that, in order to observe quantized thermal Hall effect in $\beta$-Li$_2$IrO$_3$, we need to finely adjust the field between 2.7 T and 4 T 
(Fig.\ref{phase_BT} (a)).
On the other hand, quantized thermal Hall effect might be originating from two dimensionality of spin liquid state.
If it is the case, we cannot observe quantized thermal Hall effect in three dimensional $\beta$-Li$_2$IrO$_3$.
Instead, if 3D Kitaev spin liquid is realized in $\beta$-Li$_2$IrO$_3$ at the field between 2.7 and 4 T, there should be a topological phase transition from paramagnetic state to
spin liquid state \cite{nasu2014vaporization}.

Thus, our result gives the upper bound (4 T) to observe the possible field-induced exotic quantum criticality in $\beta$-Li$_2$IrO$_3$.

\begin{figure}
  \centering
  \includegraphics[scale=0.7]{spin_gap_both.png}
  \caption{(a) Spin gap fitting for $T^{-1}_1$.
  (b) Pressure dependence of the spin gap.
  Spin gap $\Delta$ is enhanced for the dimerization above 3.5 GPa.} 
  \label{spin_gap}
\end{figure}

\begin{table}[H]
\begin{center}
\caption{Temperature range used for spin gap fitting.}
\begin{tabular}{ccc} \hline
 $P$ (GPa)& $T_\mathrm{start}$ (K)& $T_\mathrm{end}$ (K)\\ \hline
 0 & 20 (lowest)& 70\\ \hline
 1.3& 40 (lowest)& 110\\ \hline
 2.6& 20 (lowest)& 225\\ \hline
 3.5& 160& 250\\ \hline
 3.9& 180& 290\\ \hline
\end{tabular}
\label{Trange}
\end{center}
\end{table}

\begin{figure}
  \centering
  \includegraphics[scale=0.7]{phase_BT.png}
  \caption{Field-temperature phase diagram (a) for $\beta$-Li$_2$IrO$_3$ \cite{ruiz2017correlated}
  and (b) for $\alpha$-RuCl$_3$ \cite{kasahara2018majorana}.
  Quantum critical behavior is observed in $\alpha$-RuCl$_3$ in the red shaded limited region.
  It might also be true for $\beta$-Li$_2$IrO$_3$.
  Region with dashed circle below 4 T have a possibility for quantum critical behavior.
  In (a), we added the words, "measured", "spiral AFM", and "Quantum critical behavior?".} 
  \label{phase_BT}
\end{figure}

\section{Strong singlet dimer formation above 3.5 GPa}
Now we examine the three possibilities for the non-magnetic state of the dimer (Fig.\ref{three}).
Knight shift in our singlet dimer state at 3.5 GPa is almost zero (Fig.\ref{3.5GPa}).
It means almost zero susceptibility in the dimer state.

Fig.\ref{chi_cal} is a calculated magnetic susceptiblity $\chi$ for the dimer of $J_{\mathrm{eff}} = 1/2$ spins \cite{Nasu2014}.
$t'$ represents the direct $dd$ hopping integral in the scale of indirect $dpd$ hopping. 
$t' < 1$ means stronger indirect $dpd$ hopping and it stabilizes quadrupolar state.
We can see that the quadrupolar state has a finite $\chi$.
Thus, quadrupolar state is excluded because we observed almost zero susceptibility in the dimer state.

Then, we examine the possibilty of $J_{\mathrm{eff}} = 1/2$ spin singlet.
In order to form $J_{\mathrm{eff}} = 1/2$ spin singlet, we need AF Heisenberg interaction.
However, the first principle calculation revealed that Heisenberg interaction is FM even under pressure \cite{Yadav2018, Kim2016}.
This makes the $J_{\mathrm{eff}} = 1/2$ spin singlet scenario less plausible.
There is another negative discussion for the scenario.
Suppose that $J_{\mathrm{eff}} = 1/2$ spin singlet is formed by exchange interaction.
There should be the singlet-triplet gap of the order of the interaction. 
In $\beta$-Li$_2$IrO$_3$, the calculation revealed that the exchange interactions are of the order of $\sim$ 100 K \cite{Yadav2018}. 
The gap should give a finite van Vleck term in $\chi$ and correspondingly finite $K$.
In fact, the calculation of magnetic susceptibility $\chi$ for $J_{\mathrm{eff}} = 1/2$ spin singlet revealed that $\chi$ has finite value at the temperature scale of 
the energy gap (in our case $\sim 100$ K), while it decays to zero value at lower temperature \cite{Nasu2014}.
In Fig.\ref{chi_cal}, $J_{\mathrm{eff}} = 1/2$ singlet state corresponds to $t' > 1$ case, which means stronger direct $dd$ hopping.
However, we observed almost zero $K$ for singlet dimer phase. 
The discrepancy means that $J_{\mathrm{eff}} = 1/2$ spin singlet is not the case. 

\begin{figure}
  \centering
  \includegraphics[scale=0.7]{chi_cal.png}
  \caption{Calculated magngetic susceptibility $\chi$ for dimer state of $J_{\mathrm{eff}} = 1/2$ spins.
  $\beta$ is inverse temperature. \cite{Nasu2014}
  (a) $\chi$
  (b) $\mathrm{d}\chi/\mathrm{d}\beta$
  $t'$ represents the direct $dd$ hopping integral in the scale of indirect $dpd$ hopping. 
  $t' < 1$ means quadrupolar state and $t' > 1$ means singlet state.
  The singlet state has a finite peak structure at the temperature scale of singlet-triplet gap.}
  \label{chi_cal}
\end{figure}

Other possibility is the collapse of $J_{\mathrm{eff}} = 1/2$ picture and the formation of molecular orbital. 
Fig.\ref{band_cal_highP} is the calculated density of states (DOS) for ambient-pressure $Fddd$ structure and for high-pressure $C/2c$ structure at 4.4 GPa (Fig.\ref{band_cal_highP})
\cite{takayama2018pressure}.
At ambient pressure, we can observe the splitting of $t_{2g}$ states to $J_{\mathrm{eff}} = 1/2$ and $J_{\mathrm{eff}} = 3/2$ states.
On the other hand, at high-pressure $C/2c$ phase, $d_{xy}$, $d_{yz}$, $d_{zx}$ character appears.
The lowest occupied sub-band (-1.7 eV) and the highest unoccupied sub-band (0.7 eV) are dominated by $d_{zx}$ character.
It means the formation of bonding-antibonding state of $d_{zx}$ orbital.
$d_{zx}$ orbital is directed along dimerized bond (Y bond).
The degenerate $d_{xy}$, $d_{yz}$ orbitals are entangled by SOC and reside between -1.7 eV and 0.7 eV.
Thus, the dimer state is a molecular orbital of $d_{zx}$ orbital. 
The gap induced by the formation of the bonding-antibonding state is estimated as $\sim$ 7000 K. 
This is consistent with our zero $K$ because such a large excitation gap gives negligible van Vleck susceptibility.
If we assume the similar behavior of $\chi$ to $J_{\mathrm{eff}} = 1/2$ spin singlet, $\chi$ has a finite value at $\sim 7000$ K and decays to zero at lower temperature,
which should give the almost zero $\chi$ in our measuring temperature range.

Thus, our results support the molecular orbital formation for the non-magnetic dimer state. 

Also, the large spin gap $\sim 2000$ K (Fig.\ref{spin_gap}) above 3.5 GPa is consistent with our observeation.
The strong singlet dimer formation is further supported by the high transition temperature reaching to room temperature and the fact that it is first-order transition.

$J_{\mathrm{eff}} = 1/2$ state is obtained by equally mixing up the $d_{xy}$, $d_{yz}$, $d_{zx}$ orbitals with multiplying a complex factor. 
We can view the transition from $J_{\mathrm{eff}} = 1/2$ state to $d_{zx}$-molecular orbital as the pressure-induced orbital selection.

\begin{figure}
  \centering
  \includegraphics[scale=0.7]{band_cal_highP.png}
  \caption{Calculated density of states (DOS) for Ir. \cite{takayama2018pressure}
  (a) DOS for the ambient-pressure structure ($Fddd$).
  $t_{2g}$ states split to $J_{\mathrm{eff}} = 1/2$ and $J_{\mathrm{eff}} = 3/2$ states.
  (b) DOS for the high-pressure structure at 4.4 GPa ($C/2c$).
  $d_{zx}$ orbital forms bonding-antibonding state.
  $d_{xy}$ and $d_{yz}$ are entangled by SOC.
  Total DOS contains O $2p$ states.}
  \label{band_cal_highP}
\end{figure}


XRD and neutron scattering revealed that the zigzag chains compress easier than the bridging bonds (Fig.\ref{bond}) \cite{veiga2017pressure, takayama2018pressure}. 
The first principle calculation suggest that it gives the larger enhancement of direct hopping along zigzag chain than one in bridging bond \cite{Kim2016}.
It is considered that the anisotropic enhancement of hopping contributes to the stabilization and formation of $d_{zx}$-molecular orbital.

\begin{figure}
  \centering
  \includegraphics[scale=0.7]{bond.png}
  \caption{Pressure dependence of bond length \cite{veiga2017pressure, takayama2018pressure}. X, Y bond composing zigzag chain compresses easier than bridging Z bond.}
  \label{bond}
\end{figure}

\section{Competition between spin-orbit couping and dimer formation}
As we discussed in Chap.\ref{result}, there is a coexisting phase of $J_\mathrm{eff} = 1/2$ spins and singlet dimer in the intermidiate pressure range (1 GPa $< P <$ 3 GPa)
(Fig.\ref{pressure_phase}).
In Chap.\ref{result}, we presented the interpretaion that singlet dimer state in the intermidiate region (1.3 and 2.6 GPa) is affected by defects.
However, we can also have a speculation that the dimer state at 1.3 and 2.6 GPa is different from molecular orbital dimer.
Quadrupolar dimer state can still be excluded because it cannot be continuously connected to $S = 1/2$ molecular dimer state above 3.5 GPa.
They belong to different symmertries.
Here, we mean the possibility of $J_\mathrm{eff} = 1/2$ pseudo-spin singlet at 1.3 and 2.6 GPa.
It can be continuously connected to $S = 1/2$ molecular dimer state above 3.5 GPa.
In this view, finite $K$ for Li$_\mathrm{d}$ peak at 1.3 GPa (Fig.\ref{spectrum_highP}(d)) is originated from van Vleck term of modest singlet-triplet gap.
One of the coexisting component at 2.6 GPa (Fig.\ref{spectrum_highP}(e)) is also understood as $J_\mathrm{eff} = 1/2$ pseudo-spin singlet with finite $K$.
However, XRD at high pressure and at low temperature is now revealing that the structure of dimer at lower pressure region is the same as one at higher pressure region
(private communication).
This suggests the same dimer state for the overall $P-T$ phase diagram.
The conclusive results from high-pressure XRD is awaited.

What we found in $\beta$-Li$_2$IrO$_3$ under pressure is a competition between spin-orbit Mott state and molecular dimer state.
The former is characterized by the energy gain of SOC and Coulomb $U$, while the latter is characterized by the energy gain of hopping amplitude $t$.
If there is no SOC in $\beta$-Li$_2$IrO$_3$, it shows dimerization in Z-bond \cite{Kim2016}.
Note that, in reality, the dimerization in $\beta$-Li$_2$IrO$_3$ occurs in Y-bond.
$\alpha$-RuCl$_3$ also shows dimerization without SOC \cite{Kim2016a}.
These mean that there is an intrinsic instability for orbital-selective dimerization among these Kitaev magnets and strong SOC prevents the dimerization 
by mixing the orbitals at ambient pressure.
Pressure modifies the subtle balance by increasing hopping integrals, resulting in the transition from $J_\mathrm{eff} = 1/2$ state to dimerized state.
On the other hand, Li$_2$RuO$_3$ ($4d^4$) shows dimerizatoin under ambient pressure \cite{Miura2007}.
This might mean that not only SOC but also spin-orbit Mottness of $J_\mathrm{eff} = 1/2$ state is a key to prevent dimerization.

If the pressure is moderate and the system is just in a competitive region, there should be the incomplete dimer formation.
In that region, the pairing partner for dimer is not fully determined.
Fluctuation to find the dimer pair can be regarded as the similar state to resonating valence bond (RVB) state \cite{ANDERSON1973153}.
In this sense, there might be RVB-like spin liquid state in the competitive region.

\begin{figure}[H]
  \centering
  \includegraphics[scale=0.7]{pressure_phase.png}
  \caption{Schematic image of pressure dependence of the phase in $\beta$-Li$_2$IrO$_3$.
  Singlet dimer in the intermidiate pressure region is affected by defect or different from strong singlet dimer at higher pressure.}
  \label{pressure_phase}
\end{figure}

\section{Effect of defect and $\mu$SR}
Fig. \ref{lowQ} shows the crystal-quality dependence of $\chi_b$ and $T^{-1}_1$.
Crystal 1 is supplied by a collaborator.
Crystal 2 is obtained by the method described in Chap.\ref{method}.
The kink at $T_\mathrm{mag} = 38$ K disappears for crystal 1 (Fig.\ref{lowQ}(a)).
It is considered that the large amount of defect disturbs ICAF order for crystal 1.
The effect of defects is more prominently observed in $T^{-1}_1$ (Fig.\ref{lowQ}(b)).
$T^{-1}_1$ for crystal 1 at 3.2 GPa and at 7 T doesn't show non-magnetic transition.
Instead, it shows peak structure around 40 K.
The low-$T$ value is two order of magnitude larger than that for crystal 2 at 3.5 GPa.
The peak structure is also observed for crystal 1 at 3 T. 
This means the anomalous behavior in $T^{-1}_1$ is originating not from the difference in applied field but from the sample quality.
It is considered that the high concentration of defects in crystal 1 produces dynamical spins.

$\mu$SR at high pressure concluded that there is a coexistence of liquid-like dynamical spins and glass-like frozen spins 
in the pressure range from 1.37 GPa to 2.27 GPa below $T_{\mathrm{mag}}$ = 38 K \cite{Majumder2018}. 
In the $P$-$T$ region, our measurement showed the coexistence of ICAF ordered spins, singlet dimers and defects. 
The discrepancy is possibly caused by the amount of defect. 
It is considered that the large amount of defect in ICAF order phase make it close to glassy state
as we observed the diffuse of hyperfine field for paramagnetic component (or ICAF order moments in low-field regime) at 1.3 GPa and at low temperature.
Also, the high concentration of defect in singlet dimer phase can produce dynamical spins.
In fact, $T^{-1}_1$ measured for low-quality sample at 3.2 GPa was two-order of magnitude larger than that at 3.5 GPa (Fig.\ref{lowQ}(b)). 
This indicates the existence of dynamical spins can be strongly dependent on sample quality. 

\begin{figure}[H]
  \centering
  \includegraphics[scale=0.7]{lowQ.png}
  \caption{Effect of sample quality.
  (a) Magnetic susceptibility at 1 T along b-axis of $\beta$-Li$_2$IrO$_3$ single crystals.
  The high-quality crystal, crystal 2, has higher and clearer magnetic transition point, $T_\mathrm{mag} = 38$ K, than crystal 1.
  (b) Spin-lattice relaxation rate, $T^{-1}_1$.
  Non-magnetic transition is absent for low-quality crystal, crystal 1.
  Instead the large value of $T^{-1}_1$ is observed in crystal 1, indicating the existence of defect-induced dynamical spins.} 
  \label{lowQ}
\end{figure}


\section{Comparison with other compounds}
There are not so many iridates which shows valence bond solid.
CuIr$_2$S$_4$ shows structural transition with octamer formation at $T_s \sim 230$ K and at ambient pressure \cite{Radaelli2002}.
Spin-lattice realaxation rate, $T^{-1}_1$, in Cu-NMR shows abrupt drop at the transition point (Fig.\ref{T1_CuIr2S4}) \cite{Tsuji1997}.
Knight shift also change the sign at the transition point from negative above $T_s$  to positive below $T_s$ (Fig.\ref{K_CuIr2S4}) \cite{Tsuji1997}.
Knight shift below $T_s$ has finite value possibly due to orbital contribution.
This is a metal-insulator transition.
There is a discussion about mechanism that one-dimensional hopping originating from anisotropy of $t_{2g}$ orbitals induces Peierls instability and octamer formation
(orbitally-induced Peierls mechanism) \cite{Khomskii2005}, while it ignores the effect of SOC.
Our singlet dimer formation in $\beta$-Li$_2$IrO$_3$ is insulator-to-insulator transition, so it is different from CuIr$_2$S$_4$.
However, also in $\beta$-Li$_2$IrO$_3$, pressure induce anisotropic compression and anisotropic enhancement of hopping, especially in zigzag chain 
\cite{takayama2018pressure, veiga2017pressure, Kim2016}.
This might mean that the one-dimensionality in hopping is induced by pressure, resulting in Peierls instability and singlet dimer formation within zigzag chain.
This pressure-induced one-dimensionality scenario is one possibility to explain the singlet dimer formation in $\beta$-Li$_2$IrO$_3$.

\begin{figure}
  \centering
  \includegraphics[scale=0.7]{T1_CuIr2S4.png}
  \caption{$T^{-1}_1$ of CuIr$_2$S$_4$ ($x = 0$) \cite{Tsuji1997}. It shows abrupt drop at $T_s \sim 230$ K.} 
  \label{T1_CuIr2S4}
\end{figure}

\begin{figure}
  \centering
  \includegraphics[scale=0.7]{K_CuIr2S4.png}
  \caption{Knight shift of CuIr$_2$S$_4$ ($x = 0$) \cite{Tsuji1997}. Sudden sign change is observed at $T_s \sim 230$ K.} 
  \label{K_CuIr2S4}
\end{figure}

IrTe$_2$ is another example which shows dimer formation at $T_s \sim 280$ K and at ambient pressure \cite{Pascut2014}.
Spin-lattice realaxation rate, $T^{-1}_1$, in ${}^{125}$Te-NMR shows abrupt drop at the transition point (Fig.\ref{T1_IrTe2}).
This is a metal-to-metal transition.
Both of CuIr$_2$S$_4$ and IrTe$_2$ are mixed-valnent system with formal Ir$^{3.5+}$ valnece \cite{Radaelli2002, Pascut2014}.
Charge ordering accompanies with octamer or dimer formation for both compounds.
In that sense, singlet dimer formation in $\beta$-Li$_2$IrO$_3$ without charge fluctuation is rare case among dimerization in iridates.

\begin{figure}
  \centering
  \includegraphics[scale=0.7]{T1_IrTe2.png}
  \caption{$T^{-1}_1$ of IrTe$_2$ \cite{Mizuno2002}. It shows abrupt drop at $T_s \sim 280$ K.} 
  \label{T1_IrTe2}
\end{figure}

$\alpha'$-NaV$_2$O$_5$ shows dimer formation of V$^{4+}$ at $T_{SP} = $ 34 K and at ambient pressure (spin-Peierls transitoin) \cite{Isobe1996}.
Here, spin-Peierls transition is a singlet dimer formation observed in one-dimensional $S = 1/2$ spin system, which is induced by AF Heisenberg interaction.
The absolute value of Knight shift in ${}^{23}$Na-NMR shows the reduction of almost 1/10 from high-$T$ paramagnetic state to low-$T$ singlet dimer state (Fig.\ref{K_NaV2O5}).
In our singlet dimer formation in $\beta$-Li$_2$IrO$_3$, the reduction of Knight shift from paramagnetic to singlet state is almost 1/100 at 3.5 GPa (Fig.\ref{3.5GPa}(b)).
The large reduction in $\beta$-Li$_2$IrO$_3$ indicates that the dimer formation is stronger than spin-Peierls transition in $\alpha'$-NaV$_2$O$_5$.
In other words, the excitaion gap induced by singlet dimer formation in $\beta$-Li$_2$IrO$_3$ is far larger than singlet-triplet gap in $\alpha'$-NaV$_2$O$_5$.
This comparison also guarantees that the singlet dimer in $\beta$-Li$_2$IrO$_3$ originates from molecular orbital dimer, not from singlet induced by exchange interaction.

\begin{figure}
  \centering
  \includegraphics[scale=0.7]{K_NaV2O5.png}
  \caption{Knight shift of $\alpha'$-NaV$_2$O$_5$ \cite{Ohama1997}. 
  Field is along b-axis.
  The absolute value of Knight shift decrease at $T_{SP} \sim 34$ K.
  The amount of the decrease is smaller than that at $T_\mathrm{d} = 292$ K in $\beta$-Li$_2$IrO$_3$ under 3.5 GPa.} 
  \label{K_NaV2O5}
\end{figure}

The pressure-induced lattice dimerization was also reported for other Kitaev magnets, $\alpha$-Li$_2$IrO$_3$ \cite{Hermann2018}, $\alpha$-RuCl$_3$ \cite{Bastien2018}. 
Magnetism of the high-$P$ phase in $\alpha$-Li$_2$IrO$_3$ is still unknown though (Fig.\ref{phase_PT}(b)).
For $\alpha$-RuCl$_3$, the high-pressure magnetism was revealed to be a singlet dimer and similar phase diagram to $\beta$-Li$_2$IrO$_3$ was discovered (Fig.\ref{phase_PT}(a)).
It is considered that similar mechanism of molecular orbital formation is valid also for the dimerization in $\alpha$-RuCl$_3$.
For Na$_2$IrO$_3$, it is controversial whether it has a structural phase transition at high pressure \cite{Hermann2017, Xi2018}.
The discrepancy possibly comes from the air- and moisture-sensitivity of the compound.
However, at least, the reported transition is moderate and is not the dimerization \cite{Xi2018}. 
$\gamma$-Li$_2$IrO$_3$ do not show any structural phase transition under pressure \cite{breznay2017resonant}. 
The magnetic order disappears at $\sim$ 1.4 GPa without any lattice distortion. 
These indicate that there is a general tendency of pressure-induced singlet dimer formation in Kitaev magnets, 
but the appearance is highly sensitive to chemical composition or structure. 
It is necessary to reveal the whole $P-T$ phase diagram for all the Kitaev magnets and to elucidate the condition of appearance or disappearance of singlet dimer formation.

\begin{figure}
  \centering
  \includegraphics[scale=0.7]{phase_PT.png}
  \caption{Pressure-temperature phase diagram (a) for $\alpha$-RuCl$_3$ \cite{Bastien2018} and (b) for $\alpha$-Li$_2$IrO$_3$ \cite{Hermann2018}.
  Phase diagram for $\alpha$-RuCl$_3$ is similar to one for $\beta$-Li$_2$IrO$_3$.} 
  \label{phase_PT}
\end{figure}

\chapter{Summary}
Since the discovery of $J_\mathrm{eff} = 1/2$ pseudo-spin in Sr$_2$IrO$_4$ \cite{kim2008novel, kim2009phase}, 
interplay between SOC and correlation has provided the fascinating research area.
One of the most intensively studied themes is a realizatoin of Kiteav spin liquid in $J_\mathrm{eff} = 1/2$ pseudo-spin system \cite{kitaev2006anyons, jackeli2009mott}.
For Kitaev spin liquid, there are various exotic theoretical predition such as application to quantum computing \cite{kitaev2006anyons}, 
realization of fractional excitation \cite{nasu2015thermal, yoshitake2017temperature}
and appearance of exotic superconductor by hole doping \cite{you2012doping}.
$\beta$-Li$_2$IrO$_3$ was expected to realize the Kitaev spin liquid under high pressure.
It is because magnetic order of $J_\mathrm{eff} = 1/2$ at ambient pressure is suppressed by pressure without metallization \cite{takayama2015hyperhoneycomb}.
Also, there is a $\mu$SR report that the high-pressure phase is a mixture of spin liquid and spin glass  \cite{Majumder2018}.
Thus, we started the study of high-pressure magnetism of $\beta$-Li$_2$IrO$_3$.

We synthesized the largest $\beta$-Li$_2$IrO$_3$ single crystals.
At ambient pressure, we confirmed that the magnetic order in low-field regime becomes ferromagnetically polarized paramagnet in high-field regime.
We measured the high-pressure magnetism up to 3.9 GPa through magnetic susceptibility and ${}^7$Li NMR. 
We revealed $P$-$T$ phase diagram. 
The high-pressure phase is a non-magnetic singlet dimer probably based on molecular orbital. 
Also, we concluded that spin-liquid component observed in $\mu$SR is defect-induced dynamical moments in singlet dimer background. 

We revealed that there is a pressure-induced competition between orbital-disordered $J_{\mathrm{eff}} = 1/2$ state and orbital-selective molecular dimer state in $\beta$-Li$_2$IrO$_3$.
In the competitive region, pairing partner of dimer fluctuates and RVB-like spin liquid might realize.
The competition seems to be a common feature among some Kitaev magnets. 
Uncovering of $P$-$T$ phase diagram for all Kitaev candidates is awaited.


\appendix
\chapter{Simulation in paramagnetic phase \& angular rotation}
\label{appendix_simu}
In this chapter, based on the calculation of Eq.(\ref{Li1}) and Eq.(\ref{Li2}), we simulate the peak splitting in ${}^7$Li NMR of $\beta$-Li$_2$IrO$_3$ in paramagnetic phase.
Since $+ 0 .04$ kOe is one order smaller than the values in Eq.(\ref{Li1}) and Eq.(\ref{Li2}), it is ignored in the calculation below.
$\pm$ in Eq.(\ref{Li1}) and Eq.(\ref{Li2}) means that there are four Li sites as a hyperfine tensor, even though crystallographically there are two sites.  
Since Knight shift is a strength of internal field parallel to applied external field, it can be calculated as, 
\begin{align}
K_i = \frac{\vec{n}\cdot\overrightarrow{H}_{hf}(i)}{H_0}
\end{align}
In paramagnetic phase, using Eq.(\ref{Bhf_n}), it is rewritten as,
\begin{align}
K_i = \frac{p}{B_0}\vec{n}\cdot\tilde{A}(i)\cdot\vec{n} = \frac{p}{B_0}M(i, \vec{n})
\end{align}
where $B_0 = \mu_0H_0$, $M (i,\vec{n})$ is a angular dependent part of Knight shift defined by the above.
$M (i,\vec{n})$ is plotted in Fig.\ref{angle_dep} (b),
where the angles are defined as the deviation from magnetic easy b axis (Fig.\ref{angle_dep}(a)).
%Fig. (a) definition of angles & (b) angle dependence 
\begin{figure}
  \centering
  \includegraphics[scale=0.7]{angle_dep.png}
  \caption{(a) Angles $\theta$, $\phi$ are defined as the deviation from the magnetic easy b axis.
  (b) Angle dependent part of Knight shift, $M (i, \theta, \phi)$. 
  Li1(2)$\pm$ correspond in $\pm$ in Eq.(\ref{Li1}) and Eq.(\ref{Li2}).}
  \label{angle_dep}
\end{figure}
If we apply field in arbitrary direction, spectrum splits to four lines in accordance with four hyperfine Li sites.
However if the field is in bc-plane or ac-plane, spectrum consists of two lines corresponding to two crystallographic Li sites.

This observation was utilized when we confirmed that the b-axis is parallel to applied field in NMR measurement.
First, I determined crystal direction by single crystal XRD and set it carefully to orient b-axis to be parallel to applied field. 
In NMR measurement, spectrum with two-line structure was observed as expected for b-axis.
I named this original direction as b0 direction and correspondingly named the orthogonal directions as c0, a0.
Then, I conducted angular rotation with two-axis rotator from b0 to c0 axis to find the direction which shows the maximum splitting.
It is because b axis is a magnetic easy axis and the largest magnetization, i.e., splitting is expected.
I named this newly found direction as b1 direction, and correspondingly named the orthogonal directions as c1, a1.
(Note that a1 is the same as a0.)
Finally, I rotated from b1 to a1 axis and found the angle with maximum splitting again.
Larger angle rotation in a1b1 plane gives the further splitting of two-line structure as expected for ab-plane rotation.
As a result, I could confirm that the b axis is oriented to applied magnetic field.

The example of such an angular rotation is presented in Fig, \ref{b0c0} for b0-c0 rotation and in Fig. \ref{b1a1} for b1-a1 rotation.
In this case, I found b1 direction ($\theta = 95^\circ$) which shows slightly larger splitting than the original b0 ($\theta = 90^\circ$) direction in b0-c0 rotation (Fig.\ref{b0c0}).
Then the splitting is insensitive to a1b1 rotation in smaller angle range ($-10^\circ \leq \phi^\prime \leq 0^\circ$).
This indicates that b1 direction is close enough to angle region with maximum splitting. 
On the other hand, at larger angle ($\phi^\prime \leq -15$), the Li2 line splits to two lines. 
This confirms that the smaller angle region is closer to true b axis.
Therefore, for the measurement, I chose b1 as the direction close to b axis. 
The accuracy should be roughly $\pm$ $5^\circ$ because the angular dependence was measured in the increment of $5^\circ$ around b axis.

%Fig. spectrum & splitting
\begin{figure}
  \centering
  \includegraphics[scale=0.7]{ang_dep_b0c0_2.png}
  \caption{(a) Angle dependence of spectrum at 0 GPa, 1 T, 50 K for the rotation in b0-c0 plane.
  (b) Angle dependence of the split in spectrum. The splitting was read by the eye.
  Note that the definition of the angle $\theta$ is not based on Fig. \ref{angle_dep} (a).
  It is defined as the angle within b0-c0 plane in reference of the real two-axis rotator geometry (Appendix.\ref{two-axis}).}
  \label{b0c0}
\end{figure}

\begin{figure}
  \centering
  \includegraphics[scale=0.7]{ang_dep_b1a1_2.png}
  \caption{(a) Angle dependence of spectrum at 0 GPa, 1 T, 50 K for the rotation in b1-a1 plane.
  (b) Angle dependence of the split in spectrum. 
  The splitting was read by the eye.
  The splitting was measured from the leftmost peak.
  Note that the definition of the angle $\phi^\prime$ is not based on Fig. \ref{angle_dep} (a).
  It is defined as the angle within b1-a1 plane in reference of the real two-axis rotator geometry (Appendix.\ref{two-axis}).}
  \label{b1a1}
\end{figure}

\chapter{Two-axis rotation}
\label{two-axis}
To conduct the two-axis rotation of high-pressure cell (or ambient-pressure setup) in NMR measurement probe, we need to understand the mathematical geometry.
Let's define ($\hat{x}, \hat{y}, \hat{z}$) coordinate system for the probe and ($\hat{x}^\prime, \hat{y}^\prime, \hat{z}^\prime$) coordinate system for the cell 
(Fig.\ref{two_axis}(a)).
Field is applied along $\hat{z}$ axis.
($\hat{x}, \hat{y}, \hat{z}$) coordinate system is fixed to the laboratory and never move.
($\hat{x}^\prime, \hat{y}^\prime, \hat{z}^\prime$) coordinate system rotates against ($\hat{x}, \hat{y}, \hat{z}$) coordinate system.
The rotation can be done around two axes.
One is fixed $\hat{x}$ axis ($\theta$ rotation) and  the other is moving $\hat{z}^\prime$ axis ($\phi$ rotation) (Fig.\ref{two_axis}(b)).
%Figure of probe and cell, definition of theta and phi
\begin{figure}
  \centering
  \includegraphics[scale=0.7]{two_axis.png}
  \caption{(a) Initial position without rotation.
  (b) The schematic view of the cell rotation of $\theta$ and $\phi$.}
  \label{two_axis}
\end{figure}
%The crystal to measure is in the cell and its position is fixed against the cell.
Suppose, at the initial cell position, those two coordinate systems correspond.
The way of two-axis rotation of the cell gives the relation between the coordinate systems,
%matrix R(theta, phi)
\begin{align}
\label{coordinates}
(\hat{x}^\prime, \hat{y}^\prime, \hat{z}^\prime) &= (\hat{x}, \hat{y}, \hat{z}) R(\theta, \phi),\\
R(\theta, \phi) &=
\left(
\begin{array}{ccc}
\cos\phi & -\sin\phi & 0 \\
\cos\theta\sin\phi & \cos\theta\cos\phi & \sin\theta \\
-\sin\theta\sin\phi & -\sin\theta\cos\phi & \cos\theta
\end{array}
\right).
\end{align}
If a, b, c axes of the crystal perfectly correspond to $\hat{x}^\prime, \hat{y}^\prime, \hat{z}^\prime$ direction of the cell, $(\theta,\phi) = (\frac{\pi}{2},0)$,
(0,0), $(\frac{\pi}{2}, \frac{\pi}{2})$ gives the measurement of b, c, a axes, respectively.
The $\theta$ and $\phi$ rotaition gives the rotation within bc plane and ab plane, respectively.

Suppose that there is a mismatch between crystal axes and $\hat{x}^\prime, \hat{y}^\prime, \hat{z}^\prime$ directions.
Then, suppose that we find b, c axis in the direction of $(\theta,\phi) = (\frac{\pi}{2}+\theta_0,0), (\theta_0,0)$.
Direction of a-axis is still $(\theta,\phi) = (\frac{\pi}{2}, \frac{\pi}{2})$.
We can conduct the measurement within bc plane by $\theta$ rotation,
However, we need to manipulate both of $\theta$ and $\phi$ for the measurement within ab plane.
The problem is how to find the values of $\theta$ and $\phi$ for the ab-plane measurement. 

In general, if we set the rotation anlges to certain values of $(\theta,\phi)$, the direction,
%formula
\begin{align}
\label{general}
(\hat{x}^\prime, \hat{y}^\prime, \hat{z}^\prime) R^T (\theta, \phi) 
\left(
\begin{array}{c}
0 \\
0 \\
1
\end{array}
\right),
\end{align}
is oriented to be parallel to probe $\hat{z}$ direction.
(Note that, for $R$, the inverse matrix is its transposed matrix: $R^{-1} = R^T$.)
It is because, from Eq. (\ref{coordinates}), direction of Eq. (\ref{general}) is nothing but a probe $\hat{z}$ direction,
%formula.
\begin{align}
(\hat{x}^\prime, \hat{y}^\prime, \hat{z}^\prime) R^T (\theta, \phi) 
\left(
\begin{array}{c}
0 \\
0 \\
1
\end{array}
\right)
= (\hat{x}, \hat{y}, \hat{z})
\left(
\begin{array}{c}
0 \\
0 \\
1
\end{array}
\right).
\end{align}

Let's name the directions we found, $(\theta,\phi) = (\frac{\pi}{2}+\theta_0,0), (\theta_0,0). (\frac{\pi}{2}, \frac{\pi}{2})$ as b1, c1, a1 direction, respectively.
b1 direction is represented in ($\hat{x}^\prime, \hat{y}^\prime, \hat{z}^\prime$) coordinate system as, 
%\hat{b1} = formula
\begin{align}
\widehat{b1} = 
(\hat{x}^\prime, \hat{y}^\prime, \hat{z}^\prime) R^T (\theta_0 + \frac{\pi}{2}, 0) 
\left(
\begin{array}{c}
0 \\
0 \\
1
\end{array}
\right).
\end{align}
Correspondingly, a1 direction is represented in ($\hat{x}^\prime, \hat{y}^\prime, \hat{z}^\prime$) coordinate system as 
%\hat{a1} = formula
\begin{align}
\widehat{a1} = 
(\hat{x}^\prime, \hat{y}^\prime, \hat{z}^\prime) R^T (\frac{\pi}{2}, \frac{\pi}{2}) 
\left(
\begin{array}{c}
0 \\
0 \\
1
\end{array}
\right).
\end{align}
For the measurement within a1b1 plane, we have to orient the arbitrary direction $\hat{m}$ within a1b1 plane,
%\hat{m} = formula
\begin{align}
\label{m}
\hat{m} = \cos\phi^\prime\widehat{b1} + \sin\phi^\prime\widehat{a1},
\end{align}
to be parallel to probe $\hat{z}$ direction by manipulating both of $\theta$ and $\phi$.
Here $\phi^\prime$ is a rotation angle within a1b1 plane.

Here let's generalize the problem and consider the way how to orient the arbitrary direction $\hat{n}$ on ($\hat{x}^\prime, \hat{y}^\prime, \hat{z}^\prime$) coordinate system
to be parallel to probe $\hat{z}$ direction.
We introduce polar coordinate for ($\hat{x}^\prime, \hat{y}^\prime, \hat{z}^\prime$) coordinate system (Fig.\ref{polar}).
\begin{figure}
  \centering
  \includegraphics[scale=0.7]{polar.png}
  \caption{Polar coordinates for ($\hat{x}^\prime, \hat{y}^\prime, \hat{z}^\prime$).}
  \label{polar}
\end{figure}
The arbitrary direction $\hat{n}$ can be described by two angles, ($\theta_n$, $\phi_n$),
%formula
\begin{align}
\hat{n} = 
(\hat{x}^\prime, \hat{y}^\prime, \hat{z}^\prime)
\left(
\begin{array}{c}
\sin\theta_n\cos\phi_n \\
\sin\theta_n\sin\phi_n \\
\cos\theta_n
\end{array}
\right).
\end{align}
The problem is now reduced to find the rotation angle, ($\theta$, $\phi$), which satisfies,
%formula 
\begin{align}
\hat{n} = 
(\hat{x}^\prime, \hat{y}^\prime, \hat{z}^\prime)  
\left(
\begin{array}{c}
\sin\theta_n\cos\phi_n \\
\sin\theta_n\sin\phi_n \\
\cos\theta_n
\end{array}
\right)
&= (\hat{x}, \hat{y}, \hat{z}) R(\theta, \phi)
\left(
\begin{array}{c}
\sin\theta_n\cos\phi_n \\
\sin\theta_n\sin\phi_n \\
\cos\theta_n
\end{array}
\right) \notag\\
&= (\hat{x}, \hat{y}, \hat{z})
\left(
\begin{array}{c}
0 \\
0 \\
1
\end{array}
\right),
\end{align}
for arbitrary ($\theta_n$, $\phi_n$).
Here, the second equation comes from Eq. (\ref{coordinates}) and the last equation is what we have to solve.
The solution is, 
%formula
\begin{align}
\label{solution}
(\theta, \phi) = \left(-\theta_n, -\phi_n + \frac{\pi}{2}\right)
\end{align}
If ($\theta$, $\phi$) satisfies Eq. (\ref{solution}), arbitrary direction $\hat{n}$ on ($\hat{x}^\prime, \hat{y}^\prime, \hat{z}^\prime$) coordinate system
is oriented to be parallel to probe $\hat{z}$ direction.
Therefore, coming back to the original problem, we can orient $\hat{m}$ in Eq. (\ref{m}) to be parallel to probe $\hat{z}$ direction
by rewriting $\hat{m}$ as ($\theta_m$, $\phi_m$) in polar coordinate, and setting ($\theta$, $\phi$) = $\left(-\theta_m, -\phi_m+\frac{\pi}{2}\right)$.

To determine ($\theta$, $\phi$) for a1b1 rotation in programming, the above results were considered. 

Finally, $(\theta,\phi)$ is controlled by two values (A, B) in KAME interface via
%formula.
\begin{align}
\theta = \frac{A-B}{80} \\
\phi = \frac{A+B}{160}
\end{align}

\chapter{Further NMR data}
\section{Measurement along c-axis}
We also measured $T^{-1}_1$ at 4 T along c-axis and at 2.6 GPa (Fig.\ref{c-axis}).
The value in high-$T$ paramagnetic phase along c-axis is larger than that along b-axis.
There is no significant difference in dimerization transition point, $T_d = 230$ K.

\begin{figure}
  \centering
  \includegraphics[scale=0.6]{c-axis.png}
  \caption{Field-direction dependence of $T^{-1}_1$ at 2.6 GPa.
  There is no significant difference in dimerization transition point, $T_d = 230$ K.} 
  \label{c-axis}
\end{figure}

\section{Measurement for hysteresis}
Fig.\ref{hysteresis} shows $T^{-1}_1$ at 4 T along b-axis and at 3,5 GPa.
It is measured in the condition of decreasing temperature and increasing temperature around $T_d = 292$ K.
We didn't observe hysteresis by the increment of 5 K.
This means that the hysteresis is smaller than 5 K.

\begin{figure}
  \centering
  \includegraphics[scale=0.6]{hysteresis.png}
  \caption{$T^{-1}_1$ at 4 T along b-axis and at 3.5 GPa around $T_d = 292$ K.
  The hysteresis is smaller than 5 K.} 
  \label{hysteresis}
\end{figure}

\section{Condition for measurement}
Detail pulse condition for the measurement is summarized in Table.\ref{pulse}.
%Table
\begin{table}
\begin{center}
\caption{Pulse condition we used for the measurement.}
\begin{tabular}{ccc} \hline
            & 1st pulse (width, intensity)& 2nd pulse (width, intensity)\\ \hline
 ${}^7$Li   & 2 $\mu$s, -3 dB         & 2 $\mu$s, 0 dB\\ \hline
 ${}^{63}$Cu& 2 or 5 $\mu$s, -3 or -1 dB      & 2 or 5 $\mu$s, 0 dB\\ \hline
\end{tabular}
\label{pulse}
\end{center}
\end{table}
For the low-temperature measurement, lower value of master level (-15 dB) with -10 dB attenuator was enough, but as increasing temperature, we needed to increase the master level
up to -12 dB with the attenuator.
Width of comb pulse to measure $T^{-1}_1$ was 6 or 7 $\mu$s with 0 dB and the number of comb pulses was one.

We used $\tau$ listed in Table.\ref{tau}.
\begin{table}
\begin{center}
\caption{$\tau$ we used for the measurement. The first and second row are the field and pressure when the $\tau$ was used.}
\begin{tabular}{ccccc} \hline
 $B$ (T)& $P$ (GPa)& $\tau$ ($\mu$s) for ${}^7$Li& $\tau$ ($\mu$s) for ${}^{63}$Cu\\ \hline
 4& 1.3& 20& 80&\\ \hline
 4& 2.6& 20& 80&\\ \hline
 4& 3.5& 70& 160&\\ \hline
 4& 3.9& 70& 160&\\ \hline
 4& 0& 70& 180&low $T$\\ \hline
 4& 0& 60& 180&For Li1, high $T$\\ \hline
 4& 0& 40, 50& 180&For Li2, high $T$\\ \hline
 1& 0& 20& 90&$T > T_{\mathrm{mag}} = 38$ K\\ \hline
 1& 0& 14& -&$T < T_{\mathrm{mag}} = 38$ K\\ \hline
\end{tabular}
\label{tau}
\end{center}
\end{table}

For the data collection, parameters listed in Table. \ref{trig} were used.
\begin{table}
\begin{center}
\caption{Parameters for data collection.}
\begin{tabular}{cccc} \hline
                  & ${}^7$Li& ${}^{63}$Cu& ${}^7$L (1 T)\\ \hline
 From Trigger (ms)& -0.01& -0.07& -0.005\\ \hline
 Width (ms)& 0.045& 0.2& 0.045\\ \hline
\end{tabular}
\label{trig}
\end{center}
\end{table}

All NMR measurement was based on KAME interface programmed by K. Kitagawa.


\newpage
\chapter*{Acknowledgments}
\addcontentsline{toc}{chapter}{Acknowledgments}
For the initial sample preparation, I thank to T. Takayama for giving us the $\beta$-Li$_2$IrO$_3$ crystals.
I am grateful to K. Kitagawa and S. Sasaki for guiding and helping the synthesis of the improved crystals by ourselves.
I thank to N. Tateiwa for giving us the advices on using mCAC cell to measure high-pressure magnetic susceptibility.
For the high-pressure measurement of susceptibility, I am grateful to K. Kitagawa, M. Blanckenhorn and S. Sasaki for encouragement and collaboration. 
I deeply appreciate the guidance for NMR measurement and high-pressure cell for NMR by K. Kitagawa.
I am grateful for the daily discussion on the experiment with K. Kitatgawa.
I also appreciate the daily interesting, relaxing talk with all members of Takagi-Kitagawa lab. 
I am really grateful for the overall supervision by H. Takagi. 
Finally, I thank to my friends and my families for spending a joyful time together and always relaxing me.


\printbibliography[title=Reference]
\addcontentsline{toc}{chapter}{References}

%\bibliography{docter_thesis_ref}
%\bibliographystyle{phys}
%\addcontentsline{toc}{chapter}{References}
\begin{comment}
\begin{thebibliography}{99}
\addcontentsline{toc}{chapter}{References}
 \bibitem{T.Arima1993} T. Arima, Y. Tokura, and J. B. Torrance, Phys. Rev. B \textbf{48}, 17006 (1993).
 \bibitem{E.Dagotto2005} Elbio Dagotto, Science \textbf{309}, 257 (2005).
 \bibitem{B.J.Kim2008} B. J. Kim, Hosub Jin, S. J. Moon, J.-Y. Kim, B.-G. Park, C. S. Leem, Jaejun Yu, T.W. Noh, C. Kim, S.-J. Oh,
J.-H. Park, V. Durairaj, G. Cao, and E. Rotenberg, Phys. Rev. Lett. \textbf{101}, 076402 (2008).
 \bibitem{B.J.Kim2009} B. J. Kim, H. Ohsumi, T. Komesu, S. Sakai, T. Morita, H. Takagi, T. Arima, Science \textbf{323}, 1329 (2009).
 \bibitem{L.J.Sandilands2016} Luke J. Sandilands, Yao Tian, Anjan A. Reijnders, Heung-Sik Kim, K. W. Plumb, Young-June Kim, Hae-Young Kee, and Kenneth S. Burch, 
 Phys. Rev. B \textbf{93}, 075144 (2016).
 \bibitem{T.Takayama2018} T. Takayama, A. Krajewska, A. S. Gibbs, A. N. Yaresko, H. Ishii, H. Yamaoka, K. Ishii, N. Hiraoka, N. P. Funnell, C. L. Bull, and H. Takagi, arXiv:1808.0549.
 \bibitem{G.Jackeli2009} G. Jackeli and G. Khaliullin, Phys. Rev. Lett. \textbf{102}, 017205 (2009).
 \bibitem{A.Kitaev2006} A. Kitaev, Ann. Phys. (N.Y.) 321, 2 (2006).
 \bibitem{K.Kitagawa2018} K. Kitagawa, T. Takayama, Y. Matsumoto, A. Kato, R. Takano, Y. Kishimoto, S. Bette, R. Dinnebier, G. Jackeli, and
H. Takagi, Nature \textbf{554}, 341 (2018).
 \bibitem{Y-Z.You2012} Yi-Zhuang You, Itamar Kimchi, and Ashvin Vishwanath, Phys. Rev. B \textbf{86}, 085145 (2012).
 \bibitem{J.Nasu2015} Joji Nasu, Masafumi Udagawa, and Yukitoshi Motome, Phys. Rev. B \textbf{92}, 115122 (2015).
 \bibitem{J.Yoshitake2017} Junki Yoshitake, Joji Nasu, and Yukitoshi Motome, Phys. Rev. B \textbf{96}, 064433 (2017).
 \bibitem{K.O'Brien2016} Kevin O’Brien, Maria Hermanns, and Simon Trebst, Phys. Rev. B \textbf{93}, 085101 (2016).
 \bibitem{S.Mandal2009} S. Mandal and N. Surendran, Phys. Rev. B, \textbf{79}, 024426 (2009). 
 \bibitem{J.Nasu2014} Joji Nasu, Masafumi Udagawa, and Yukitoshi Motome, Phys. Rev. Lett. \textbf{113}, 197205 (2014).
 \bibitem{Y.Singh2010} Yogesh Singh and P. Gegenwart, Phys. Rev. B \textbf{82}, 064412 (2010).%Na2IrO3
 \bibitem{F.Ye2012} F. Ye, S. Chi, H. Cao, B. C. Chakoumakos, J. A. Fernandez-Baca, R. Custelcean, T. F. Qi, O. B. Korneta, and G. Cao, Phys. Rev. B \textbf{85}, 180403(R) (2012).
 \bibitem{S.H.Chun2015} Sae Hwan Chun, Jong-Woo Kim, Jungho Kim, H. Zheng, Constantinos C. Stoumpos,
C. D. Malliakas, J. F. Mitchell, Kavita Mehlawat, Yogesh Singh, Y. Choi, T. Gog, A. Al-Zein,
M. Moretti Sala, M. Krisch, J. Chaloupka, G. Jackeli, G. Khaliullin and B. J. Kim, Nat. Phys. \textbf{11}, 462, (2015).
 \bibitem{Y.Singh2012} Yogesh Singh, S. Manni, J. Reuther, T. Berlijn, R. Thomale, W. Ku, S. Trebst, and P. Gegenwart, Phys. Rev. Lett. \textbf{108}, 127203 (2012).
 %alpha-Li2IrO3 & Na2IrO3
 \bibitem{S.C.Williams2016} S. C. Williams, R. D. Johnson, F. Freund, Sungkyun Choi, A. Jesche, I. Kimchi,
S. Manni, A. Bombardi, P. Manuel, P. Gegenwart, and R. Coldea, Phys. Rev. B \textbf{93}, 195158 (2016).
 \bibitem{M.Majumder2015} M. Majumder, M. Schmidt, H. Rosner, A. A. Tsirlin, H. Yasuoka, and M. Baenitz, Phys. Rev. B \textbf{91}, 180401(R) (2015).
 \bibitem{A.Banerjee2017} Arnab Banerjee, Jiaqiang Yan, Johannes Knolle, Craig A. Bridges,
Matthew B. Stone, Mark D. Lumsden, David G. Mandrus, David A. Tennant,
Roderich Moessner, Stephen E. Nagler, Science \textbf{356}, 1055 (2017).
 \bibitem{Y.Kasahara2018} Y. Kasahara, T. Ohnishi, Y. Mizukami, O. Tanaka, Sixiao Ma, K. Sugii, N. Kurita, H. Tanaka, J. Nasu, Y. Motome,
T. Shibauchi and Y. Matsuda, Nature \textbf{559}, 227 (2018).
 \bibitem{S.-H.Baek2017} S.-H. Baek, S.-H. Do, K.-Y. Choi, Y. S. Kwon, A. U. B. Wolter, S. Nishimoto, Jeroen van den Brink, and B. B\"uchner, 
 Phys. Rev. Lett. \textbf{119}, 037201 (2017). %alpha-RuCl3 high-field NMR
 \bibitem{K.A.Modic2014} K. A. Modic, T. E. Smidt, I. Kimchi, N. P. Breznay, A. Biffin, S. Choi,
R. D. Johnson, R. Coldea, P. Watkins-Curry, G. T. McCandless, J. Y. Chan,
F. Gandara, Z. Islam, A. Vishwanath, A. Shekhter, R. D. McDonald, and J. G. Analytis, Nat. Commun. \textbf{5}, 4203 (2014).
 \bibitem{A.Biffin2014_2} A. Biffin, R. D. Johnson, I. Kimchi, R. Morris, A. Bombardi, J. G. Analytis, A. Vishwanath, and R. Coldea, Phys. Rev. Lett. \textbf{113}, 197201 (2014).
 \bibitem{K.A.Modic2017} K.A. Modic, B.J. Ramshaw, J.B. Betts, Nicholas P. Breznay, James G. Analytis, Ross D. McDonald, and Arkady Shekhter, Nat. Commun. \textbf{8}, 180 (2017).
 \bibitem{N.P.Breznay2017} Nicholas P. Breznay, Alejandro Ruiz, Alex Frano, Wenli Bi, Robert J. Birgeneau, Daniel Haskel, 
 and James G. Analytis, Phys. Rev. B \textbf{96}, 020402(R) (2017).%gamma-Li2IrO3 under high P
 \bibitem{L.S.I.Veiga2017} L. S. I. Veiga, M. Etter, K. Glazyrin, F. Sun, C. A. Escanhoela, Jr., G. Fabbris,2 J. R. L. Mardegan,
P. S. Malavi, Y. Deng, P. P. Stavropoulos, H.-Y. Kee, W. G. Yang, M. van Veenendaal, J. S. Schilling,
T. Takayama, H. Takagi, and D. Haskel, Phys. Rev. B. \textbf{96}, 140402(R) (2017) and its Supplemental Material.
 \bibitem{T.Takayama2015} T. Takayama, A. Kato, R. Dinnebier, J. Nuss, H. Kono, L. S. I. Veiga, G. Fabbris, 
 D. Haskel, and H. Takagi, Phys. Rev. Lett. \textbf{114}, 077202 (2015).
 \bibitem{A.Glamazda2016} A. Glamazda, P. Lemmens, S.-H. Do, Y.S. Choi, and K.-Y. Choi, Nat. Commun. \textbf{7}, 12286 (2016).
 \bibitem{A.Biffin2014} A. Biffin, R. D. Johnson, S. Choi, F. Freund, S. Manni, A. Bombardi, P. Manuel, P. Gegenwart, and R. Coldea,
  Phys. Rev. B. \textbf{90}, 205116 (2014).
 \bibitem{E.K.H.Lee2015} Eric Kin-Ho Lee and Yong Baek Kim, Phys. Rev. B, \textbf{91}, 064407 (2015).
 \bibitem{J.Nasu2014_2} J. Nasu and Y. Motome, Phys. Rev. B \textbf{90}, 045102 (2014).%quadrupolar
 \bibitem{V.M.Katukuri2016} Vamshi M. Katukuri, Ravi Yadav, Liviu Hozoi, Satoshi Nishimoto, and Jeroen van den Brink, Sci. Rep. \textbf{6}, 29585 (2016).
 \bibitem{M.Majumder2018} M. Majumder, R. S. Manna, G. Simutis, J. C. Orain, T. Dey, F. Freund, A. Jesche, R. Khasanov, P. K. Biswas, E. Bykova, N. Dubrovinskaia, L. S. Dubrovinsky,
  R. Yadav, L. Hozoi, S. Nishimoto, A. A. Tsirlin, and P. Gegenwart, Phys. Rev. Lett. \textbf{120}, 237202 (2018).
 \bibitem{R.Yadav2018} Ravi Yadav, Stephan Rachel, Liviu Hozoi, Jeroen van den Brink, and George Jackeli, Phys. Rev. B \textbf{98}, 121107(R) (2018).
 \bibitem{H-S.Kim2016} Heung-Sik Kim, Yong Baek Kim, and Hae-Young Kee, Phys. Rev. B \textbf{94}, 245127 (2016).
 \bibitem{I.Rousochatzakis2017} Ioannis Rousochatzakis and Natalia B. Perkins, Phys. Rev. Lett. \textbf{118}, 147204 (2017).
 \bibitem{F.Freund2016} F. Freund, S. C. Williams, R. D. Johnson, R. Coldea, P. Gegenwart, and A. Jesche, Sci. Rep. \textbf{6}, 35362 (2016).
 \bibitem{N.Tateiwa2011} N. Tateiwa, Y. Haga, Z. Fisk, and Y. $\bar{\mathrm{O}}$nuki, Rev. Sci. Instrum. \textbf{82}, 053906 (2011).
 \bibitem{Daphne} Successor of Daphne 7474 (K. Murata, K. Yokogawa, H. Yoshino, S. Klotz, P. Munsch, A. Irizawa, M. Nishiyama, K. Iizuka,
T. Nanba, T. Okada, Y. Shiraga, and S. Aoyama, Rev. Sci. Instrum., \textbf{79}, 085101 (2008).
 \bibitem{Y.Kitaoka} 北岡良雄, 共鳴型磁気測定の基礎と応用 (内田老鶴圃, 2014)
 \bibitem{K.Asayama} 朝山邦輔, 遍歴電子系の核磁気共鳴:金属磁性と超伝導 (裳華房, 2002)
 \bibitem{H.Takigawa} 瀧川仁, 核磁気共鳴とその固体物理学への応用 (物性若手夏の学校全体講義テキスト, 2009)
 \bibitem{K.Kitagawa2010} K. Kitagawa, H. Gotou, T. Yagi, A. Yamada,
 T. Matsumoto, Y. Uwatoko, and M. Takigawa, J. Phys. Soc. Jpn., \textbf{79}, 024001 (2010).
 \bibitem{J.Zheng2017} Jiacheng Zheng, Kejing Ran, Tianrun Li, Jinghui Wang, Pengshuai Wang, Bin Liu, Zheng-Xin Liu, B. Normand, Jinsheng Wen, and Weiqiang Yu, 
 Phys. Rev. Lett. \textbf{119}, 227208 (2017).
 \bibitem{P.W.Anderson1973} P. W. Anderson, Mater. Res. Bull. \textbf{8}, 153-160 (1973).
 \bibitem{P.G.Radaelli2002} Paolo G. Radaelli, Y. Horibe, Matthias J. Gutmann, Hiroki Ishibashi, C. H. Chen, Richard M. Ibberson, Y. Koyama,
Yew-San Hor, Valery Kiryukhin and Sang-Wook Cheong, Nature (London) 416, 155 (2002).
 \bibitem{S.Tsuji1997} S. Tsuji, K. Kumagai, N. Matsumoto and S. Nagata, Physica C 282-287 (1997) 1107-1108.
 \bibitem{D.I.Khomskii2005} D. I. Khomskii and T. Mizokawa, Phys. Rev. Lett. \textbf{94}, 156402 (2005)
 \bibitem{G.L.Pascut2014} G. L. Pascut, K. Haule, M. J. Gutmann, S. A. Barnett, A. Bombardi, S. Artyukhin, T. Birol,
D. Vanderbilt, J. J. Yang, S.-W. Cheong, and V. Kiryukhin, Phys. Rev. Lett. \textbf{112}, 086402 (2014).
 \bibitem{K.Mizuno2002} Kiyoshi Mizuno, Ko-ichi Magishi, Yasuaki Shinonome, Takahito Saito, Kuniyuki Koyama, Nobuhiro Matsumoto, and Shoichi Nagata, Physica B 312–313 (2002) 818–819.
 \bibitem{M.Isobe1996} M. Isobe and Y. Ueda, J. Phys. Soc. Jpn. \textbf{65}, 1178 (1996).
 \bibitem{T.Ohama1997} T. Ohama, M. Isobe, H. Yasuoka, and Y. Ueda, J. Phys. Soc. Jpn. \textbf{66}, 545 (1997).
 \bibitem{V.Hermann2018} V. Hermann, M. Altmeyer, J. Ebad-Allah, F. Freund, A. Jesche, A. A. Tsirlin, M. Hanfland, P. Gegenwart, I. I. Mazin, D. I. Khomskii, R. Valent\`i,
  and C. A. Kuntscher, Phys. Rev. B \textbf{97}, 020104(R) (2018).%alpha-Li2IrO3 under high P
 \bibitem{G.Bastien2018} G. Bastien, G. Garbarino, R. Yadav, F. J. Martinez-Casado, R. Beltr\`an Rodr\`iguez, Q. Stahl, M. Kusch, S. P. Limandri, R. Ray, P. Lampen-Kelley, 
 D. G. Mandrus, S. E. Nagler, M. Roslova, A. Isaeva, T. Doert, L. Hozoi, A. U. B. Wolter, B. B\"uchner, J. Geck, and J. van den Brink, Phys. Rev. B \textbf{97}, 241108(R) (2018).
 %RuCl3 magnetization under high P
 \bibitem{V.Hermann2017} V. Hermann, J. Ebad-Allah, F. Freund, I. M. Pietsch, A. Jesche, A. A. Tsirlin, J. Deisenhofer, M. Hanfland, P. Gegenwart, 
 and C. A. Kuntscher, Phys. Rev. B \textbf{96}, 195137 (2017). %Na2IrO3 under high P
 \bibitem{X.Xi2018} Xiaoxiang Xi, Xiangyan Bo, X. S. Xu, P. P. Kong, Z. Liu, X. G. Hong, C. Q. Jin, G. Cao, Xiangang Wan, and G. L. Carr, Phys. Rev. B \textbf{98}, 125117 (2018).
 %Na2IrO3 under high P
 \bibitem{J.Kanamori} 金森順次郎, 磁性 (培風館, 1992).
\end{thebibliography}
\end{comment}
\begin{comment}
\bibitem{J.A.Sears2015} J. A. Sears, M. Songvilay, K. W. Plumb, J. P. Clancy, Y. Qiu, Y. Zhao, D. Parshall, and Young-June Kim, Phys. Rev. B \textbf{91}, 144420 (2015).%alpha-RuCl3
\end{comment}
%\` %Just because of unusual number of tables stacked at end
%%\bibliography{document} 


\end{document}
