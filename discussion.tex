\chapter{Discussion}
\section{Quantum critical point at ambient pressure}
The size of the low-$T$ magnetic moment at 4 T is $\sim$ 0.4 $\mu_B$ (Fig.\ref{Bdep_mag}), which is still less than 1.6 $\mu_B$ at paramagnetic phase.
The magnetic moments are not fully polarized at 4 T.
This means that there is still a magnetically fluctuating component in the high-field regime.
In fact, the magnetization is not saturated at 4 T (Fig.\ref{Bdep_mag}).
The problem is whether the fluctuating component can be understood as spin liquid or not.
In the case of field-induced spin liquid, $\alpha$-RuCl$_3$, the size of the low-$T$ magnetic moment at 8 T is $\sim$ 0.55 $\mu_B$ \cite{Zheng2017}.
In that sense, there is a possibility that the unpolarized component in $\beta$-Li$_2$IrO$_3$ can be a spin liquid.
It has a spin gap of 90 $\pm$ 20 K (Fig.\ref{spin_gap}(b)).

Here, spin gap fitting was done with the function,
\begin{equation}
T^{-1}_1 = b\mathrm{e}^{-\Delta/T},
\end{equation}
where $b$ and spin gap $\Delta$ are fitting parameters, and $T$ is temperature.
Temperature range used for fitting is summarized in Table.\ref{Trange}.
For the pressure of 2.6, 3.5, 3.9 GPa, we estimated the spin gap by the fitting to the decrease from the intermidiate value to almost zero value.

For $\alpha$-RuCl$_3$, there are two reports that one claim the gapless quantum spin liquid \cite{Zheng2017} and the other claim the gapful quantum spin liquid \cite{baek2017evidence}.
However, the common feature is that it become gapless around the quantum critical point.
This might mean that the gapped behavior we observed at 4 T is too far away from the quantum critical point.
In $\alpha$-RuCl$_3$, exotic behavior like quantized thermal Hall effect is only reported around quantum critical point $\sim 8$ T (Fig.\ref{phase_BT}(b)) \cite{kasahara2018majorana}.
Our result might suggest that, in order to observe quantized thermal Hall effect in $\beta$-Li$_2$IrO$_3$, we need to finely adjust the field between 2.7 T and 4 T
(Fig.\ref{phase_BT} (a)).
On the other hand, quantized thermal Hall effect might be originating from two dimensionality of spin liquid state.
If it is the case, we cannot observe quantized thermal Hall effect in three dimensional $\beta$-Li$_2$IrO$_3$.
Instead, if 3D Kitaev spin liquid is realized in $\beta$-Li$_2$IrO$_3$ at the field between 2.7 and 4 T, there should be a topological phase transition from paramagnetic state to
spin liquid state \cite{nasu2014vaporization}.

Thus, our result gives the upper bound (4 T) to observe the possible field-induced exotic quantum criticality in $\beta$-Li$_2$IrO$_3$.

\begin{figure}
  \centering
  \includegraphics[scale=0.7]{spin_gap_both.png}
  \caption{(a) Spin gap fitting for $T^{-1}_1$.
  (b) Pressure dependence of the spin gap.
  Spin gap $\Delta$ is enhanced for the dimerization above 3.5 GPa.}
  \label{spin_gap}
\end{figure}

\begin{table}[H]
\begin{center}
\caption{Temperature range used for spin gap fitting.}
\begin{tabular}{ccc} \hline
 $P$ (GPa)& $T_\mathrm{start}$ (K)& $T_\mathrm{end}$ (K)\\ \hline
 0 & 20 (lowest)& 70\\ \hline
 1.3& 40 (lowest)& 110\\ \hline
 2.6& 20 (lowest)& 225\\ \hline
 3.5& 160& 250\\ \hline
 3.9& 180& 290\\ \hline
\end{tabular}
\label{Trange}
\end{center}
\end{table}

\begin{figure}
  \centering
  \includegraphics[scale=0.7]{phase_BT.png}
  \caption{Field-temperature phase diagram (a) for $\beta$-Li$_2$IrO$_3$ \cite{ruiz2017correlated}
  and (b) for $\alpha$-RuCl$_3$ \cite{kasahara2018majorana}.
  Quantum critical behavior is observed in $\alpha$-RuCl$_3$ in the red shaded limited region.
  It might also be true for $\beta$-Li$_2$IrO$_3$.
  Region with dashed circle below 4 T have a possibility for quantum critical behavior.
  In (a), we added the words, "measured", "spiral AFM", and "Quantum critical behavior?".}
  \label{phase_BT}
\end{figure}

\section{Strong singlet dimer formation above 3.5 GPa}
Now we examine the three possibilities for the non-magnetic state of the dimer (Fig.\ref{three}).
Knight shift in our singlet dimer state at 3.5 GPa is almost zero (Fig.\ref{3.5GPa}).
It means almost zero susceptibility in the dimer state.

Fig.\ref{chi_cal} is a calculated magnetic susceptiblity $\chi$ for the dimer of $J_{\mathrm{eff}} = 1/2$ spins \cite{Nasu2014}.
$t'$ represents the direct $dd$ hopping integral in the scale of indirect $dpd$ hopping.
$t' < 1$ means stronger indirect $dpd$ hopping and it stabilizes quadrupolar state.
We can see that the quadrupolar state has a finite $\chi$.
Thus, quadrupolar state is excluded because we observed almost zero susceptibility in the dimer state.

Then, we examine the possibilty of $J_{\mathrm{eff}} = 1/2$ spin singlet.
In order to form $J_{\mathrm{eff}} = 1/2$ spin singlet, we need AF Heisenberg interaction.
However, the first principle calculation revealed that Heisenberg interaction is FM even under pressure \cite{Yadav2018, Kim2016}.
This makes the $J_{\mathrm{eff}} = 1/2$ spin singlet scenario less plausible.
There is another negative discussion for the scenario.
Suppose that $J_{\mathrm{eff}} = 1/2$ spin singlet is formed by exchange interaction.
There should be the singlet-triplet gap of the order of the interaction.
In $\beta$-Li$_2$IrO$_3$, the calculation revealed that the exchange interactions are of the order of $\sim$ 100 K \cite{Yadav2018}.
The gap should give a finite van Vleck term in $\chi$ and correspondingly finite $K$.
In fact, the calculation of magnetic susceptibility $\chi$ for $J_{\mathrm{eff}} = 1/2$ spin singlet revealed that $\chi$ has finite value at the temperature scale of
the energy gap (in our case $\sim 100$ K), while it decays to zero value at lower temperature \cite{Nasu2014}.
In Fig.\ref{chi_cal}, $J_{\mathrm{eff}} = 1/2$ singlet state corresponds to $t' > 1$ case, which means stronger direct $dd$ hopping.
However, we observed almost zero $K$ for singlet dimer phase.
The discrepancy means that $J_{\mathrm{eff}} = 1/2$ spin singlet is not the case.

\begin{figure}
  \centering
  \includegraphics[scale=0.7]{chi_cal.png}
  \caption{Calculated magngetic susceptibility $\chi$ for dimer state of $J_{\mathrm{eff}} = 1/2$ spins.
  $\beta$ is inverse temperature. \cite{Nasu2014}
  (a) $\chi$
  (b) $\mathrm{d}\chi/\mathrm{d}\beta$
  $t'$ represents the direct $dd$ hopping integral in the scale of indirect $dpd$ hopping.
  $t' < 1$ means quadrupolar state and $t' > 1$ means singlet state.
  The singlet state has a finite peak structure at the temperature scale of singlet-triplet gap.}
  \label{chi_cal}
\end{figure}

Other possibility is the collapse of $J_{\mathrm{eff}} = 1/2$ picture and the formation of molecular orbital.
Fig.\ref{band_cal_highP} is the calculated density of states (DOS) for ambient-pressure $Fddd$ structure and for high-pressure $C/2c$ structure at 4.4 GPa (Fig.\ref{band_cal_highP})
\cite{takayama2018pressure}.
At ambient pressure, we can observe the splitting of $t_{2g}$ states to $J_{\mathrm{eff}} = 1/2$ and $J_{\mathrm{eff}} = 3/2$ states.
On the other hand, at high-pressure $C/2c$ phase, $d_{xy}$, $d_{yz}$, $d_{zx}$ character appears.
The lowest occupied sub-band (-1.7 eV) and the highest unoccupied sub-band (0.7 eV) are dominated by $d_{zx}$ character.
It means the formation of bonding-antibonding state of $d_{zx}$ orbital.
$d_{zx}$ orbital is directed along dimerized bond (Y bond).
The degenerate $d_{xy}$, $d_{yz}$ orbitals are entangled by SOC and reside between -1.7 eV and 0.7 eV.
Thus, the dimer state is a molecular orbital of $d_{zx}$ orbital.
The gap induced by the formation of the bonding-antibonding state is estimated as $\sim$ 7000 K.
This is consistent with our zero $K$ because such a large excitation gap gives negligible van Vleck susceptibility.
If we assume the similar behavior of $\chi$ to $J_{\mathrm{eff}} = 1/2$ spin singlet, $\chi$ has a finite value at $\sim 7000$ K and decays to zero at lower temperature,
which should give the almost zero $\chi$ in our measuring temperature range.

Thus, our results support the molecular orbital formation for the non-magnetic dimer state.

Also, the large spin gap $\sim 2000$ K (Fig.\ref{spin_gap}) above 3.5 GPa is consistent with our observeation.
The strong singlet dimer formation is further supported by the high transition temperature reaching to room temperature and the fact that it is first-order transition.

$J_{\mathrm{eff}} = 1/2$ state is obtained by equally mixing up the $d_{xy}$, $d_{yz}$, $d_{zx}$ orbitals with multiplying a complex factor.
We can view the transition from $J_{\mathrm{eff}} = 1/2$ state to $d_{zx}$-molecular orbital as the pressure-induced orbital selection.

\begin{figure}
  \centering
  \includegraphics[scale=0.7]{band_cal_highP.png}
  \caption{Calculated density of states (DOS) for Ir. \cite{takayama2018pressure}
  (a) DOS for the ambient-pressure structure ($Fddd$).
  $t_{2g}$ states split to $J_{\mathrm{eff}} = 1/2$ and $J_{\mathrm{eff}} = 3/2$ states.
  (b) DOS for the high-pressure structure at 4.4 GPa ($C/2c$).
  $d_{zx}$ orbital forms bonding-antibonding state.
  $d_{xy}$ and $d_{yz}$ are entangled by SOC.
  Total DOS contains O $2p$ states.}
  \label{band_cal_highP}
\end{figure}


XRD and neutron scattering revealed that the zigzag chains compress easier than the bridging bonds (Fig.\ref{bond}) \cite{veiga2017pressure, takayama2018pressure}.
The first principle calculation suggest that it gives the larger enhancement of direct hopping along zigzag chain than one in bridging bond \cite{Kim2016}.
It is considered that the anisotropic enhancement of hopping contributes to the stabilization and formation of $d_{zx}$-molecular orbital.

\begin{figure}
  \centering
  \includegraphics[scale=0.7]{bond.png}
  \caption{Pressure dependence of bond length \cite{veiga2017pressure, takayama2018pressure}. X, Y bond composing zigzag chain compresses easier than bridging Z bond.}
  \label{bond}
\end{figure}

\section{Competition between spin-orbit couping and dimer formation}
As we discussed in Chap.\ref{result}, there is a coexisting phase of $J_\mathrm{eff} = 1/2$ spins and singlet dimer in the intermidiate pressure range (1 GPa $< P <$ 3 GPa)
(Fig.\ref{pressure_phase}).
In Chap.\ref{result}, we presented the interpretaion that singlet dimer state in the intermidiate region (1.3 and 2.6 GPa) is affected by defects.
However, we can also have a speculation that the dimer state at 1.3 and 2.6 GPa is different from molecular orbital dimer.
Quadrupolar dimer state can still be excluded because it cannot be continuously connected to $S = 1/2$ molecular dimer state above 3.5 GPa.
They belong to different symmertries.
Here, we mean the possibility of $J_\mathrm{eff} = 1/2$ pseudo-spin singlet at 1.3 and 2.6 GPa.
It can be continuously connected to $S = 1/2$ molecular dimer state above 3.5 GPa.
In this view, finite $K$ for Li$_\mathrm{d}$ peak at 1.3 GPa (Fig.\ref{spectrum_highP}(d)) is originated from van Vleck term of modest singlet-triplet gap.
One of the coexisting component at 2.6 GPa (Fig.\ref{spectrum_highP}(e)) is also understood as $J_\mathrm{eff} = 1/2$ pseudo-spin singlet with finite $K$.
However, XRD at high pressure and at low temperature is now revealing that the structure of dimer at lower pressure region is the same as one at higher pressure region
(private communication).
This suggests the same dimer state for the overall $P-T$ phase diagram.
The conclusive results from high-pressure XRD is awaited.

What we found in $\beta$-Li$_2$IrO$_3$ under pressure is a competition between spin-orbit Mott state and molecular dimer state.
The former is characterized by the energy gain of SOC and Coulomb $U$, while the latter is characterized by the energy gain of hopping amplitude $t$.
If there is no SOC in $\beta$-Li$_2$IrO$_3$, it shows dimerization in Z-bond \cite{Kim2016}.
Note that, in reality, the dimerization in $\beta$-Li$_2$IrO$_3$ occurs in Y-bond.
$\alpha$-RuCl$_3$ also shows dimerization without SOC \cite{Kim2016a}.
These mean that there is an intrinsic instability for orbital-selective dimerization among these Kitaev magnets and strong SOC prevents the dimerization
by mixing the orbitals at ambient pressure.
Pressure modifies the subtle balance by increasing hopping integrals, resulting in the transition from $J_\mathrm{eff} = 1/2$ state to dimerized state.
On the other hand, Li$_2$RuO$_3$ ($4d^4$) shows dimerizatoin under ambient pressure \cite{Miura2007}.
This might mean that not only SOC but also spin-orbit Mottness of $J_\mathrm{eff} = 1/2$ state is a key to prevent dimerization.

If the pressure is moderate and the system is just in a competitive region, there should be the incomplete dimer formation.
In that region, the pairing partner for dimer is not fully determined.
Fluctuation to find the dimer pair can be regarded as the similar state to resonating valence bond (RVB) state \cite{ANDERSON1973153}.
In this sense, there might be RVB-like spin liquid state in the competitive region.

\begin{figure}[H]
  \centering
  \includegraphics[scale=0.7]{pressure_phase.png}
  \caption{Schematic image of pressure dependence of the phase in $\beta$-Li$_2$IrO$_3$.
  Singlet dimer in the intermidiate pressure region is affected by defect or different from strong singlet dimer at higher pressure.}
  \label{pressure_phase}
\end{figure}

\section{Effect of defect and $\mu$SR}
Fig. \ref{lowQ} shows the crystal-quality dependence of $\chi_b$ and $T^{-1}_1$.
Crystal 1 is supplied by a collaborator.
Crystal 2 is obtained by the method described in Chap.\ref{method}.
The kink at $T_\mathrm{mag} = 38$ K disappears for crystal 1 (Fig.\ref{lowQ}(a)).
It is considered that the large amount of defect disturbs ICAF order for crystal 1.
The effect of defects is more prominently observed in $T^{-1}_1$ (Fig.\ref{lowQ}(b)).
$T^{-1}_1$ for crystal 1 at 3.2 GPa and at 7 T doesn't show non-magnetic transition.
Instead, it shows peak structure around 40 K.
The low-$T$ value is two order of magnitude larger than that for crystal 2 at 3.5 GPa.
The peak structure is also observed for crystal 1 at 3 T.
This means the anomalous behavior in $T^{-1}_1$ is originating not from the difference in applied field but from the sample quality.
It is considered that the high concentration of defects in crystal 1 produces dynamical spins.

$\mu$SR at high pressure concluded that there is a coexistence of liquid-like dynamical spins and glass-like frozen spins
in the pressure range from 1.37 GPa to 2.27 GPa below $T_{\mathrm{mag}}$ = 38 K \cite{Majumder2018}.
In the $P$-$T$ region, our measurement showed the coexistence of ICAF ordered spins, singlet dimers and defects.
The discrepancy is possibly caused by the amount of defect.
It is considered that the large amount of defect in ICAF order phase make it close to glassy state
as we observed the diffuse of hyperfine field for paramagnetic component (or ICAF order moments in low-field regime) at 1.3 GPa and at low temperature.
Also, the high concentration of defect in singlet dimer phase can produce dynamical spins.
In fact, $T^{-1}_1$ measured for low-quality sample at 3.2 GPa was two-order of magnitude larger than that at 3.5 GPa (Fig.\ref{lowQ}(b)).
This indicates the existence of dynamical spins can be strongly dependent on sample quality.

\begin{figure}[H]
  \centering
  \includegraphics[scale=0.7]{lowQ.png}
  \caption{Effect of sample quality.
  (a) Magnetic susceptibility at 1 T along b-axis of $\beta$-Li$_2$IrO$_3$ single crystals.
  The high-quality crystal, crystal 2, has higher and clearer magnetic transition point, $T_\mathrm{mag} = 38$ K, than crystal 1.
  (b) Spin-lattice relaxation rate, $T^{-1}_1$.
  Non-magnetic transition is absent for low-quality crystal, crystal 1.
  Instead the large value of $T^{-1}_1$ is observed in crystal 1, indicating the existence of defect-induced dynamical spins.}
  \label{lowQ}
\end{figure}


\section{Comparison with other compounds}
There are not so many iridates which shows valence bond solid.
CuIr$_2$S$_4$ shows structural transition with octamer formation at $T_s \sim 230$ K and at ambient pressure \cite{Radaelli2002}.
Spin-lattice realaxation rate, $T^{-1}_1$, in Cu-NMR shows abrupt drop at the transition point (Fig.\ref{T1_CuIr2S4}) \cite{Tsuji1997}.
Knight shift also change the sign at the transition point from negative above $T_s$  to positive below $T_s$ (Fig.\ref{K_CuIr2S4}) \cite{Tsuji1997}.
Knight shift below $T_s$ has finite value possibly due to orbital contribution.
This is a metal-insulator transition.
There is a discussion about mechanism that one-dimensional hopping originating from anisotropy of $t_{2g}$ orbitals induces Peierls instability and octamer formation
(orbitally-induced Peierls mechanism) \cite{Khomskii2005}, while it ignores the effect of SOC.
Our singlet dimer formation in $\beta$-Li$_2$IrO$_3$ is insulator-to-insulator transition, so it is different from CuIr$_2$S$_4$.
However, also in $\beta$-Li$_2$IrO$_3$, pressure induce anisotropic compression and anisotropic enhancement of hopping, especially in zigzag chain
\cite{takayama2018pressure, veiga2017pressure, Kim2016}.
This might mean that the one-dimensionality in hopping is induced by pressure, resulting in Peierls instability and singlet dimer formation within zigzag chain.
This pressure-induced one-dimensionality scenario is one possibility to explain the singlet dimer formation in $\beta$-Li$_2$IrO$_3$.

\begin{figure}
  \centering
  \includegraphics[scale=0.7]{T1_CuIr2S4.png}
  \caption{$T^{-1}_1$ of CuIr$_2$S$_4$ ($x = 0$) \cite{Tsuji1997}. It shows abrupt drop at $T_s \sim 230$ K.}
  \label{T1_CuIr2S4}
\end{figure}

\begin{figure}
  \centering
  \includegraphics[scale=0.7]{K_CuIr2S4.png}
  \caption{Knight shift of CuIr$_2$S$_4$ ($x = 0$) \cite{Tsuji1997}. Sudden sign change is observed at $T_s \sim 230$ K.}
  \label{K_CuIr2S4}
\end{figure}

IrTe$_2$ is another example which shows dimer formation at $T_s \sim 280$ K and at ambient pressure \cite{Pascut2014}.
Spin-lattice realaxation rate, $T^{-1}_1$, in ${}^{125}$Te-NMR shows abrupt drop at the transition point (Fig.\ref{T1_IrTe2}).
This is a metal-to-metal transition.
Both of CuIr$_2$S$_4$ and IrTe$_2$ are mixed-valnent system with formal Ir$^{3.5+}$ valnece \cite{Radaelli2002, Pascut2014}.
Charge ordering accompanies with octamer or dimer formation for both compounds.
In that sense, singlet dimer formation in $\beta$-Li$_2$IrO$_3$ without charge fluctuation is rare case among dimerization in iridates.

\begin{figure}
  \centering
  \includegraphics[scale=0.7]{T1_IrTe2.png}
  \caption{$T^{-1}_1$ of IrTe$_2$ \cite{Mizuno2002}. It shows abrupt drop at $T_s \sim 280$ K.}
  \label{T1_IrTe2}
\end{figure}

$\alpha'$-NaV$_2$O$_5$ shows dimer formation of V$^{4+}$ at $T_{SP} = $ 34 K and at ambient pressure (spin-Peierls transitoin) \cite{Isobe1996}.
Here, spin-Peierls transition is a singlet dimer formation observed in one-dimensional $S = 1/2$ spin system, which is induced by AF Heisenberg interaction.
The absolute value of Knight shift in ${}^{23}$Na-NMR shows the reduction of almost 1/10 from high-$T$ paramagnetic state to low-$T$ singlet dimer state (Fig.\ref{K_NaV2O5}).
In our singlet dimer formation in $\beta$-Li$_2$IrO$_3$, the reduction of Knight shift from paramagnetic to singlet state is almost 1/100 at 3.5 GPa (Fig.\ref{3.5GPa}(b)).
The large reduction in $\beta$-Li$_2$IrO$_3$ indicates that the dimer formation is stronger than spin-Peierls transition in $\alpha'$-NaV$_2$O$_5$.
In other words, the excitaion gap induced by singlet dimer formation in $\beta$-Li$_2$IrO$_3$ is far larger than singlet-triplet gap in $\alpha'$-NaV$_2$O$_5$.
This comparison also guarantees that the singlet dimer in $\beta$-Li$_2$IrO$_3$ originates from molecular orbital dimer, not from singlet induced by exchange interaction.

\begin{figure}
  \centering
  \includegraphics[scale=0.7]{K_NaV2O5.png}
  \caption{Knight shift of $\alpha'$-NaV$_2$O$_5$ \cite{Ohama1997}.
  Field is along b-axis.
  The absolute value of Knight shift decrease at $T_{SP} \sim 34$ K.
  The amount of the decrease is smaller than that at $T_\mathrm{d} = 292$ K in $\beta$-Li$_2$IrO$_3$ under 3.5 GPa.}
  \label{K_NaV2O5}
\end{figure}

The pressure-induced lattice dimerization was also reported for other Kitaev magnets, $\alpha$-Li$_2$IrO$_3$ \cite{Hermann2018}, $\alpha$-RuCl$_3$ \cite{Bastien2018}.
Magnetism of the high-$P$ phase in $\alpha$-Li$_2$IrO$_3$ is still unknown though (Fig.\ref{phase_PT}(b)).
For $\alpha$-RuCl$_3$, the high-pressure magnetism was revealed to be a singlet dimer and similar phase diagram to $\beta$-Li$_2$IrO$_3$ was discovered (Fig.\ref{phase_PT}(a)).
It is considered that similar mechanism of molecular orbital formation is valid also for the dimerization in $\alpha$-RuCl$_3$.
For Na$_2$IrO$_3$, it is controversial whether it has a structural phase transition at high pressure \cite{Hermann2017, Xi2018}.
The discrepancy possibly comes from the air- and moisture-sensitivity of the compound.
However, at least, the reported transition is moderate and is not the dimerization \cite{Xi2018}.
$\gamma$-Li$_2$IrO$_3$ do not show any structural phase transition under pressure \cite{breznay2017resonant}.
The magnetic order disappears at $\sim$ 1.4 GPa without any lattice distortion.
These indicate that there is a general tendency of pressure-induced singlet dimer formation in Kitaev magnets,
but the appearance is highly sensitive to chemical composition or structure.
It is necessary to reveal the whole $P-T$ phase diagram for all the Kitaev magnets and to elucidate the condition of appearance or disappearance of singlet dimer formation.

\begin{figure}
  \centering
  \includegraphics[scale=0.7]{phase_PT.png}
  \caption{Pressure-temperature phase diagram (a) for $\alpha$-RuCl$_3$ \cite{Bastien2018} and (b) for $\alpha$-Li$_2$IrO$_3$ \cite{Hermann2018}.
  Phase diagram for $\alpha$-RuCl$_3$ is similar to one for $\beta$-Li$_2$IrO$_3$.}
  \label{phase_PT}
\end{figure}

\chapter{Summary}
Since the discovery of $J_\mathrm{eff} = 1/2$ pseudo-spin in Sr$_2$IrO$_4$ \cite{kim2008novel, kim2009phase},
interplay between SOC and correlation has provided the fascinating research area.
One of the most intensively studied themes is a realizatoin of Kiteav spin liquid in $J_\mathrm{eff} = 1/2$ pseudo-spin system \cite{kitaev2006anyons, jackeli2009mott}.
For Kitaev spin liquid, there are various exotic theoretical predition such as application to quantum computing \cite{kitaev2006anyons},
realization of fractional excitation \cite{nasu2015thermal, yoshitake2017temperature}
and appearance of exotic superconductor by hole doping \cite{you2012doping}.
$\beta$-Li$_2$IrO$_3$ was expected to realize the Kitaev spin liquid under high pressure.
It is because magnetic order of $J_\mathrm{eff} = 1/2$ at ambient pressure is suppressed by pressure without metallization \cite{takayama2015hyperhoneycomb}.
Also, there is a $\mu$SR report that the high-pressure phase is a mixture of spin liquid and spin glass  \cite{Majumder2018}.
Thus, we started the study of high-pressure magnetism of $\beta$-Li$_2$IrO$_3$.

We synthesized the largest $\beta$-Li$_2$IrO$_3$ single crystals.
At ambient pressure, we confirmed that the magnetic order in low-field regime becomes ferromagnetically polarized paramagnet in high-field regime.
We measured the high-pressure magnetism up to 3.9 GPa through magnetic susceptibility and ${}^7$Li NMR.
We revealed $P$-$T$ phase diagram.
The high-pressure phase is a non-magnetic singlet dimer probably based on molecular orbital.
Also, we concluded that spin-liquid component observed in $\mu$SR is defect-induced dynamical moments in singlet dimer background.

We revealed that there is a pressure-induced competition between orbital-disordered $J_{\mathrm{eff}} = 1/2$ state and orbital-selective molecular dimer state in $\beta$-Li$_2$IrO$_3$.
In the competitive region, pairing partner of dimer fluctuates and RVB-like spin liquid might realize.
The competition seems to be a common feature among some Kitaev magnets.
Uncovering of $P$-$T$ phase diagram for all Kitaev candidates is awaited.
