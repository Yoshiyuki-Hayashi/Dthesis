\chapter{Introduction}
Historically, band theory succeeded to explain the physics of weakly correlated $s$ or $p$ electron system like alkhali metal or semiconducting behavior of Si or Ge.
The interest of physicists pursuing the exotic quantum phenomena, then, focused on surveying the effect of strong Coulomb correlation.
The target was $3d$ electron system, where strong correlation originated from well-localized $3d$ orbital was expected ($U \sim 5$eV for NiO \cite{bengone2000implementation}).
For example, Mott/charge-transfer insulator \cite{arima1993variation}, colossal magnetoresistance in manganites \cite{dagotto2005complexity}
and high-$T_C$ superconducting cuprate \cite{dagotto2005complexity} are the topics in 3d electron system.
These phenomena are beyond the conventional band theory or mean-field theory.
They stimulated the theoretical development to correctly include strong Coulomb $U$ such as dynamical mean-field theory \cite{georges1996dynamical}.
In that sense, $4d, 5d$ electron systems had been forgot because the effect of correlation was considered to be weaker in the extended $4d, 5d$ orbitals compared to $3d$.


The situation is drastically changed from the discovery of layered perovskite Sr$_2$IrO$_4$ \cite{kim2008novel, kim2009phase}.
In Sr$_2$IrO$_4$, octahedral crystal field split $5d$ orbital into $t_{2g}$ and $e_g$ orbitals.
Metallic state was expected since five electrons fill the widely-spread $t_{2g}$ orbital.
However, it is actually a magnetic insulator.
The insulating state is induced by the interplay of strong spin-orbit coupling (SOC) ($\lambda_{SO} \sim 0.3$ eV \cite{kim2008novel}, 0.01 eV for 3d ion \cite{Kanamori})
and modest Coulomb $U$($U \sim 2$eV for Sr$_2$IrO$_4$ \cite{arita2012ab}).
Strong SOC mixes up $S = 1/2$ of the hole and the effective angular momentum of $t_{2g}$ orbital, $L_\mathrm{eff} = -1$ (Fig.\ref{five_t2g}).
It gives the splitting of $t_{2g}$ orbitals into completely filled $J_{\mathrm{eff}} = 3/2$ quartet and half-filled $J_{\mathrm{eff}} = 1/2$ doublet.
The band width of $t_{2g}$ orbitals is reduced to that of $J_{\mathrm{eff}} = 1/2$ state by SOC.
In the narrow $J_{\mathrm{eff}} = 1/2$ state, the modest value of Coulomb $U$ is enough to open the Mott gap.
Thus, by reducing the band width, strong SOC enhances the effect of Coulomb $U$, which makes $4d$, $5d$ systems nontrivial spin-orbit Mott insulators.

The existence of $J_{\mathrm{eff}} = 1/2$ state in Sr$_2$IrO$_4$ was established by resonant X-ray scattering (RXS) \cite{kim2009phase}.
Fig.\ref{Sr2IrO4} shows the RXS intensity for L$_2$ and L$_3$ edge of Ir.
In $S = 1/2$ model, both of L$_2$ and L$_3$ edge should be enhanced, while in $J_\mathrm{eff} = 1/2$ model, only L$_3$ edge is enhanced.
The enhancement in RXS intensity is only observed in L$_3$ edge, indicating the validity of $J_\mathrm{eff} = 1/2$ model in Sr$_2$IrO$_4$.

%Fig. Energy level splitting of Ir$^{4+}$ or Ru$^{3+} ion.
\begin{figure}
  \centering
  \includegraphics[scale=0.7]{five_t2g.png}
  \caption{Energy level splitting of Ir$^{4+}$ or Ru$^{3+}$ ion under octahedral crystal field.
  $\Delta_{\mathrm{cubic}}$ is the strength of crystal field splitting.
  Angular momentum of $d$ orbital ($L$ = 2) reduces to $L_{\mathrm{eff}}$ = -1 in $t_{2g}$ manifold.
  Strong SOC mixes $L_{\mathrm{eff}}$ = -1 and $S = \frac{1}{2}$ of one hole and gives $J_{\mathrm{eff}} = \frac{1}{2}$ state.
  $\lambda_{\mathrm{SO}}$ is the strength of SOC.
  $\Delta_{\mathrm{cubic}}$ is of the order of 3.3 eV for Sr$_2$IrO$_4$ \cite{ishii2011momentum}, 2.2 eV for RuCl$_3$ \cite{sandilands2016spin}
  and 3.5 eV for $\beta$-Li$_2$IrO$_3$ \cite{takayama2018pressure}.
  Splitting $\frac{3}{2}\lambda_{\mathrm{SOC}}$ is of the order of 0.4 eV for Sr$_2$IrO$_4$ \cite{kim2008novel} 0.14 eV for RuCl$_3$ \cite{sandilands2016spin}
  and 0.7 eV for $\beta$-Li$_2$IrO$_3$ \cite{takayama2018pressure}.}
  \label{five_t2g}
\end{figure}

\begin{figure}
  \centering
  \includegraphics[scale=0.7]{Sr2IrO4.png}
  \caption{(A) Solid lines are X-ray absorption spectra. The dotted red lines is the intensity of magnetic (1 0 22) peak.
  (B) Equal resonant enhancement for L$_2$ and L$_3$ edge is expected in $S = 1/2$ model. In contrast, the enhancement occurs only in L$_3$ edge in $J_\mathrm{eff} = 1/2$ model
  \cite{kim2009phase}.}
  \label{Sr2IrO4}
\end{figure}

In conventional $S = 1/2$ Mott insulator, the interaction between spins is isotropic AF Heisenberg interaction.
However, in the spin-orbit $J_\mathrm{eff} = 1/2$ Mott insulator, the interaction between $J_\mathrm{eff} = 1/2$ pseudo-spins can be strongly anisotropic \cite{jackeli2009mott}
and it depends on how the octahedra are connected.
First, let's consider two $J_\mathrm{eff} = 1/2$ spins in corner-shared octahedra (Fig.\ref{180_90bond}(a)).
In this case, isotropic Heisenberg interaction remains.
Interation between $J_\mathrm{eff} = 1/2$ spins is
\begin{equation}
\mathcal{H}_{ij} = J_1\bm{S}_i\cdot\bm{S}_j + J_2(\bm{S}_i\cdot\bm{r}_{ij})(\bm{r}_{ij}\cdot\bm{S}_j),
\end{equation}
where $\bm{S}_{i(j)}$ is  $J_{\mathrm{eff}} = 1/2$ spin at site $i(j)$, $\bm{r}_{ij}$ is a unit vector along $ij$ bond.
$J_1$, $J_2$ is a coupling constant and the isotropic Heisenberg interaction, $J_1$, is more dominant.
On the other hand, let's consider two $J_\mathrm{eff} = 1/2$ pseudo-spins in edge-shared anion octahedra (Fig.\ref{180_90bond}(b)).
In this case, as shown in Fig.\ref{180_90bond}, we can consider two indirect virtual hopping paths mediated by anion,
$d_{yz}$-$p_z$ (upper anion)-$d_{xz}$ and $d_{xz}$-$p_z$ (lower anion)-$d_{yz}$.
It is because the $J_{\mathrm{eff}} = 1/2$ state is a spin-orbital-entangled wavefunction including $d_{xz}$ and $d_{yz}$ \cite{jackeli2009mott}:
\begin{align}
\ket{J^z_{eff} = \pm \frac{1}{2}} = \frac{1}{\sqrt{3}}(\ket{d_{xy}, \pm}\pm\ket{d_{yz}, \mp}+i\ket{d_{xz}, \pm}),
\label{Jeff}
\end{align}
where signs in $\ket{}$ of the right-hand side denotes the spin state.
The two contribution interfere in destructive manner and the isotropic exchange interaction vanishes.
This is a striking difference from 180$^\circ$ bond situation.
Only the anisotropic FM Ising interaction called Kitaev interaction,
\begin{equation}
\mathcal{H}^{(\gamma)}_{ij} = -KS^\gamma_iS^\gamma_j,
\label{Kitaev}
\end{equation}
remains, where $\bm{S}_{i(j)}$ is  $J_{\mathrm{eff}} = 1/2$ spin at site $i(j)$, $\gamma$ is a direction perpendicular to $ij$ bond and sharing edge, and $K$ is positive.
The magnetic easy axis of this FM Ising interaction, $\gamma$ direction, is bond-dependent.
This feature is key ingredient to realize Kitaev model in $J_{\mathrm{eff}} = 1/2$ pseudo-spins.

\begin{figure}
  \centering
  \includegraphics[scale=0.7]{180_90bond.png}
  \caption{Virtual hopping path for two $J_{\mathrm{eff}} = \frac{1}{2}$ spins in (a) corner-shared anion octahedra and (b) edge-shared anion octahedra \cite{jackeli2009mott}.
  Site $i, j$ is the position of two $J_{\mathrm{eff}} = \frac{1}{2}$ spins and small circle is anion $X$.
  In (a), Heisenberg interaction reamains, but in (b), two virtual $dpd$ hoppings via upper or lower anion gives Kitaev interaction.}
  \label{180_90bond}
\end{figure}

Here, Kitaev model is the model where one-half spins are aligned in honeycomb lattice and the Kitaev interaction (Eq.\ref{Kitaev}) is given for each bond (Fig.\ref{Kitaev_model})
\cite{kitaev2006anyons}.
In Hamiltonian, it can be written as,
%formula.
\begin{equation}
\mathcal{H} = -K_x\sum_\mathrm{X-bond} S^x_iS^x_j - K_y\sum_\mathrm{Y-bond} S^y_iS^y_j - K_z\sum_\mathrm{Z-bond} S^z_iS^z_j,
\label{Kitaev_H}
\end{equation}
where $K_x$, $K_y$, $K_z$ is positive, $\bm{S}$ is one-half spin, and X, Y, Z bonds are denoted in Fig.\ref{Kitaev_model}.
Since the interactions on three bonds try to orient the spin to three different direction, it gives the frustration.
The frustration disturbs the magnetic order and gives spin liquid in the ground state.

We can exactly solve Kitaev model as follows.
First, let's rewrite one-half spin in Kitaev model as,
%formula
\begin{equation}
S^\gamma_j = \frac{i}{2}b^\gamma_jc_j
\end{equation}
Here, $\gamma = x,y,z$ and $b^\gamma_j$, $c_j$ is Majorana fermions.
$j$ represents the site index.
This means that $\bm{S}_j$ is represented by four Majorana fermions.
One is itinerant Majorana fermion, $c_j$, and the other are localized ones, $b^x_j, b^y_j, b^z_j$.
Generally, Majorana fermions $a_j$ is an operator which satisfies,
\begin{equation}
\{a_i, a_j\} = 2\delta_{ij}, \vspace{4mm} a^\dag_i = a_i.
\end{equation}
$S^\gamma_j$ actually behaves one-half spins if we assume one constrain for Majorana fermions,
\begin{equation}
D_j = b^x_jb^y_jb^z_jc_j = 1.
\end{equation}
Then, Kitaev model (Eq.\ref{Kitaev_H}) can be rewritten as,
\begin{equation}
\mathcal{H} = \frac{i}{4}\sum_{j,k}A_{jk}c_jc_k.
\label{hopping}
\end{equation}
Here, $A_{jk}$ is defined as,
\begin{equation}
A_{jk} = \begin{cases}
2K_{\alpha_{jk}}u_{jk} & \text{if j and k are connected}\\
0 & \text{otherwise}
\end{cases}
\end{equation}
$u_{jk}$ is defined as,
\begin{equation}
u_{jk} = ib^{\alpha_{jk}}_jb^{\alpha_{jk}}_k,
\end{equation}
where $\alpha_{jk} = x, y, z$ if $jk$ is X, Y, Z-bond.
$u_{jk}$ is a constant ($\pm 1$) since it commutes with $\mathcal{H}$ and satisfies $u^2_{jk} = 1$.
Thus, Kitaev model (Eq.\ref{Kitaev_H}) is mapped to nearest neighbor hopping model of Majorana fermions coupled with $Z_2$ variables.
The product of $u_{jk}$ around hexagon, $W$, is also conserved quantity with the value of $\pm1$.
Numerical calculation or Lieb's theorem \cite{kitaev2006anyons} tell that the ground state is obtained if $W = 1$ for all hexagons.
This means that one of the ground state is achieved if all $u_{jk}$ are equal to one.
Thus, the model (Eq.\ref{hopping}) reduces to just a nearest neighbor hopping model on honeycomb lattice.
It gives the Dirac cone dispersion of itinerant Majorana fermion $c_j$, similarly to graphene.

For the Kitaev spin liquid, variety of novel phenomena have been predicted: application to topologaical quantum computing \cite{kitaev2006anyons},
realization of exotic superconductor by hole-doping \cite{you2012doping}, and fractional excitation by Majorana fermions \cite{nasu2015thermal, yoshitake2017temperature}.

The essential ingredients for Kitaev model is three 120$^\circ$ bond structure.
This means that we can consider the extension of Kitaev model on the lattice different from honeycomb at least if the lattice has three 120$^\circ$ bond structure.
Theorists invented a lot of such extension of Kitaev model \cite{o2016classification}.
One of them is 3D hyperhoneycomb lattice (Fig.\ref{hyperhoneycomb}).
The hyperhoneycomb structure can be constructed from the three-dimensional stack of two-dimensional honeycomb layers,
by dividing the honeycomb lattice into zigzag chains and the bridging bonds, rotating the zigzag chains alternatively and reconnecting them with those in the upper and lower planes.
Kitaev model on hyperhoneycomb lattice also shows spin liquid ground state with nodal line of itinerant Majorana fermion \cite{mandal2009exactly}.
Because of the three dimensionality, it shows topological phase transition to spin liquid ground state \cite{nasu2014vaporization}.

\begin{figure}
  \centering
  \includegraphics[scale=0.7]{Kitaev_model.png}
  \caption{Kitaev model on honeyomcb lattice \cite{kitaev2006anyons, kitagawa2018spin}.
  For each X, Y, Z bond, Kitaev interaction $\mathcal{H}^{(x)}$,$\mathcal{H}^{(y)}$,$\mathcal{H}^{(z)}$ (e.q.(\ref{Kitaev})) is defined.
  Each bond has different magnetic easy axis and frustrates.}
  \label{Kitaev_model}
\end{figure}

\begin{figure}
  \centering
  \includegraphics[scale=0.7]{hyperhoneycomb.png}
  \caption{Hyperhoneycomb lattice \cite{nasu2014vaporization}. a, b, c represent primitive translation vectors.}
  \label{hyperhoneycomb}
\end{figure}

Kiteav model was originally toy model.
However, since it was suggested that $J_{\mathrm{eff}} = 1/2$ pseudo-spins in edge-shared anion octahedra can show the Kitaev interaction \cite{jackeli2009mott},
the materialzation of Kitaev model become realistic.
It triggered the intensive material reseach on honeycomb-based Ir$^{4+}$ or Ru$^{3+}$ compounds pursuing Kitaev spin liquid.

Na$_2$IrO$_3$ has 2D honeycomb-layered structure of Ir$^{4+}$.
It is an insulator with optical gap $\sim 340$ meV, which can be explained only by LDA + SO + $U$, not LDA + SO \cite{comin20122}.
This means that Na$_2$IrO$_3$ is spin-orbit Mott insulator.
The Ir$^{4+}$ ions are surrounded by edge-shared O$^{2-}$ octahedra, so the building blocks for Kiteav model are provided.
However, it has AF interaction with Curie-Weiss temperature $\theta_{\mathrm{CW}} = - 116$ K \cite{singh2010antiferromagnetic}
and shows zigzag order of $J_{\mathrm{eff}} = 1/2$ pseudo-spins \cite{ye2012direct}.
It is considered that additional AF interaction exists and makes the iridate far from ideal FM Kitaev spin liquid.
On the other hand, the dominance of bond-directional interaction was revealed by diffuse magnetic X-ray scattering \cite{chun2015direct}.
For Na$_2$IrO$_3$, instead of $J_{\mathrm{eff}} = 1/2$ picture, quasi-molecular orbital (QMO) picture is proposed \cite{mazin20122}.
In QMO picture, electrons delocalize within Ir hexagon, which is in contrast to atomically localizing $J_{\mathrm{eff}} = 1/2$ picture.
Other calculation indicates that anisotropic interaction induced by trigonal distortion of O$^{2-}$ octahedra as well as SOC play a key role to stabilze zigzag order
\cite{yamaji2014first}.
In that sense, uniaxial strain to reduce trigonal distortion might be helpful to reach quantum spin liquid in Na$_2$IrO$_3$.


$\alpha$-Li$_2$IrO$_3$ has the same structure as Na$_2$IrO$_3$.
Again, it has AF interaction ($\theta_{\mathrm{CW}} = - 33$ K \cite{singh2012relevance}),
and shows non-collinear order \cite{williams2016incommensurate}.
However, if we substitute Li between the honeycomb layers of $\alpha$-Li$_2$IrO$_3$ to H,
the resulting compound, $($H$_{\frac{3}{4}}$Li$_{\frac{1}{4}}$)$_2$IrO$_3$
or namely H$_3$LiIr$_2$O$_6$, shows no magnetic order.
The spin liquid behavior was confirmed down to 50 mK by specific heat measurement.
The interaction is still AF for H$_3$LiIr$_2$O$_6$ ($\theta_{\mathrm{CW}} = -105$ K) \cite{kitagawa2018spin}, suggesting the deviation from ideal FM Kitaev spin liquid.
In fact, the characteristic behavior for Kitaev spin liquid like fractional excitation have not been reported yet.
Instead, fermionic excitation based on impurities was captured by NMR relaxation measurement \cite{kitagawa2018spin}.
Theorists revealed that configuration of interlayer hydrogen modifies the interaction between pseudo-spins via H-O bonding \cite{li2018role}.
Zero-point fluctution of hydrogen is considered as a key to understand spin-liquid behavior in H$_3$LiIr$_2$O$_6$ \cite{li2018role}.

$\alpha$-RuCl$_3$ is one of the most intensively studied Kitaev candidates.
It has Ru$^{3+}$ honeycomb-layered structure surrounded by edge-shared Cl$^-$ octahedra.
Photoemission spectroscopy confirmed that it is Mott insulator with substaintial SOC \cite{koitzsch2016j}.
The sign of Curie-Weiss temperature is dependent on field direction: $\theta_{\mathrm{CW}} = 37$ K ($H \parallel$ ab-plane) and
$\theta_{\mathrm{CW}} = -150$ K ($H \parallel$ c-axis) \cite{majumder2015anisotropic}.
In calculation, Kitaev term is FM and 3-5 times larger than AF Heisenberg \cite{yadav2016kitaev}.
However, surpringly, inelastic neutron scattering revealed the existence of AF Kitaev interaction, not FM Kitaev interaction \cite{banerjee2017neutron}.
It shows zigzag AF order at low temperature, but the order can be suppressed under the field within ab-plane above $\sim 7$ T \cite{kasahara2018majorana}.
In the high-field regime, the spin liquid behavior is confirmed by NMR measurement \cite{baek2017evidence}.
Anomolous thermal quantum Hall effect, which can be regarded as emergence of Majorana fermions, is also reported in the high-field spin liquid state of $\alpha$-RuCl$_3$
\cite{kasahara2018majorana}.

$\gamma$-Li$_2$IrO$_3$ \cite{modic2014realization} has 3D stripy honeycomb structure of Ir$^{4+}$.
The FM Curie-Weiss behavior in high temperature region is reported.
It shows the complex spiral order \cite{biffin2014noncoplanar}.
Also for $\gamma$-Li$_2$IrO$_3$, field above $\sim 3$ T suppresses the magnetic order and liquid state is proposed \cite{modic2017robust},
while the microscopic confirmation like NMR has not been done yet.
On the other hand, high-pressure $\sim 1.4$ GPa suppress the order without structural phase transition \cite{breznay2017resonant}.
The nature of high-pressure phase still remains unclear.

%Fig. crystal structure (a) honeyocmb (b) stripy (c) beta-Li2IrO3
\begin{figure}
  \centering
  \includegraphics[scale=0.7]{lattices2.png}
  \caption{(a) Crystal structure of 2D honeycomb-layered H$_3$LiIr$_2$O$_6$ \cite{kitagawa2018spin}.
  (b) 3D stripy honeycomb structure of $\gamma$-Li$_2$IrO$_3$ \cite{modic2014realization}.
  (c) Crystal structure of 3D hyperhoneycomb $\beta$-Li$_2$IrO$_3$ \cite{takayama2015hyperhoneycomb}.}
  \label{lattices}
\end{figure}

Our material, $\beta$-Li$_2$IrO$_3$ \cite{takayama2015hyperhoneycomb}, is one of such Kitaev candidates.
$\beta$-Li$_2$IrO$_3$ crystallizes in orthorhombic structure with the space group $F_{ddd}$ at ambient pressure.
Ir$^{4+}$ ions have a spin-orbital-entangled $J_\mathrm{eff}$ = 1/2 pseudo-spin due to the octahedral crystal field from surrounding O$^{2-}$ ions
as well as the strong spin-orbit coupling.
The Ir ions form hyperhoneycomb sub-lattice, which can be viewed as a three-dimensional analogue of two-dimensional honeycomb lattice (Fig. \ref{lattices}(c)).
The hyperhoneycomb lattice is surrounded by edge-shared O$^{2-}$ octahedra.
It means the existence of Kitaev interaction between Ir$^{4+}$ ions.
Since the hyperhoneycomb lattice shares 120$^\circ$ bond structure with honeycomb lattice, the frustration of Kitaev interactions also arises.
It leads spin liquid state theoretically \cite{mandal2009exactly}.
There are two crystallographically inequivalent Li sites.
Li2 is surrounded by pseudo-honeycomb five-membered ring of Ir, while Li1 is located outside the pseudo-honeycomb ring.

Fig.\ref{band_cal} shows band calculation for $\beta$-Li$_2$IrO$_3$ under ambient pressure \cite{veiga2017pressure}.
We can see the splitting of $J_\mathrm{eff} = 1/2$ band and $J_\mathrm{eff} = 3/2$ band.
$J_\mathrm{eff} = 1/2$ band lies at fermi level, which means the physics of $\beta$-Li$_2$IrO$_3$ is governed by $J_\mathrm{eff} = 1/2$ pseudo-spins.

\begin{figure}
  \centering
  \includegraphics[scale=0.7]{band_cal.png}
  \caption{Band calculation for $\beta$-Li$_2$IrO$_3$ under ambient pressure \cite{veiga2017pressure}. $U_\mathrm{eff} = U - J_H = 2.5$ eV, where $U$ is on-site Coulomb repulsion
  and $J_H$ is Hund coupling.}
  \label{band_cal}
\end{figure}

Fig. \ref{beta0GPa} (a) shows the magnetic susceptibility $\chi = M/B$ at 1 T for polycrystalline sample of $\beta$-Li$_2$IrO$_3$ \cite{takayama2015hyperhoneycomb}.
The high-$T$ Curie-Weiss fit yields the effective moment $p_{\mathrm{eff}} \sim$ 1.61 $\mu_B$, which is close to 1.73 $\mu_B$ of ideal $J_{\mathrm{eff}} = 1/2$ moment,
and the ferromagnetic Curie-Weiss constant $\theta_{CW} \sim 40$ K.
The ferromagnetically interacting $J_{\mathrm{eff}} = 1/2$ spins indicate the proximity to ideal Kitaev model in $\beta$-Li$_2$IrO$_3$.
Also, Raman spectroscopy suggests the emergence of Majorana fermionic excitation originating from Kitaev model \cite{glamazda2016raman}.
However, $\chi$ shows steep increase below 50 K and have a kink at $T_\mathrm{mag} = 38$ K.
The kink corresponds to the appearance of incommensurate non-coplanar antiferromagnetic (ICAF) order, confirmed by neutron scattering \cite{Biffin2014}.
Fig. \ref{beta0GPa}(b) shows the magnetic structure for ICAF order phase.
The theoretical investigations suggest that the existence of additional interaction different from nearest neighbor FM Kitaev interaction
can explain the realization of the spiral order in $\beta$-Li$_2$IrO$_3$ \cite{Lee2015}.
For example, direct $dd$ hoppings between the Ir$^{4+}$ ions gives isotropic Heisenberg interaction,
%formula
\begin{equation}
\mathcal{H}_{ij} = J\bm{S}_i\cdot\bm{S}_j,
\end{equation}
where $\bm{S}_{i(j)}$ is  $J_{\mathrm{eff}} = 1/2$ spin at site $i(j)$.
The combination of $dd$ and $dpd$ hopping gives symmetric off-diagonal interaction \cite{Nasu2014},
%formula
\begin{equation}
\mathcal{H}_{ij} = \Gamma_{xy} (S^x_iS^y_j + S^y_iS^x_j).
\label{symoff}
\end{equation}
Also, the importance of next nearest neighbor AF Heisenberg interaction is pointed out \cite{Katukuri2016}.
These interactions disturb Kitaev spin liquid state and stabilize the ICAF order phase in $\beta$-Li$_2$IrO$_3$.

%Fig. magnetic susceptibilty
\begin{figure}
  \centering
  \includegraphics[scale=0.7]{beta0GPa.png}
  \caption{Physical property of $\beta$-Li$_2$IrO$_3$ under ambient pressure.
  (a) Temperature dependence of magnetic susceptibility $\chi$ and $\chi^{-1}$ (inset) at 1 T for polycrystalline sample\cite{takayama2015hyperhoneycomb}.
  It shows the magnetic order below $T_{\mathrm{AF}} = 38$ K.
  (b) Magnetic structure in the ordered phase revealed by neutron scattering \cite{Biffin2014}}
  \label{beta0GPa}
\end{figure}

However, there are two ways to suppress the ICAF order --- applying high field along b-axis above $\sim$ 2.7 T
and applying high pressure above $\sim$ 2 GPa \cite{takayama2015hyperhoneycomb}.
In the case of high field, the kink corresponding to ICAF order vanishes (Fig. \ref{Bdep_mag} (inset)),
and system remains paramagnetic with field induced moment of $\sim$ 0.35 $\mu_B$ (Fig. \ref{Bdep_mag}).
The size of magnetic moment in ICAF order phase is 0.47 $\mu_B$, which is close to the value of the field-induced moment \cite{Biffin2014}.
Thus, the high field destroys the ICAF order by ferromagnetically orienting $J_{\mathrm{eff}} = 1/2$ moment along the field direction.
On the other hand, magnetic x-ray scattering revealed that high-field paramagnetic state admixes zigzag component along a-axis \cite{ruiz2017correlated}.
There is also a possibility that the high-field state can be a spin liquid state.
Thus, our first motivation is to explore the high-field state of $\beta$-Li$_2$IrO$_3$ by NMR measurement.

\begin{figure}
  \centering
  \includegraphics[scale=0.7]{magnetization.png}
  \caption{Field dependence of magnetization of $\beta$-Li$_2$IrO$_3$ under ambient pressure.
  (inset) Field depnedence of magnetic susceptibility $\chi$ \cite{takayama2015hyperhoneycomb}.
  The kink of ICAF order vanishes at high field.}
  \label{Bdep_mag}
\end{figure}

In the case of high pressure, the ICAF order is suppressed without metallization.
Fig. \ref{beta2GPa} (a) shows the pressure dependence of X-ray Magnetic Circular Dichroism (XMCD) at 5 K, 4 T.
XMCD corresponds to bulk magnetization.
The field-induced moment is suppressed around 2 GPa.
On the other hand, the resistivity shows the insulating state (Fig. \ref{beta2GPa} (a) (inset)).
The disappearance of ICAF order without metallization indicates the possible pressure-induced Kitaev spin liquid in $\beta$-Li$_2$IrO$_3$.
Also, $\mu$SR found the coexistence of liquid-like dynamic spins and glass-like frozen spins at $\sim$ 2 GPa (Fig. \ref{beta2GPa} (b)) \cite{Majumder2018}.
First principle calculation suggested the enhancement of symmetric off-diagonal term (Eq.\ref{symoff}) as applying pressure in $\beta$-Li$_2$IrO$_3$ \cite{Yadav2018, Kim2016}.
In the model with the dominance of symmetric off-diagonal term, classical spin liquid instability is proposed \cite{Rousochatzakis2017}.
Therefore, our second motivation is to test the spin-liquid component with different time scale, NMR relaxation measurement.

\begin{figure}
  \centering
  \includegraphics[scale=0.7]{beta2GPa.png}
  \caption{Evolution of field-induced moment or magnetic order in $\beta$-Li$_2$IrO$_3$ as pressurizing.
  (a) Pressure dependence of X-ray Magnetic Circular Dichroism (XMCD) at 5 K, 4 T and resistivity (inset) \cite{takayama2015hyperhoneycomb}.
  XMCD reflects bulk magnetization.
  Around 2 GPa, the ICAF order is surppressed, but the state is insulating, which excludes metallization.
  (b) Phase diagram based on $\mu$SR \cite{Majumder2018}.
  Coexistence of frozen glassy spins and dynamic spins is proposed in the order-surppressed pressure region around 2 GPa.}
  \label{beta2GPa}
\end{figure}

On the other hand, the larger pressure around 4 GPa induces a structural phase transition at room temperature for $\beta$-Li$_2$IrO$_3$.
Fig. \ref{beta4GPa} (a) is pressure dependence of lattice parameter $\beta$ obtained by powder and single-crystal XRD at room temperature \cite{veiga2017pressure}.
Abrupt decrease of the lattice parameter is observed around 4 GPa.
It corresponds to structural phase transition from $F_{ddd}$ struccture to $C2/c$ structure.
The high-pressure $C2/c$ structure has dimerized Ir-Ir bonds, which is 12 \% shorter than other bonds (Fig. \ref{beta4GPa}(b)).

There are three possibilities for the non-magnetic state of dimer in $J_{\mathrm{eff}} = 1/2$ system (Fig. \ref{three}).
The first one is $J_{\mathrm{eff}} = 1/2$ spin singlet based on the spin-orbital-entangled wave function.
This is conventional spin singlet,
\begin{equation}
\frac{1}{\sqrt{2}}(\ket{\uparrow\downarrow}-\ket{\downarrow\uparrow}),
\end{equation}
where $\uparrow, \downarrow$ means $J_{\mathrm{eff}} = 1/2$ up state and down state.
It is realized when direct $dd$ hopping, $t_{dd}$, is larger than indirect $dpd$ hopping, $t'_{dpd}$
as long as $J_{\mathrm{eff}} = 1/2$ picture is valid.
The singlet-triplet excitation gap $\sim t^2_{dd}/U$ is relatively small.
Here $U$ is on-site Coulomb repulsion.

The second one is quadrupolar state of $J_{\mathrm{eff}} = 1/2$ spins \cite{Nasu2014}.
It can be written as,
%formula
\begin{equation}
\ket{\psi^-_Q} = \frac{1}{\sqrt{2}}(\ket{\uparrow\uparrow}-i\ket{\downarrow\downarrow}),
\label{quad_m}
\end{equation}
where $\uparrow, \downarrow$ means $J_{\mathrm{eff}} = 1/2$ up state and down state.
It is stabilized when the symmetric off-diagonal interaction (Eq.(\ref{symoff})) exists and the $t'_{dpd}$ is larger than $t_{dd}$.
The state has a finite van Vleck term.
If the symmetric off-diagonal interaction (Eq.(\ref{symoff})) doesn't exist, the state (Eq.\ref{quad_m}) is degenerate with another quadrupolar state,
\begin{equation}
\ket{\psi^+_Q} = \frac{1}{\sqrt{2}}(\ket{\uparrow\uparrow}+i\ket{\downarrow\downarrow}).
\label{quad_p}
\end{equation}

The last one is band-insulator-like non-magnetic dimer based on molecular orbital.
This is a singlet of $S = 1/2$.
It is realized when $t_{dd}$ is larger than $U$.
The $J_{\mathrm{eff}} = 1/2$ wavefunction collapses in this limit.
The bonding-antibonding excitation gap $\sim t$ is relatively large.
Thus, our third motivation is to judge which non-magnetic state is realzied in the high-pressure dimer phase via NMR measurement.

\begin{figure}
  \centering
  \includegraphics[scale=0.7]{beta4GPa2.png}
  \caption{Structural property of $\beta$-Li$_2$IrO$_3$ at high pressure \cite{veiga2017pressure}.
  (a) Pressure dependence of lattice parameter $\beta$.
  Structural phase transition occurs around 4 GPa at room temperature
  (b) Structure in the high pressure $C2/c$ phase.
  Emphasized Y bond become 12 \% shorter than X, Z bond (X : 3.0139(12) \AA, Y : 2.6620(13) \AA, Z : 3.0122(15) \AA). }
  \label{beta4GPa}
\end{figure}

\begin{figure}
  \centering
  \includegraphics[scale=0.7]{three_possibilities.eps}
  \caption{Three possiblities for the non-magnetic state of dimer in $J_{\mathrm{eff}} = 1/2$ system.}
  \label{three}
\end{figure}

Here, we completed the $P$-$T$ phase diagram of $\beta$-Li$_2$IrO$_3$ (Fig. \ref{phase_pre}).
We revealed that the high-$P$ phase is non-magnetic singlet dimer.
It is probably based on molecular orbital.
The high-$T$ high-$P$ singlet dimer formation is connected to low-$T$ low-$P$ magnetic collapse.
From the continuity of the phase, we concluded that spin-liquid component observed in $\mu$SR is dynamical defect-induced moments embedded in singlet dimer phase.
Our results suggest that the pressure-induced competition between orbitally-disordered $J_{\mathrm{eff}} = 1/2$ magnetism and orbital-selective molecular dimer
is a ubiquitous phenomenon among some Kitaev magnets.

\begin{figure}[h]
  \centering
  \includegraphics[scale=1.0]{phase19.png}
  \caption{Pressure-temperature phase diagram for $\beta$-Li2IrO3.
  PM, ICAF means paramagnetic phase, incommensurate non-coplanar antiferromagnetic phase, respectively.
  Dashed region is coexisting phase.
  $\times$ is from chi at 1 T along b-axis and $\bullet$ is from 4 T NMR.
  $\circ$ is from chi at 1 T for polycrystalline sample \cite{Majumder2018} and $\blacktriangledown$ comes from XRD \cite{veiga2017pressure}.
  Note that ICAF order is destroyed and substituted to PM at 4 T. (Inset) crystal structures for each phase.
  In $F_{ddd}$ phase at ambient pressure, all bonds have almost equivalent lengths \cite{takayama2015hyperhoneycomb}.
  In $C2/c$ phase, the emphasized bonds are 12 \% shorter than other bonds \cite{veiga2017pressure}.}
  \label{phase_pre}
\end{figure}
